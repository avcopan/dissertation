\chapter[%
	Ground-State Density Cumulant Theory:\\
	Thermochemical and Kinetic Benchmark Calculations
]{%
	Ground-State Density Cumulant Theory:\\
	Thermochemical and Kinetic Benchmark Calculations\footnote{%
        A.~V.~Copan, A.~Yu.~Sokolov, and H.~F.~Schaefer.,
        J.~Chem.~Theory~Comput.
        {\bfseries 10},
        2389
        (2014).
        Adapted with permission of the American Chemical Society.
    }
}
\label{ch:benchmark}

\section{Abstract}
We present an extensive benchmark study of density cumulant functional theory
(DCFT) for thermochemistry and kinetics of closed- and open-shell molecules. The
performance of DCFT methods (DC-06, DC-12, ODC-06, and ODC-12) is compared to
that of coupled-electron pair methods (CEPA$_0$ and OCEPA$_0$) and
coupled-cluster theory (CCSD and CCSD(T)) for the description of noncovalent
interactions (A24 database), barrier heights of hydrogen-transfer reactions
(HTBH38), radical stabilization energies (RSE30), adiabatic ionization energies
(AIE), and covalent bond stretching in diatomic molecules. Our results indicate
that out of four DCFT methods the ODC-12 method is the most reliable and
accurate DCFT formulation to date. Compared to CCSD, ODC-12 shows superior
results for all benchmark tests employed in our study. With respect to
coupled-pair theories, ODC-12 outperforms CEPA$_0$, and shows similar accuracy
to the orbital-optimized CEPA$_0$ variant (OCEPA$_0$) for systems at equilibrium
geometries. For covalent bond stretching, ODC-12 is found to be more reliable
than OCEPA$_0$. For the RSE30 and AIE datasets, ODC-12 shows competitive
performance with CCSD(T). In addition to benchmark results, we report new
reference values for the RSE30 dataset computed using coupled cluster theory
with up to perturbative quadruple excitations.


\section{Introduction}

Recent developments in {\itshape ab initio} quantum chemistry have resulted in a
variety of computational models for studying molecules.
Apart from concerns about efficiency and accuracy, several concepts have evolved
as criteria for judging the merits of a particular method.
Energy-based criteria typically define an ``ideal'' approximation as one
yielding correlation energies that are size-consistent,
extensive\cite{Nooijen:2005p2277},
well-defined (giving continuous, unique potential surfaces), and
variational.\cite{Pople:1976p1} 
While it has been argued that the practical benefits of variationality are
rather limited,\cite{Bartlett:1981p359} 
the efficiency of gradient computations, at least, is improved by formulating a
theory in terms of a Hermitian and stationary energy
functional.\cite{Szalay:1995p281}
With respect to scope and stability, methods that show consistent performance
for open-shell systems, strongly correlated states, and non-equilibrium
geometries are particularly valuable.\cite{Bartlett:1981p359}

The incorrect scaling of truncated configuration interaction (CI) energies with
system size has inspired the development of size-extensive alternatives. Among
the earliest formulations, the coupled electron pair approximations (CEPAs)
\cite{Kelly:1963p2091,Kelly:1964pA1450,Meyer:1973p1017,Ahlrichs:1979p31,Koch:1981p387}
attracted much attention in
1970s,\cite{Gelus:1970p503,Staemmler:1972p187,Ahlrichs:1975p1235,Kollmar:1977p3583,Wasilewski:1988p1289}
offering rigorous extensivity and size-consistency while retaining much of the
linearity\cite{Taube:2009p1441122} of CI in their equations.
CEPA methods, however, have been shown to rapidly deteriorate as the molecular
geometry deviates from equilibrium\cite{Taube:2009p1441122} and yield energies
that vary under the rotation of the occupied orbitals.\cite{Ahlrichs:1979p31}
Partly in light of such defects, CEPA has been largely displaced by
coupled-cluster (CC)
theory.\cite{Coester:1958p421,Coester:1960p477,Cizek:1966p4256,Bartlett:1978p561,Bartlett:1981p359,Crawford:2000p33,Bartlett:2007p291,Shavitt:2009}
In addition to size-extensivity, CC offers orbital invariance and improved
stability for non-equilibrium structures\cite{Taube:2009p1441122}, but has a
non-Hermitian energy functional and non-linear equations which are not readily
amenable to parallel implementation.
Although neither class of methods is strictly variational, VCEPA (variational
CEPA) has been shown to be effectively equivalent to its non-variational
counterpart.\cite{Kollmar:2010p2449}
Various other modifications to resolve the deficiencies of traditional CEPA have
been explored, including self-consistent size-consistent CI,
\cite{Daudey:1993p1240,Malrieu:2010p179} orbital-invariant CEPA,
\cite{Nooijen:2006p25,Kollmar:2011p084102} and orbital-optimized CEPA
formulations.
\cite{Kollmar:2010p311,Bozkaya:2013p054104,Soydas:2014p1073,Bozkaya:2013p154105}
Recently, the CEPA methods have been revived by Neese and
co-workers\cite{Wennmohs:2008p217,Neese:2009p114108,Kollmar:2010p2449} who
developed the local pair-natural-orbital CEPA (LPNO-CEPA) methods and have
implemented them for massively parallel computer architectures.

It has recently been
demonstrated\cite{Kutzelnigg:2006p171101,Mazziotti:2008p253002,Mazziotti:2010p062515,DePrince:2012p1917}
that CEPA methods naturally arise in the context of theories that obtain the
molecular energies from density cumulants, the connected and extensive
components of the reduced density matrices
(RDMs).\cite{Kutzelnigg:1997p432,Mazziotti:1998p419,Mazziotti:1998p4219,Kutzelnigg:1999p2800,Kong:2011p3541,Hanauer:2012p50}
The advantage of cumulant-based theories is that, unlike their RDM-based
counterparts,\cite{Nakata:2009p042109,vanAggelen:2010p114112,Verstichel:2010p114113}
they are naturally size-extensive and
size-consistent.\cite{Kutzelnigg:1999p2800,Herbert:2007p261}
We have recently achieved the first
implementation\cite{Simmonett:2010p174122,Sokolov:2012p054105} of density
cumulant functional theory (DCFT), proposed by Kutzelnigg in
2006.\cite{Kutzelnigg:2006p171101}
In DCFT, the molecular energy is obtained in terms of a mean-field one-particle
RDM and the two-particle density cumulant, constrained to be at least
approximately $N$-representable ({\it i.e.}\@ to correspond to a physical
$N$-electron wavefunction).
Like traditional CC theory, DCFT is size-extensive and orbital-invariant, but it
has the additional advantage of a stationary and Hermitian energy functional,
which simplifies the computation of molecular properties.
In the original DCFT formulation
(DC-06)\cite{Kutzelnigg:2006p171101,Simmonett:2010p174122,Sokolov:2012p054105}
$N$-representability conditions derived from second-order M\o ller-Plesset
perturbation theory (MPPT) were used,\cite{Kutzelnigg:2004p7350} yielding
equations similar to those of the simplest CEPA model
(CEPA$_0$),\cite{Meyer:1973p1017,Koch:1981p387} but including higher-order terms
in the description of one-particle correlation effects.
Using the same set of conditions, we have developed new formulations of DCFT
that take advantage of an improved description of the one-particle density
matrix (DC-12)\cite{Sokolov:2013p024107} and full orbital optimization (ODC-06
and ODC-12 methods).\cite{Sokolov:2013p204110}

Our previous
studies\cite{Simmonett:2010p174122,Sokolov:2012p054105,Sokolov:2013p024107,Sokolov:2013p204110}
demonstrated for a limited set of systems that the DC-06, DC-12, ODC-06 and
ODC-12 methods generally yield molecular energies and properties competitive
with those obtained by CCSD and CCSD(T), but may exhibit unstable performance
due to imbalances in the description of electron correlation.
Herein, we present an extensive benchmark of the DCFT methods with respect to
thermochemical and kinetic molecular properties, including noncovalent
interactions, barrier heights in hydrogen-transfer reactions, radical
stabilization energies, and adiabatic ionization energies for challenging
electron-dense systems.
We conclude our benchmark study by testing the performance of DCFT for covalent
bond stretching in diatomic molecules.


\section{Overview of DCFT}
In this section a short overview of DCFT is presented. For details on the theory
the reader is referred to our earlier
publications.\cite{Simmonett:2010p174122,Sokolov:2013p024107,Sokolov:2013p204110}
In the RDM methods\cite{Mazziotti:2007} the exact molecular energy is expressed
as a functional of the one- and two-particle reduced density matrices,
$\boldsymbol{\gamma}_1$ and $\boldsymbol{\gamma}_2$ (1-RDM and 2-RDM):
\begin{equation}
	\label{e-rdm}
	E
    = 
	h_p^q
    \gamma_q^p
    +
    \tfrac{1}{2}
    g_{pq}^{rs}
    \gamma_{rs}^{pq} \ ,
	\qquad
	[\boldsymbol{\gamma}_1]_q^p
    \equiv
    \gamma_q^p \ , 
	\qquad
	[\boldsymbol{\gamma}_2]_{rs}^{pq}
    \equiv
    \gamma_{rs}^{pq}\,.
\end{equation}
In \cref{e-rdm}, $h_p^q$ and $g_{pq}^{rs}$ are the usual one- and two-electron
integrals in the orthonormal spin-orbital basis $\{\psi_p\}$ and summation over
the repeated indices is implied.
Expressing $\boldsymbol{\gamma}_1$ through $\boldsymbol{\gamma}_2$ via
the partial trace relation $\sum_r\gamma_{qr}^{pr}=(N-1)\gamma_q^p$, the energy
functional \eqref{e-rdm} can be minimized by varying $\boldsymbol{\gamma}_2$
subject to $N$-representability constraints.
This is the essence of the variational 2-RDM approach.\cite{Mazziotti:2007}


In DCFT, some of the challenges of the 2-RDM approach are circumvented by
expanding $\boldsymbol{\gamma}_2$ in terms of its irreducible components -- the 1-RDM and
the two-particle cumulant (denoted by $\boldsymbol{\lambda}_2$):
\begin{equation}
	\label{lambda}
	\gamma_{rs}^{pq}
    =
	\gamma_r^p
    \gamma_s^q
    -
    \gamma_r^q
    \gamma_s^p
	+
    \lambda_{rs}^{pq}\,.
\end{equation}
In \cref{lambda}, $\boldsymbol{\lambda}_2$ describes the correlated part of $\boldsymbol{\gamma}_2$ that
cannot be expressed via $\boldsymbol{\gamma}_1$. The cumulant also determines the correlation
contribution to $\boldsymbol{\gamma}_1$, allowing the 1-RDM to be decomposed as the sum of an
idempotent 1-RDM ($\boldsymbol{\kappa}$) and a correlation correction ($\boldsymbol{\tau}$):
\begin{equation}
	\label{tau}
	\boldsymbol{\gamma}_1
    =
    \boldsymbol{\kappa}
    +
    \boldsymbol{\tau}\,.
\end{equation}
The correlation component $\boldsymbol{\tau}$ is fully specified by $\boldsymbol{\lambda}_2$, whereas
$\boldsymbol{\kappa}$ is independent of $\boldsymbol{\lambda}_2$. \cref{lambda,tau} allow us to write an
equivalent energy expression with $\boldsymbol{\kappa}$ and $\boldsymbol{\lambda}_2$ as independent
functional parameters:
\begin{equation}
	\label{e-dcft}
    \begin{array}{c}
        E[\boldsymbol{\kappa},\boldsymbol{\lambda}_2]
        =
        \tfrac{1}{2}
        (h_p^q+f_p^q)
        (\kappa_q^p+\tau_q^p)
        +
        \tfrac{1}{4}
        \overline{g}_{pq}^{rs}
        \lambda_{pq}^{rs}
        ,
        \\
        f_p^q
        =
        h_p^q
        +
        \overline{g}_{pr}^{qs}
        (\kappa_s^r+\tau_s^r)
        ,
        \quad
        \overline{g}_{rs}^{pq}
        =
        g_{rs}^{pq}
        -
        g_{rs}^{qp}\,.
    \end{array}
\end{equation}
