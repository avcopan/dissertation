\chapter[%
    Introduction and Literature Review
]{%
    Introduction and Literature Review
}

\section{Naive Electronic Structure Theory}

As an entry point to electronic structure theory, let us begin by forgetting
what we know about electrons from the standard model of particle physics.
From the standpoint of Pauli at the end of 1925, solving the spectrum of
hydrogen\cite{Pauli:1926p336} with Heisenberg's new matrix
mechanics,\cite{Heisenberg:1925p879} chemical matter was described by nimble,
negatively-charged electrons orbiting heavy, positively-charged nuclei.  This
theory would be conceptually clarified over the following weeks by
Schr\"odinger's development of wave mechanics,
\cite{Schrodinger:1926p361,Schrodinger:1926p489,Schrodinger:1926p734}
which described the possible states of electrons in a molecule as eigenfunctions
of the quantum-mechanical Hamiltonian, oscillating in time with a frequency
proportional to their energy.
In atomic units:
\begin{equation}
    \Psi(t)
    =
    \Psi
    e^{-iEt}
    \qquad
    \hat{H}
    \Psi
    =
    E
    \Psi.
\end{equation}
Crudely speaking, this Hamiltonian is derived from its classical counterpart by
replacing its momentum variables with del operators divided by the unit
imaginary,
\(
    \hat{\mathbf{p}}
    =
    \frac{1}{i}
    \nabla
\).
It can be written as a sum over one- and two-electron terms
\begin{equation}
    \hat{H}
    =
    \sum_i^\text{electrons}
    \hat{h}_i
    +
    \sum_{i<j}^{\substack{\text{electron}\\\text{pairs}}}
    \hat{g}_{ij}
\end{equation}
where the one-electron operator
\(
    \hat{h}_i
\)
describes the kinetic energy of the \(i^\text{th}\) electron and its
electrostatic attraction to the nuclei (Coulomb's law), and the two-electron
operator
\(
    \hat{g}_{ij}
\)
describes the electrostatic repulsion between electrons \(i\) and \(j\).
\begin{equation}
    \hat{h}_i
    \equiv
    \tfrac{1}{2}
    \hat{\mathbf{p}}_i^2
    -
    \sum_A^\text{nuclei}
    \frac{Z_A}{|\mathbf{r}_A - \mathbf{r}_i|}
    \qquad
    \hat{g}_{ij}
    \equiv
    \frac{1}{|\mathbf{r}_i - \mathbf{r}_j|}
\end{equation}
The vector space containing the eigenfunctions of a Hamiltonian is its Hilbert
space, \(\mathcal{H}\), which in this case is the space of square integrable
functions of \(n\) position variables,
\(
    L^2(\mathbb{R}^{3n})
\),
one for each electron in the molecule.
Since this space is infinite-dimensional, the only way to make progress in
general is to determine a basis that approximately spans the states of interest.
Here we are helped by the fact that a function of \(n\) position variables can
be written as a linear combination of products
\begin{equation}
    \Psi(\mathbf{r}_1, \dots, \mathbf{r}_n)
    =
    \sum_{p_1\cdots p_n}
    c_{p_1\cdots p_n}\,
    \phi_{p_1}(\mathbf{r}_1)
    \cdots
    \phi_{p_n}(\mathbf{r}_n)
\end{equation}
where the one-electron functions or {\itshape orbitals} in this expansion span a
one-electron Hilbert space,
\(
    \mathcal{H}_i
    =
    L^2(\mathbb{R}^3)
\).
More generally, the Hilbert space of a system is always given by a tensor
product of the Hilbert spaces for its constituent particles.
\begin{equation}
    \mathcal{H}
    =
    \mathcal{H}_1
    \otimes
    \mathcal{H}_2
    \otimes
    \cdots
    \otimes
    \mathcal{H}_n
\end{equation}
This enables a general strategy for bootstrapping the electronic structure
problem:
\begin{enumerate}
    \item
        \label{item:atomic-orbitals}
        The states of a single electron orbiting a single nucleus are
        analytically known products of Laguerre polynomials and spherical
        Harmonics, and the energy of these 1s, 2s, 2p, \dots\ atomic orbitals
        increases with the number of nodes.
    \item
        \label{item:mean-field}
        Treating the other electrons in a molecule as static fields which
        partially shield the nuclear charges, the states of individual electrons
        in a molecule are well described as linear combinations of atomic
        one-electron functions centered on each nucleus.
        The dominant contributions to these molecular orbitals come from atomic
        orbitals with a similar energy, so that the lowest energy molecular
        orbital typically looks like the 1s orbital of the nucleus with the
        greatest charge, with very little contribution from, say, the 42g
        orbital of another atom.
        The higher energy states are linear combinations of higher energy atomic
        orbitals with increasing numbers of nodes.
    \item
        \label{item:full-ci}
        Finally, the total electronic wavefunction is described by a linear
        combination of products of these molecular orbitals, and the electronic
        ground-state will be spanned to a good approximation by products of the
        low-energy molecular orbitals.
        When the molecular orbitals are weakly interacting, so that the
        mean-field approximation of step~\ref{item:mean-field} furnishes a good
        description of their motions, the wavefunction will be heavily dominated
        by a single one of these products.
        From a statistical perspective, this means that the electron probability
        density of the wavefunction approximately separates into a product of
        one-electron densities, because the molecular orbitals are {\itshape
        weakly correlated}.
\end{enumerate}
Step~\ref{item:full-ci} yields a matrix eigenvalue equation in the orbital
product basis.
\begin{equation}
    \mathbf{H}\mathbf{c}
    =
    E\mathbf{c}
    \qquad
    (\mathbf{H})_{PQ}
    =
    \langle \phi_{p_1}\cdots \phi_{p_n}|
    \hat{H}
    |\phi_{q_1}\cdots \phi_{q_n} \rangle
\end{equation}
In the limit of a complete expansion, this matrix equation is exactly equivalent
to the Schr\"odinger equation and the coefficients of the solution vector
correspond to the components of the wavefunction along our product functions,
\(
    (\mathbf{c})_P
    =
    \langle \phi_{p_1}\cdots \phi_{p_n}|\Psi\rangle
\).
In order to stack the deck in favor of an individual product in the expansion,
we can determine the molecular orbitals in step~\ref{item:mean-field} to
minimize its Hamiltonian (i.e.~energy) expectation value
\begin{equation}
    \label{eq:introduction:orbital-product-expectation-value}
    \langle \phi_1\cdots \phi_n|
    \hat{H}
    |\phi_1\cdots \phi_n \rangle
    =
    \sum_{i=1}^\mathrm{orbitals}
    h_i^i
    +
    \sum_{i<j}^{\substack{\mathrm{orbital}\\\mathrm{pairs}}}
    g_{ij}^{ij}
\end{equation}
where we have defined the all-important {\itshape one- and two-electron
integrals}.
\begin{equation}
    h_p^q
    \equiv
    \langle\phi_p|\hat{h}_1|\phi_q\rangle
    =
    \int_{\mathbb{R}^3}
    d^3\mathbf{r}_1\,
    \phi_p^*(\mathbf{r}_1)
    \hat{h}_1
    \phi_q(\mathbf{r}_1)
\end{equation}
\begin{equation}
    g_{pq}^{rs}
    \equiv
    \langle\phi_p\phi_q|\hat{g}_{12}|\phi_r\phi_s\rangle
    =
    \int_{\mathbb{R}^6}
    d^3\mathbf{r}_1
    d^3\mathbf{r}_2\,
    \phi_p^*(\mathbf{r}_1)
    \phi_q^*(\mathbf{r}_2)
    \hat{g}_{12}
    \phi_r(\mathbf{r}_1)
    \phi_s(\mathbf{r}_2)
\end{equation}
According to the variational principle for Hermitian operators, the
minimum-energy orbital product will be the best statistically uncorrelated
approximation to the ground-state wavefunction.
For an orthonormal orbital basis, this minimization is equivalent to an
effective Schr\"odinger equation for one-electron states
\begin{equation}
    \label{eq:introduction:mean-field-orbitals}
    (
        \hat{h}_1
        +
        \hat{v}_1
    )
    \phi_i(\mathbf{r}_1)
    =
    \epsilon_i\,
    \phi_i(\mathbf{r}_1)
\end{equation}
where \(\hat{v}_1\) is the mean electrostatic field of the other electrons in
the molecule.
\begin{equation}
    \hat{v}_1
    \equiv
    \sum_{j\neq i}
    \int_{\mathbb{R}^3}
    d\mathbf{r}_2\,
    \frac{%
        \phi_j^*(\mathbf{r}_2)\phi_j(\mathbf{r}_2)
    }{%
        |\mathbf{r}_1 - \mathbf{r}_2|
    }
\end{equation}
This is the mean-field approximation described in step~\ref{item:mean-field}
above.
We complete the bootstrapping cycle by expanding these unknown molecular
orbitals in the basis of the atomic orbitals, \(\{\chi_\mu\}\), described in
step~\ref{item:atomic-orbitals}
\begin{equation}
    \phi_i(\mathbf{r}_1)
    =
    \sum_\mu
    d_{\mu i}\,
    \chi_\mu(\mathbf{r}_1)
\end{equation}
which turns \cref{eq:introduction:mean-field-orbitals} into a generalized
eigenvalue equation
\begin{equation}
    \label{eq:introduction:self-consistent-field-equations}
    (
        \mathbf{h}
        +
        \mathbf{v}
    )
    \mathbf{d}_i
    =
    \epsilon_i
    \mathbf{s}\,
    \mathbf{d}_i
\end{equation}
where \((\mathbf{d}_i)_\mu=d_{\mu i}\) is a vector of the unknown orbital
coefficients, and we have defined the following integral matrices.
\begin{equation}
    (\mathbf{s})_{\mu\nu}
    \equiv
    \langle\chi_\mu|\chi_\nu\rangle
    \quad
    (\mathbf{h})_{\mu\nu}
    \equiv
    \langle\chi_\mu|\hat{h}_1|\chi_\nu\rangle
    \quad
    (\mathbf{v})_{\mu\nu}
    \equiv
    \langle\chi_\mu|\hat{v}_1|\chi_\nu\rangle
    =
    \sum_{j\neq i}
    \langle\chi_\mu\phi_j|\hat{g}_{12}|\chi_\nu\phi_j\rangle
\end{equation}
The overlap matrix \(\mathbf{s}\) is not an identity matrix, because the
atom-centered 1s, 2s, 2p, \dots\ orbitals are generally not orthogonal.
This is an unusual eigenvalue problem, because the electron field integrals in
\(\mathbf{v}\) themselves depend on the unknown molecular orbitals.
In practice, this is solved by starting with an initial guess for the orbitals
and repeatedly solving \cref{eq:introduction:self-consistent-field-equations}
until it spits back the current set, at which point we have determined the
{\itshape self-consistent field} for a given chemical system.


\section{Spin and Exchange Symmetry}


\subsection{Na\"ive Electronic Structure Theory}

\subsection{Spin}

\subsection{Antisymmetry}

\subsection{Fock Space}

\subsection{Wavefunction Methods}

\section{Density Cumulants}

\subsection{Generating Functions}

\subsection{Generalized Normal Ordering}

\section{Prospectus}

\subsection{History}

\subsection{Contents}



The general strategy of many-body quantum mechanics:
\begin{enumerate}
    \item
        Determine the Hilbert spaces for the individual particles in your
        system, ideally by solving an effective one-body problem for the
        stationary states.
    \item
        The Hilbert space for the many-body system is the tensor-product of
        one-particle Hilbert spaces, i.e.~the many-body states are spanned by
        all possible products of one-particle states.
    \item
        Solve for the many-body wavefunction in the basis one or more of these
        product functions.
        If the particles are approximately decoupled, and the one-body basis is
        carefully chosen, one may be able to describe the wavefunction to a good
        approximation using only one or a few product functions.
    \item
        If one or more particles are highly coupled, one may need to include
        larger expansions, such as all possible combinations of states in a
        given energy range, in order to achieve even qualitative accuracy.
\end{enumerate}

\section{One-Electron States}

Hydrogen or silver or alkali metals
\begin{equation}
    (\hat{h} + \hat{v}_\mathrm{e})
    |\psi\rangle
    =
    \epsilon
    |\psi\rangle
    \qquad
    \hat{h}
    =
    \tfrac{1}{2}
    \hat{\mathbf{p}}^2
    +
    \hat{v}_\mathrm{n}
    \qquad
    \hat{v}_\mathrm{n}
    =
    \sum_a^{\mathrm{nuclei}}
    \frac{q_a}{|\mathbf{r}_a-\hat{\mathbf{r}}|}
\end{equation}
\begin{equation}
    \mathcal{H}
    =
    L^2(\mathbb{R}^3)
\end{equation}
Assume the motion of the nuclei is infinitely slow on the time-scale of
electronic motion (Born-Oppenheimer approximation).\\
Assume the electron only sees an average field describing the shielding due to
the other electrons (mean-field approximation).

\subsection{Symmetry}

Symmetry properties of the potential give us spatial symmetries
\begin{equation}
    \hat{o}
    |\psi\rangle
    =
    o
    |\psi\rangle
\end{equation}
point groups and lattice groups.
For one electron atoms the O(3) symmetries are described by the conservation of
angular momentum, \(\hat{\mathbf{j}}^2\) and \(\hat{j}_z\).

\noindent
Atoms: angular momentum (s, p, d)

\noindent
Free electrons: plane waves

\noindent
Molecules and crystals are described by symmetry adapted combinations of these
functions

\subsection{Naive \(n\)-Electron States and the Mean Field}
The mean field of the electrons could be derived by analyzing the full
Schr\"odinger equation
\begin{equation}
    \hat{H}
    |\Psi\rangle
    =
    E
    |\Psi\rangle
    \qquad
    \hat{H}_1
    =
    \sum_{i=1}^n
    \hat{h}_i
    +
    \sum_{i=1}^n
    \sum_{j=i+1}^n
    \hat{g}_{ij}
    \qquad
    \hat{g}_{ij}
    =
    \frac{1}{|\hat{\mathbf{r}}_i-\hat{\mathbf{r}}_j|}
\end{equation}
\begin{equation}
    \mathcal{H}_n
    =
    \mathcal{H}_1
    \otimes
    \cdots
    \otimes
    \mathcal{H}_1
\end{equation}

Mean field is described by assuming statistically independent electrons.

Discuss how symmetries combine for the \(n\)-electron state.

\section{Spin and Pauli Exclusion}

Goudsmit-Uhlenbeck (Uhlenbeck:1926p264) give us spin, which is a constant intrinsic to the electron
as a particle
\begin{equation}
    \hat{s}^2
    |\psi\rangle
    =
    \tfrac{3}{4}
    |\psi\rangle
\end{equation}

The earlier Stern-Gerlach experiment (Gerlach:1922p349) showed in hindsight that
spin can be oriented up or down along a given axis
\begin{equation}
    \hat{s}_z
    |\psi\rangle
    =
    \pm
    \tfrac{1}{2}
    |\psi\rangle
\end{equation}

\begin{equation}
    \hat{s}_z
    =
    \tfrac{1}{2}
    (
        |\alpha\rangle\langle\alpha|
        -
        |\beta\rangle\langle\beta|
    )
\end{equation}

\begin{equation}
    \mathcal{H}
    =
    L^2(\mathbb{R}^3)
    \otimes
    \mathrm{SU}(2)
\end{equation}


Two problems led to particle stats: the anomalous Zeeman effect, and the absence
of electron {\itshape bunching}.
In an orbital picture, you would expect all of the electrons to fall in the
lowest energy state, such that the ground state is qualitatively described by a
product
\begin{equation}
    \Psi_0(1,\ldots,n)
    \approx
    \psi_0(1)
    \cdots
    \psi_0(n)
\end{equation}
which is indeed what you find for bosons.

Paul describing spin before people knew what it was (Pauli:1925p373):
``The closed energy configurations contribute nothing to the magnetic moment or
angular momentum of the atom.
In particular, in alkali metals the angular momentum values of the atom and its
energy changes in the presence of an external magnetic field are appropriately
interpreted as the sole working of the optically active electron, and the same
situation is thought to be the case in the observations of the anomalous Zeeman
effect.
From this standpoint, the doublet structure of the alkali spectra, as well as
the breakdown of Larmor's theorem, must therefore come from some intrinsic,
classically non-describable type of two-valuedness that is a characteristic of
the optically active electron's state.''
Pauli proposed this without proposing any model for this extra degree of
freedom.
That was the contribution of Uhlenbeck and Goudsmit in 1926.


Pauli exlusion (Pauli:1925p756):
``By considering the case of strong magnetic fields we can reduce [E.~C.]
Stoner's result, that the number of electrons in a completed subgroup [of an
atom] is the same as the number of the corresponding terms of the Zeeman effect
of the alkali spectra, to the following more general rule about the occurrence
of equivalent electrons in an atom:
{\itshape
    There can never be two or more equivalent electrons in an atom for which in
    strong fields the values of all quantum numbers \(n, k_1, k_2, m_1\) (or,
    equivalently \(n, k_1, m_1, m_2\)) are  the same.
    If an electron is present in the atom for which these quantum numbers (in an
    external field) have definite values, this state is ``occupied.''%
}
\dots We cannot give a further justification for this rule, but it seems to be a
very plausible one.
It refers, as mentioned, first of all to the case of strong field.
However, from thermodynamic arguments (the invariance of statistic weights under
adiabatic transformations of the system) it follows that the number of
stationary states of an atom must be the same in strong and weak fields for
given values of the number \(k_1\) and \(k_2\) of the separate electrons and a
value of \(\overline{m_1}=\sum m_1\) for the whole atom.''

This is in contrast to photons in stimulated emission, where they are all forced
to occupy the same state.

\subsection{Antisymmetry}

Heisenberg to Pauli (May 5, 1926) on the connection between Pauli exclusion and
the singlet triplet gap in helium (Mehra:1982):
{\itshape
    I want to write to you that I have found a rather decisive argument that
    your exclusion of equivalent orbitals is connected with the singlet-triplet
    separation [in helium].
}

Heisenberg to Born (May 26, 1926) on the challenge of calculating the energy
states of helium (Mehra:1982):
{\itshape
    The large separation between the singlet and triplet systems could not be
    explained by a magnetic interaction of the spinning electrons; one had
    therefore to assume with Hund a large force of unknown origin between the
    magnets [Hund's rule].
}
\\
He explains everything by qualitative arguments in his paper on ``The Many-Body
Problem and Resonance in Quantum Mechanics'' (Heisenberg:1926p411), which
{\itshape
    aims to lay the groundwork for a quantum-mechanical treatment of the
    many-body problem.
}
\\
He is the first to propose the form of the Slater determinant.
\begin{equation}
    \Phi
    =
    \sum_{\pi\in\mathrm{S}_n}
    (-)^\pi
    \phi_(\pi(1))
    \cdots
    \phi_n(\pi(n))
\end{equation}



\section{Fock Space}

\begin{equation}
    \hat{H}
    =
    \sum_{i=1}^n
    \hat{h}(i)
    +
    \sum_{i=1}^n
    \sum_{j=i+1}^n
    \hat{g}(i,j)
\end{equation}

\begin{equation}
    \hat{g}(i,j)
    =
    \frac{1}{|\hat{\mathbf{r}}_i-\hat{\mathbf{r}}_j|}
\end{equation}

\begin{equation}
    \mathcal{H}^n
    =
    \mathcal{H}(1)\otimes\cdots\otimes\mathcal{H}(n)
\end{equation}

\begin{equation}
    |\Psi\rangle
    =
    \sum_{p_1\cdots p_n}
    c_{p_1\cdots p_n}
    |\psi_{p_1}\rangle
    \otimes
    \cdots
    \otimes
    |\psi_{p_n}\rangle
\end{equation}
