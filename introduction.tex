\chapter[%
    Introduction and Literature Review
]{%
    Introduction and Literature Review
}

\section{Naive Electronic Structure Theory}

As an entry point to electronic structure theory, let us begin by forgetting
what we know about electrons from the standard model of particle physics.
From the standpoint of Heisenberg and others developing the new quantum theory
in 1925\cite{Heisenberg:1925p879} chemical matter was described by nimble,
negatively-charged electrons orbiting heavy, positively-charged nuclei.
This theory would be conceptually clarified in the following year by
Schr\"odinger's development of wave mechanics,
\cite{Schrodinger:1926p361,Schrodinger:1926p489,Schrodinger:1926p734} which
described the possible states of electrons in a molecule as eigenfunctions of
the quantum-mechanical Hamiltonian, oscillating in time with a frequency
proportional to their energy.
In atomic units:
\begin{equation}
    \Psi(t)
    =
    \Psi
    e^{-iEt}
    \qquad
    \hat{H}
    \Psi
    =
    E
    \Psi.
\end{equation}
Crudely speaking, this Hamiltonian is derived from its classical counterpart by
replacing its momentum variables with del operators divided by the imaginary
unit,
\(
    \hat{\mathbf{p}}
    =
    \frac{1}{i}
    \nabla
\).
It can be written as a sum over one- and two-electron terms
\begin{equation}
    \hat{H}
    =
    \sum_i^\text{electrons}
    \hat{h}_i
    +
    \sum_{i<j}^{\substack{\text{electron}\\\text{pairs}}}
    \hat{g}_{ij}
\end{equation}
where the one-electron operator
\(
    \hat{h}_i
\)
describes the kinetic energy of the \(i^\text{th}\) electron and its
electrostatic (Coulomb's law) attraction to the nuclei, and the two-electron
operator
\(
    \hat{g}_{ij}
\)
describes the Coulombic repulsion between electrons \(i\) and \(j\).
\begin{equation}
    \hat{h}_i
    \equiv
    \tfrac{1}{2}
    \hat{\mathbf{p}}_i^2
    -
    \sum_A^\text{nuclei}
    \frac{Z_A}{|\mathbf{r}_A - \mathbf{r}_i|}
    \qquad
    \hat{g}_{ij}
    \equiv
    \frac{1}{|\mathbf{r}_i - \mathbf{r}_j|}
\end{equation}
The vector space containing the wavefunction \(\Psi\) is the system's Hilbert
space, \(\mathcal{H}\), which in this case is the space of square integrable
functions of \(n\) position variables,
\(
    L^2(\mathbb{R}^{3n})
\),
one for each electron in the molecule.
Since this space is infinite-dimensional, the only way forward in most cases is
to determine a basis that approximately spans the states of interest.
Here we are helped by the fact that a function of \(n\) position variables can
be written as a linear combination of products
\begin{equation}
    \Psi(\mathbf{r}_1, \dots, \mathbf{r}_n)
    =
    \sum_{p_1\cdots p_n}
    c_{p_1\cdots p_n}\,
    \phi_{p_1}(\mathbf{r}_1)
    \cdots
    \phi_{p_n}(\mathbf{r}_n)
\end{equation}
where the one-electron functions or {\itshape orbitals} in this expansion span a
one-electron Hilbert space,
\(
    \mathcal{H}_\mathrm{e}
    =
    L^2(\mathbb{R}^3)
\).
More generally, the Hilbert space of a system is always given by a tensor
product of the Hilbert spaces for its constituent particles.
\begin{equation}
    \mathcal{H}
    =
    \mathcal{H}_\mathrm{e}
    \otimes
    \mathcal{H}_\mathrm{e}
    \otimes
    \cdots
\end{equation}
In words, this says that the states of a system are spanned by all possible
combinations of the states its components exhibit in isolation.
This enables a general strategy for bootstrapping the electronic structure
problem:
\begin{enumerate}
    \item
        \label{item:atomic-orbitals}
        The states of a single electron orbiting a single nucleus are determined
        by the spherical symmetry of the nuclear potential to have a degenerate
        shell structure where the energy of the state increases with the number
        of nodal surfaces in the function.
        These 1s, 2s, 2p, \dots\ atomic orbitals have analytically known
        functional forms.
    \item
        \label{item:mean-field}
        Treating the other electrons in a molecule as static fields which
        partially shield the nuclear charges, the states of individual electrons
        in a molecule are well described as linear combinations of atomic
        one-electron functions centered on each nucleus.
        The dominant contributions to these molecular orbitals come from atomic
        orbitals with a similar energy, so that the lowest energy molecular
        orbital typically looks like the 1s orbital of the nucleus with the
        greatest charge, with very little contribution from, say, the 42g
        orbital of another atom.
        The higher energy states are linear combinations of higher energy atomic
        orbitals with increasing numbers of nodes.
    \item
        \label{item:full-ci}
        Finally, the total electronic wavefunction is described by a linear
        combination of products of these molecular orbitals, and the electronic
        ground-state will be spanned to a good approximation by products of the
        low-energy molecular orbitals.
        When the molecular orbitals are weakly interacting, the mean-field
        approximation of step~\ref{item:mean-field} furnishes a good description
        of their motions, and the wavefunction will be heavily dominated by a
        single one of these products.
        From a statistical perspective, this means that the electron probability
        density of the wavefunction approximately separates into a product of
        one-electron densities, because the molecular orbitals are {\itshape
        weakly correlated}.
\end{enumerate}
At the end of this procedure, we have a matrix equation in the product basis.
\begin{equation}
    \mathbf{H}\mathbf{c}
    =
    E\mathbf{c}
    \qquad
    (\mathbf{H})_{PQ}
    =
    \langle \phi_{p_1}\cdots \phi_{p_n}|
    \hat{H}
    |\phi_{q_1}\cdots \phi_{q_n} \rangle
\end{equation}
In the limit of a complete expansion, this eigenvalue equation is exactly
equivalent to the Schr\"odinger equation and the coefficients of the solution
vector correspond to the components of the wavefunction along our product
functions.
In order to stack the deck in favor of one product in the expansion, we can
determine the molecular orbitals in step~\ref{item:mean-field} to minimize its
energy expectation value
\begin{equation}
    \label{eq:introduction:orbital-product-expectation-value}
    \langle \phi_1\cdots \phi_n|
    \hat{H}
    |\phi_1\cdots \phi_n \rangle
    =
    \sum_{i=1}^\mathrm{orbitals}
    h_i^i
    +
    \sum_{i<j}^{\substack{\mathrm{orbital}\\\mathrm{pairs}}}
    g_{ij}^{ij}
\end{equation}
where we have defined the all-important {\itshape one- and two-electron
integrals}.
\begin{equation}
    h_p^q
    \equiv
    \langle\phi_p|\hat{h}_1|\phi_q\rangle
    =
    \int
    d^3\mathbf{r}_1\,
    \phi_p^*(\mathbf{r}_1)
    \hat{h}_1
    \phi_q(\mathbf{r}_1)
\end{equation}
\begin{equation}
    g_{pq}^{rs}
    \equiv
    \langle\phi_p\phi_q|\hat{g}_{12}|\phi_r\phi_s\rangle
    =
    \int
    d^3\mathbf{r}_1
    d^3\mathbf{r}_2\,
    \phi_p^*(\mathbf{r}_1)
    \phi_q^*(\mathbf{r}_2)
    \hat{g}_{12}
    \phi_r(\mathbf{r}_1)
    \phi_s(\mathbf{r}_2)
\end{equation}
For an orthonormal orbital basis, this minimization leads to an effective
Schr\"odinger equation for one-electron states
\begin{equation}
    \label{eq:introduction:mean-field-orbitals}
    (
        \hat{h}_1
        +
        \hat{v}_1
    )
    \phi_i(\mathbf{r}_1)
    =
    \epsilon_i\,
    \phi_i(\mathbf{r}_1)
\end{equation}
where \(\hat{v}_1\) is the mean electrostatic field of the other electrons in
the molecule.
\begin{equation}
    \hat{v}_1
    \equiv
    \sum_{j\neq i}
    \int
    d\mathbf{r}_2\,
    \frac{%
        \phi_j^*(\mathbf{r}_2)\phi_j(\mathbf{r}_2)
    }{%
        |\mathbf{r}_1 - \mathbf{r}_2|
    }
\end{equation}
This is the mean-field approximation described in step~\ref{item:mean-field}
above.
We complete the bootstrapping cycle by expanding these unknown molecular
orbitals in the basis of the atomic orbitals described in
step~\ref{item:atomic-orbitals} and solving for the expansion coefficients.

The general strategy we have just outlined carries over into modern electronic
structure theory, but it is missing two essential ingredients: the spin of the
electron, and the antisymmetric permutational symmetry of electrons as
indistinguishable particles.
We turn to these next.


\section{Spin}

The non-relativistic theory of one-electron states was in a sense completed with
Pauli's publication solution of the spectrum of hydrogen at the start of
1926.\cite{Pauli:1926p336}
His work concludes with a discussion of the recent work by Goudsmit and
Uhlenbeck\cite{Uhlenbeck:1925p953} showing that the anomalous Zeeman splitting
of the alkali metals could be explained by positing an intrinsic source of
angular momentum and magnetism for the electron besides that generated by its
orbital motion about the nucleus.
This was the electron's spin.
The need for this additional quantum number had already been understood by Pauli
in his analysis of alkali metal spectra at the end of 1924:
\begin{quote}
    In alkali metals the angular momentum values of the atom, and its energy
    changes in the presence of an external magnetic field, are appropriately
    interpreted as the sole working of the optically active electron, and the
    same situation is thought to be the case in the observations of the
    anomalous Zeeman effect.
    \emph{%
        From this standpoint, the doublet structure of the alkali
        spectra, as well as the breakdown of Larmor's theorem, must
        therefore come from some intrinsic, classically non-describable
        type of two-valuedness that is a characteristic of the optically
        active electron.%
    }\footnote{%
        Translated from Ref.~\citenum{Pauli:1925p373} with emphasis added.
    }
\end{quote}
In hindsight, the Stern-Gerlach experiment\cite{Gerlach:1922p349} had already
shown in 1922 that the 5s electron of the silver atom was quantized into two
magnetic states, whereas the new quantum theory predicted an odd number of these
(\(0, \pm1, \pm2, \dots\)) for the spatial orbits of a charged particle.
This new source of angular momentum was characterized by half-integer values
whose eigenfunctions cannot exist in \(L^2(\mathbb{R}^3)\).
\begin{equation}
    \hat{s}_z
    \psi
    =
    \pm
    \tfrac{1}{2}
    \psi
\end{equation}
The new ``spinor'' component of the electron's state is represented in
\(\mathbb{C}^2\) as
\begin{equation}
    \hat{s}_z
    =
    \frac{1}{2}
    \begin{pmatrix}
        1 & \hphantom{+}0 \\
        0 & -1
    \end{pmatrix}
    \qquad
    \alpha
    =
    \begin{pmatrix}
        1 \\ 0
    \end{pmatrix}
    \qquad
    \beta
    =
    \begin{pmatrix}
        0 \\ 1
    \end{pmatrix}
\end{equation}
where \(\alpha\) is the ``spin-up'' state and \(\beta\) is the
``spin-down'' state, and the states of individual electrons are described not by
orbitals but by {\itshape spin-orbitals}:
\begin{equation}
    \psi(\mathbf{r}, \sigma)
    =
    \phi(\mathbf{r})\,
    \omega_\sigma
    \qquad
    \omega_\sigma
    \equiv
    \left\{
        \begin{array}{cl}
            \alpha_{\sigma}
            &
            \text{if spin projection is \(+1/2\)}
            \\[10pt]
            \beta_{\sigma}
            &
            \text{if spin projection is \(+1/2\)}
        \end{array}
    \right.
\end{equation}
which live in an extended Hilbert space,
\(
    \mathcal{H}_\mathrm{e}
    =
    L^2(\mathbb{R}^3)
    \otimes
    \mathbb{C}^2
\).
The new spin variable \(\sigma\) refers to the first or second vector component
of the spinor, which is 1 or 0 depending on whether the state is spin-up or
spin-down.

Having completed the system of quantum numbers for an electron in a spherical
potential with what would eventually be recognized as the spin projection, Pauli
was struck with a curious observation, which he discussed in January of 1925:
\begin{quote}
    By considering the case of strong magnetic fields we can reduce [earlier
    observations], that the number of electrons in a completed subgroup is the
    same as the number of corresponding terms in the Zeeman effect of the alkali
    spectra, to the following more general rule about the occurrence of
    equivalent electrons in an atom:
    {\itshape
        There can never be two or more equivalent electrons in an atom for which
        in strong fields the values of all quantum numbers \(n, l, k, m_l\) (or,
        equivalently, \(n, l, m_l, m_s\)) are the same.
        If an electron is present in the atom for which these quantum numbers
        (in an external field) have definite values, this state is ``occupied.''
    }
    \dots\
    We cannot give a further justification for this rule, but it seems to be a
    very plausible one.\footnote{%
        Translated from Ref.~\citenum{Pauli:1925p756} with emphasis added.
    }
\end{quote}
This ``housing office for equivalent orbits''\cite{Mehra:1982} would remain a
mystery until Heisenberg began studying two-electron systems in earnest with the
new quantum mechanics in 1926.


\section{Indistinguishability}

In June of 1926, Heisenberg published an article on {\itshape The Many-Body
Problem and Resonance in Quantum Mechanics}\cite{Heisenberg:1926p411} in which
he aimed to ``lay the foundations for a quantum mechanical treatment of
many-body systems.''
He sought to reconcile three outstanding problems in quantum theory as it stood:
\begin{quote}
    [1.] The aspects of de Broglie's theory of waves that lead to Bose-Einstein
    statistics appear to have no analogue in quantum mechanics;
    [2.] Ad hoc rules like Pauli's ban on equivalent orbitals have no expression
    in the mathematical formalism of quantum mechanics as formulated\dots\ 
    [3.] Finally there is one known difficulty in the quantitative
    interpretation of spectra that we should remind ourselves of:
    The splitting of singlet and triplet states in the spectra of the alkaline
    earth metals and in helium is too big by an order of magnitude to be
    explained as a difference in the magentic interaction energies of two
    rotating electrons.\footnote{%
        Translated from Ref.~\citenum{Heisenberg:1926p411}.
    }
\end{quote}

Bose-Einstein\cite{Bose:1924p178,Einstein:1924p261,Einstein:1925p3}


\begin{equation}
    \Phi_{p_1\cdots p_n}(1, \ldots, n)
    =
    \tfrac{1}{\sqrt{n!}}
    \sum_{\pi}
    (-)^\pi
    \psi_{p_1}(\pi_1)
    \cdots
    \psi_{p_n}(\pi_n)
\end{equation}



\section{Modern Electronic Structure Theory}

\begin{equation}
    a_p
    \Psi(1, \ldots, n)
    \equiv
    \sqrt{n}
    \int
    d(1)\,
    \psi_p^*(1)
    \Psi(1, \ldots, n)
\end{equation}

\begin{equation}
    1_n
    =
    \tfrac{1}{\sqrt{n}}
    \sum_p
    \psi_p(1)\,
    a_p
\end{equation}

\begin{equation}
    \langle\Psi|\hat{H}|\Psi\rangle
    =
    \sum_{pq}
    h_p^q\,
    \langle\Psi|a_p^\dagger a_q|\Psi\rangle
    +
    \tfrac{1}{2}
    \sum_{pqrs}
    g_{pq}^{rs}\,
    \langle\Psi|a_p^\dagger a_q^\dagger a_s a_r|\Psi\rangle
\end{equation}


\section{Prospectus}



\newpage
Hydrogen or silver or alkali metals
\begin{equation}
    (\hat{h} + \hat{v}_\mathrm{e})
    |\psi\rangle
    =
    \epsilon
    |\psi\rangle
    \qquad
    \hat{h}
    =
    \tfrac{1}{2}
    \hat{\mathbf{p}}^2
    +
    \hat{v}_\mathrm{n}
    \qquad
    \hat{v}_\mathrm{n}
    =
    \sum_a^{\mathrm{nuclei}}
    \frac{q_a}{|\mathbf{r}_a-\hat{\mathbf{r}}|}
\end{equation}
\begin{equation}
    \mathcal{H}
    =
    L^2(\mathbb{R}^3)
\end{equation}

\subsection{Symmetry}

Symmetry properties of the potential give us spatial symmetries
\begin{equation}
    \hat{o}
    |\psi\rangle
    =
    o
    |\psi\rangle
\end{equation}
point groups and lattice groups.
For one electron atoms the O(3) symmetries are described by the conservation of
angular momentum, \(\hat{\mathbf{j}}^2\) and \(\hat{j}_z\).

\noindent
Atoms: angular momentum (s, p, d)

\noindent
Free electrons: plane waves

\noindent
Molecules and crystals are described by symmetry adapted combinations of these
functions


Goudsmit-Uhlenbeck (Uhlenbeck:1926p264) give us spin, which is a constant intrinsic to the electron
as a particle
\begin{equation}
    \hat{s}^2
    |\psi\rangle
    =
    \tfrac{3}{4}
    |\psi\rangle
\end{equation}

The earlier Stern-Gerlach experiment (Gerlach:1922p349) showed in hindsight that
spin can be oriented up or down along a given axis
\begin{equation}
    \hat{s}_z
    |\psi\rangle
    =
    \pm
    \tfrac{1}{2}
    |\psi\rangle
\end{equation}

\begin{equation}
    \hat{s}_z
    =
    \tfrac{1}{2}
    (
        |\alpha\rangle\langle\alpha|
        -
        |\beta\rangle\langle\beta|
    )
\end{equation}

\begin{equation}
    \mathcal{H}
    =
    L^2(\mathbb{R}^3)
    \otimes
    \mathrm{SU}(2)
\end{equation}


Two problems led to particle stats: the anomalous Zeeman effect, and the absence
of electron {\itshape bunching}.
In an orbital picture, you would expect all of the electrons to fall in the
lowest energy state, such that the ground state is qualitatively described by a
product
\begin{equation}
    \Psi_0(1,\ldots,n)
    \approx
    \psi_0(1)
    \cdots
    \psi_0(n)
\end{equation}
which is indeed what you find for bosons.

Paul describing spin before people knew what it was (Pauli:1925p373):
``The closed energy configurations contribute nothing to the magnetic moment or
angular momentum of the atom.
In particular, in alkali metals the angular momentum values of the atom and its
energy changes in the presence of an external magnetic field are appropriately
interpreted as the sole working of the optically active electron, and the same
situation is thought to be the case in the observations of the anomalous Zeeman
effect.
From this standpoint, the doublet structure of the alkali spectra, as well as
the breakdown of Larmor's theorem, must therefore come from some intrinsic,
classically non-describable type of two-valuedness that is a characteristic of
the optically active electron's state.''
Pauli proposed this without proposing any model for this extra degree of
freedom.
That was the contribution of Uhlenbeck and Goudsmit in 1926.


Pauli exlusion (Pauli:1925p756):
``By considering the case of strong magnetic fields we can reduce [E.~C.]
Stoner's result, that the number of electrons in a completed subgroup [of an
atom] is the same as the number of the corresponding terms of the Zeeman effect
of the alkali spectra, to the following more general rule about the occurrence
of equivalent electrons in an atom:
{\itshape
    There can never be two or more equivalent electrons in an atom for which in
    strong fields the values of all quantum numbers \(n, k_1, k_2, m_1\) (or,
    equivalently \(n, k_1, m_1, m_2\)) are  the same.
    If an electron is present in the atom for which these quantum numbers (in an
    external field) have definite values, this state is ``occupied.''%
}
\dots We cannot give a further justification for this rule, but it seems to be a
very plausible one.
It refers, as mentioned, first of all to the case of strong field.
However, from thermodynamic arguments (the invariance of statistic weights under
adiabatic transformations of the system) it follows that the number of
stationary states of an atom must be the same in strong and weak fields for
given values of the number \(k_1\) and \(k_2\) of the separate electrons and a
value of \(\overline{m_1}=\sum m_1\) for the whole atom.''

This is in contrast to photons in stimulated emission, where they are all forced
to occupy the same state.

\subsection{Antisymmetry}

Heisenberg to Pauli (May 5, 1926) on the connection between Pauli exclusion and
the singlet triplet gap in helium (Mehra:1982):
{\itshape
    I want to write to you that I have found a rather decisive argument that
    your exclusion of equivalent orbitals is connected with the singlet-triplet
    separation [in helium].
}

Heisenberg to Born (May 26, 1926) on the challenge of calculating the energy
states of helium (Mehra:1982):
{\itshape
    The large separation between the singlet and triplet systems could not be
    explained by a magnetic interaction of the spinning electrons; one had
    therefore to assume with Hund a large force of unknown origin between the
    magnets [Hund's rule].
}
\\
He explains everything by qualitative arguments in his paper on ``The Many-Body
Problem and Resonance in Quantum Mechanics'' (Heisenberg:1926p411), which
{\itshape
    aims to lay the groundwork for a quantum-mechanical treatment of the
    many-body problem.
}
\\
He is the first to propose the form of the Slater determinant.



\section{Fock Space}

\begin{equation}
    \hat{H}
    =
    \sum_{i=1}^n
    \hat{h}(i)
    +
    \sum_{i=1}^n
    \sum_{j=i+1}^n
    \hat{g}(i,j)
\end{equation}

\begin{equation}
    \hat{g}(i,j)
    =
    \frac{1}{|\hat{\mathbf{r}}_i-\hat{\mathbf{r}}_j|}
\end{equation}

\begin{equation}
    \mathcal{H}^n
    =
    \mathcal{H}(1)\otimes\cdots\otimes\mathcal{H}(n)
\end{equation}

\begin{equation}
    |\Psi\rangle
    =
    \sum_{p_1\cdots p_n}
    c_{p_1\cdots p_n}
    |\psi_{p_1}\rangle
    \otimes
    \cdots
    \otimes
    |\psi_{p_n}\rangle
\end{equation}
