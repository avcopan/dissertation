\chapter[%
    Introduction
]{%
    Introduction\footnote{%
        Adapted from my lecture notes at
        \url{https://github.com/CCQC/chem-8950}.
    }
}

The goal of electronic structure theory is to solve the time-independent
Born-Oppenheimer Schr\"odinger equation with an optimal balance of accuracy and
efficiency for the problem of interest.
\begin{equation}
    \label{eq:electronic-schrodinger-equation}
    \hat{H}\Psi_k
    =
    E_k\Psi_k
\end{equation}
\begin{equation}
    \hat{H}
    =
    V_{\mathrm{nuc}}
    +
    \hat{H}_e
    =
    \sum_{a<b}^{\text{nuc.}}
    \frac{Z_aZ_b}{|\mathbf{R}_a-\mathbf{R}_b|}
    -
    \frac{1}{2}
    \sum_i^{\text{elec.}}
    \nabla_i^2
    -
    \sum_a^{\text{nuc.}}
    \sum_i^{\text{elec.}}
    \frac{Z_a}{|\mathbf{R}_a-\mathbf{r}_i|}
    +
    \sum_{i<j}^{\text{elec.}}
    \frac{1}{|\mathbf{r}_i-\mathbf{r}_j|}
\end{equation}
The most accurate solution possible for a given one-particle basis set of
spin-orbitals
\(
    \{\psi_p\}
\)
results from expanding the wavefunction as a linear combination of all
antisymmetrized products of these one-particle states, which are known as Slater
determinants
\(
    \{\Phi_\mu\}
\).
\begin{equation}
    \label{eq:full-ci-wavefunction-expansion}
    \Psi_k
    =
    \sum_\mu
    \Phi_\mu
    c_{\mu k}
\end{equation}
The expansion coefficients \((\mathbf{c})_k=c_{\mu k}\) are eigenvectors of the
matrix \((\mathbf{H})_{\mu\nu}=\langle\Phi_\mu|\hat{H}|\Phi_\nu\rangle\), which
is the matrix representation of the Hamiltonian in the determinant basis.
This is called the {\itshape full configuration-interaction}\ (FCI) solution.

Any one-electron basis spans the same ``function space'' as the AO basis set
itself, and the full $n$-electron basis $\{\Phi_\mu\}$ spans the same space of
$n$-electron functions regardless of how one forms spin orbitals from the AO
basis set.
As a result, one obtains the same FCI solution for any choice of spin-orbitals.
In general, however, FCI solutions are completely unfeasible for basis sets of
sufficient size to approach the complete basis set limit because the number of
determinants grows exponentially with the dimension of the one-particle basis.
The upshot is that we usually have to omit some Slater determinants in order to get an answer in a reasonable amount of time.
As soon as we truncate our determinant expansion
(\ref{eq:full-ci-wavefunction-expansion}), our choice of spin MOs makes a
significant difference in the quality of our results.
In particular, we need to choose our set of one-electron functions to minimize
the number of Slater determinants it takes to ``get close to'' the exact
wavefunction.
One approach is to determine the best single-determinant approximation to the
wavefunction by optimizing its energy expectation value with respect to
variations in the orbitals
\begin{align}
    \frac{\delta\langle\Phi|\hat{H}_e|\Phi\rangle}{\delta\psi_i}
    \overset{!}{=}
    0
    \qquad
    \Phi(1,\ldots,n)
    =
    \frac{1}{\sqrt{n!}}
    \left|
    \begin{matrix}
      \psi_1(1)&\psi_2(1)&\cdots&\psi_n(1)\\
      \psi_1(2)&\psi_2(2)&\cdots&\psi_n(2)\\
      \vdots    &\vdots    &\ddots&\vdots    \\
      \psi_1(n)&\psi_2(n)&\cdots&\psi_n(n)
    \end{matrix}
    \right|
\end{align}
where the arguments \(1,\ldots,n\) are short hand for the spin,
\(s_1,\ldots,s_n\), and spatial, \(\mathbf{r}_1,\ldots,\mathbf{r}_n\), degrees
of freedom of individual electrons.

