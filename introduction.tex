\chapter[%
    Introduction and Literature Review
]{%
    Introduction and Literature Review
}

\section{Naive Electronic Structure Theory}
\label{sec:introduction:naive-electronic-structure}

As an entry point into electronic structure theory, let us begin by forgetting
what we know about electrons from the standard model of particle physics.
From the standpoint of Heisenberg and others developing the new quantum theory
in 1925\cite{Heisenberg:1925p879} chemical matter was described by nimble,
negatively-charged electrons orbiting heavy, positively-charged nuclei.
This theory would be conceptually clarified in the following year by
Schr\"odinger's development of wave mechanics,
\cite{Schrodinger:1926p361,Schrodinger:1926p489,Schrodinger:1926p734} which
described the possible electronic states of an isolated molecule as
eigenfunctions of the quantum-mechanical Hamiltonian, oscillating in time with a
frequency proportional to their energy.
In atomic units:
\begin{equation}
    \Psi(t)
    =
    \Psi
    e^{-iEt}
    \qquad
    \hat{H}
    \Psi
    =
    E
    \Psi
\end{equation}
Crudely speaking, the Hamiltonian operator is derived from its classical
counterpart by replacing momentum variables with del operators divided by the
imaginary unit,
\(
    \hat{\mathbf{p}}
    =
    \frac{1}{i}
    \nabla
\).
It can be written as a sum over one and two electron terms
\begin{equation}
    \hat{H}
    =
    \sum_i^\text{electrons}
    \hat{h}_i
    +
    \sum_{i<j}^{\substack{\text{electron}\\\text{pairs}}}
    \hat{g}_{ij}
\end{equation}
where the one-electron operator
\(
    \hat{h}_i
\)
describes the kinetic energy of the \(i^\text{th}\) electron and its
electrostatic (Coulomb's law) attraction to the nuclei, and the two-electron
operator
\(
    \hat{g}_{ij}
\)
describes the Coulombic repulsion between electrons \(i\) and \(j\).
\begin{equation}
    \hat{h}_i
    \equiv
    \tfrac{1}{2}
    \hat{\mathbf{p}}_i^2
    -
    \sum_A^\text{nuclei}
    \frac{Z_A}{|\mathbf{r}_A - \mathbf{r}_i|}
    \qquad
    \hat{g}_{ij}
    \equiv
    \frac{1}{|\mathbf{r}_i - \mathbf{r}_j|}
\end{equation}
The vector space containing the wavefunction is the system's Hilbert space,
\(\mathcal{H}\), which in our case is the space of square integrable functions
of \(n\) position variables,
\(
    L^2(\mathbb{R}^{3n})
\),
one for each electron in the molecule.
We are helped by the fact that the electronic Hilbert space decomposes into a
product of one-electron Hilbert spaces
\begin{equation}
    \mathcal{H}
    =
    \mathcal{H}_\mathrm{e}
    \otimes
    \mathcal{H}_\mathrm{e}
    \otimes
    \cdots
    \qquad
    \mathcal{H}_\mathrm{e}
    =
    L^2(\mathbb{R}^3)
\end{equation}
which allows us to expand the electronic wavefunction as a linear combination of
products of one-electron states.
These one-electron states are known as {\itshape orbitals}.

A convenient orbital basis for many-electron systems is the solution set of the
following equation, which is an effective Schr\"odinger equation for one
electron in the mean Coulombic field generated by the other electrons.
\begin{equation}
    \label{eq:introduction:mean-field-orbitals}
    (
        \hat{h}_1
        +
        \hat{v}_1
    )
    \phi_i(\mathbf{r}_1)
    =
    \epsilon_i\,
    \phi_i(\mathbf{r}_1)
    \qquad
    \hat{v}_1
    \equiv
    \sum_{j\neq i}
    \int
    d^3\mathbf{r}_2\,
    \frac{%
        \phi_j^*(\mathbf{r}_2)\phi_j(\mathbf{r}_2)
    }{%
        |\mathbf{r}_1 - \mathbf{r}_2|
    }
\end{equation}
This can be shown to minimize energy of a single product of orbitals
\begin{equation}
    \label{eq:introduction:orbital-product-expectation-value}
    \langle \phi_1\cdots \phi_n|
    \hat{H}
    |\phi_1\cdots \phi_n \rangle
    =
    \sum_{i=1}^\mathrm{orbitals}
    h_i^i
    +
    \sum_{i<j}^{\substack{\mathrm{orbital}\\\mathrm{pairs}}}
    g_{ij}^{ij}
\end{equation}
which is a sum over the following {\itshape one-} and {\itshape two-electron
integrals}.
\begin{equation}
    h_p^q
    \equiv
    \int
    d^3\mathbf{r}_1\,
    \phi_p^*(\mathbf{r}_1)
    \hat{h}_1
    \phi_q(\mathbf{r}_1)
    \qquad
    g_{pq}^{rs}
    \equiv
    \int
    d^3\mathbf{r}_1
    d^3\mathbf{r}_2\,
    \phi_p^*(\mathbf{r}_1)
    \phi_q^*(\mathbf{r}_2)
    \hat{g}_{12}
    \phi_r(\mathbf{r}_1)
    \phi_s(\mathbf{r}_2)
\end{equation}
\cref{eq:introduction:mean-field-orbitals} is most effective for {\itshape
weakly correlated} electronic states, where the electron probability density
approximately factors into one-electron densities and the wavefunction is well
approximated by a single product.

Having determined a basis for \(\mathcal{H}_e\), the Schr\"odinger equation can
be solved as
\begin{equation}
    \mathbf{H}\mathbf{c}
    =
    E\mathbf{c}
    \qquad
    (\mathbf{H})_{PQ}
    =
    \langle \phi_{p_1}\cdots \phi_{p_n}|
    \hat{H}
    |\phi_{q_1}\cdots \phi_{q_n} \rangle
\end{equation}
which is exactly equivalent to the Sch\"odinger equation in the limit of a
complete expansion.
The coefficients of the solution vector are the components of the wavefunction
along each orbital product.

The general strategy we have just outlined carries over into modern electronic
structure theory, but it is missing two essential ingredients: the spin of the
electron, and the antisymmetric permutational symmetry of electrons as fermions.


\section{Spin}

The non-relativistic theory of one-electron states was completed with Pauli's
solution of the hydrogen spectrum at the start of 1926.\cite{Pauli:1926p336}
His work concludes with a discussion of the recent work by Goudsmit and
Uhlenbeck\cite{Uhlenbeck:1925p953} showing that the anomalous Zeeman splitting
of alkali metals could be explained by positing an intrinsic source of angular
momentum and magnetism for the electron besides that generated by its orbital
motion about the nucleus.
This was the electron's spin.
The need for an additional quantum number had already been understood by Pauli
in his analysis of alkali metal spectra at the end of 1924:
\begin{quote}
    In alkali metals the angular momentum values of the atom, and its energy
    changes in the presence of an external magnetic field, are appropriately
    interpreted as the sole working of the optically active electron, and the
    same situation is thought to be the case in observations of the anomalous
    Zeeman effect.
    From this standpoint, the doublet structure of the alkali
    spectra, as well as the breakdown of Larmor's theorem, must
    therefore come from some intrinsic, classically non-describable
    type of two-valuedness that is a characteristic of the optically
    active electron.\cite{Pauli:1925p373}
\end{quote}
In hindsight, the Stern-Gerlach experiment\cite{Gerlach:1922p349} had already
shown in 1922 that the 5s electron of the silver atom was quantized into two
magnetic states, whereas the new quantum theory predicted an odd number of
states (\(0, \pm1, \pm2, \dots\)) for the spatial orbits of a charged particle.
This new source of angular momentum was characterized by half-integer values
whose eigenfunctions cannot exist in \(L^2(\mathbb{R}^3)\).
\begin{equation}
    \hat{s}_z
    \psi
    =
    \pm
    \tfrac{1}{2}
    \psi
\end{equation}
Instead, the new ``spinor'' part of the one-electron state was encoded in a
two-component vector space
\begin{equation}
    \hat{s}_z
    =
    \begin{pmatrix}
        \tfrac{1}{2} & \hphantom{+}0 \\
        0 & -\tfrac{1}{2}
    \end{pmatrix}
    \qquad
    \alpha
    =
    \begin{pmatrix}
        1 \\ 0
    \end{pmatrix}
    \qquad
    \beta
    =
    \begin{pmatrix}
        0 \\ 1
    \end{pmatrix}
\end{equation}
where \(\alpha\) is the ``up'' spinor and \(\beta\) is the ``down'' spinor.
The states of individual electrons therefore had to be described not by orbitals
but by {\itshape spin-orbitals}:
\begin{equation}
    \psi(\mathbf{x})
    =
    \phi(\mathbf{r})\,
    \omega_\sigma
    \qquad
    \omega_\sigma
    \equiv
    \left\{
        \begin{array}{cl}
            \alpha_{\sigma}
            &
            \text{if the spin projection is \(+\tfrac{1}{2}\)}
            \\[10pt]
            \beta_{\sigma}
            &
            \text{if the spin projection is \(-\tfrac{1}{2}\)}
        \end{array}
    \right.
\end{equation}
which live in an extended one-electron Hilbert space,
\(
    \mathcal{H}_\mathrm{e}
    =
    L^2(\mathbb{R}^3)
    \otimes
    \mathbb{C}^2
\),
whose degrees of freedom are described by a pair of space and spin variables,
\(
    \mathbf{x}
    \equiv
    (\mathbf{r},\sigma)
\).
The spin variable \(\sigma\) refers to the first or second vector component of
the spinor, which evaluates to 1 or 0 depending on whether the state is spin-up
or spin-down.

Having completed the system of quantum numbers for an electron in a spherical
potential with what would eventually be recognized as spin, Pauli was struck
with a curious observation:
\begin{quote}
    By considering the case of strong magnetic fields we can reduce [earlier
    observations], that the number of electrons in a completed subgroup is the
    same as the number of corresponding terms in the Zeeman effect of the alkali
    spectra, to the following more general rule about the occurrence of
    equivalent electrons in an atom:
    There can never be two or more equivalent electrons in an atom for which
    in strong fields the values of all quantum numbers \(n, l, k, m_l\) (or,
    equivalently, \(n, l, m_l, m_s\)) are the same.
    If an electron is present in the atom for which these quantum numbers
    (in an external field) have definite values, this state is ``occupied.''
    \dots\
    We cannot give a further justification for this rule, but it seems to be a
    very plausible one.\cite{Pauli:1925p756}
\end{quote}
This ``housing office for equivalent orbits''\cite{Mehra:1982} would remain a
mystery until Heisenberg's work on two-electron systems the following year.


\section{Antisymmetry}

In June of 1926, Heisenberg published an article on {\itshape The Many-Body
Problem and Resonance in Quantum Mechanics}, which sought to address
foundational issues arising in his attempts to apply the new quantum theory to
the helium atom.
In his words, there were three outstanding problems:
\begin{quote}
    [1.] The aspects of de Broglie's theory of waves that lead to Bose-Einstein
    statistics appear to have no analogue in quantum mechanics;
    [2.] Ad hoc rules like Pauli's ban on equivalent orbitals cannot be
    expressed in the current mathematical formalism of quantum mechanics\dots\ 
    [3.] Finally there is one known difficulty in the quantitative
    interpretation of spectra that we should remind ourselves of:
    The splitting of singlet and triplet states in the spectra of the alkaline
    earth metals and in helium is too big by an order of magnitude to be
    explained as a difference in the magnetic interaction energies of two
    spinning electrons.\cite{Heisenberg:1926p411}
\end{quote}
By considering two quantum harmonic oscillators and treating them as
indistinguishable in the surprising sense discovered by Bose\cite{Bose:1924p178}
and Einstein\cite{Einstein:1924p261} two years prior, Heisenberg found that the
eigenstates of the coupled system could exist in symmetric or antisymmetric
combinations, and that only the antisymmetric states were consistent with the
spectroscopic observations for helium.
This raised the intriguing possibility that, as a rule, the electronic
wavefunction was antisymmetric under particle exchange.
If so, the appropriate basis state would not be the orbital product, which
places each electron into its own distinct orbit, but the determinant function:
\begin{equation}
    \Phi_{p_1\cdots p_n}(\mathbf{x}_1, \ldots, \mathbf{x}_n)
    =
    \tfrac{1}{\sqrt{n!}}
    \sum_\pi
    (-)^\pi
    \psi_{p_1}(\mathbf{x}_{\pi_1})
    \cdots
    \psi_{p_n}(\mathbf{x}_{\pi_n})
\end{equation}
where \(\pi\) is a permutation of the electron labels \(1\cdots n\) and
\((-)^\pi\) is its signature.
Whenever an orbital appears twice in a product its determinant vanishes, so this
provided a mathematical basis for Pauli exclusion.
The determinant's energy is
\begin{equation}
    \label{eq:introduction:determinant-expectation-value}
    \langle \Phi_{1\cdots n}|
    \hat{H}
    |\Phi_{1\cdots n} \rangle
    =
    \sum_{i=1}^\mathrm{orbitals}
    h_i^i
    +
    \sum_{i<j}^{\substack{\mathrm{orbital}\\\mathrm{pairs}}}
    \overline{g}_{ij}^{ij}
    \qquad
    \overline{g}_{ij}^{ij}
    \equiv
    g_{ij}^{ij}
    -
    g_{ij}^{ji}
\end{equation}
which is the same as the product expectation value of
\cref{eq:introduction:orbital-product-expectation-value} except for the
\(
    g_{ij}^{ji}
\)
integrals.
These ``exchange interactions'' between electrons in orbitals \(i\) and \(j\)
serve to lower the energy by cancelling out part of the Coulomb repulsion where
they overlap.
The orthogonality of opposite spinors means that this effect only takes place
between electrons of the same spin, lowering the energy of high-spin states
relative to low-spin ones of the same configuration, as had been observed by
Hund.\cite{Hund:1925p345}
Thus the new theory could also explain the ``large force of unknown
origin''\cite{Mehra:1982} that was lowering the energy of the triplet states in
helium.


\section{Modern Electronic Structure Theory}

Heisenberg's insights laid the foundation for a quantum mechanical treatment of
many-electron systems,\cite{Heisenberg:1926p411} but the determinant functions
needed to describe these antisymmetric states were difficult to work with.
This challenge was addressed in a 1932 article by Vladimir Fock, which developed
a new mathematical framework for indistinguishable particles that he called
{\itshape second quantization}.\cite{Fock:1932p622}
Building on earlier work by Dirac,\cite{Dirac:1927p243} the new formalism
replaced the opaque combinatorial arguments of the previous ``first quantized''
formalism with transparent algebraic manipulations.
Center stage in the new approach was the annihilation operator:
\begin{equation}
    (\hat{a}_p\Psi)(\mathbf{x}_2, \ldots, \mathbf{x}_n)
    \equiv
    \sqrt{n}
    \int
    d^4\mathbf{x}'\,
    \psi_p^*(\mathbf{x}')\,
    \Psi(\mathbf{x}', \mathbf{x}_2, \ldots, \mathbf{x}_n)
\end{equation}
whose physical meaning becomes clear from its effects on the determinant basis.
\begin{equation}
    \hat{a}_{p_k}
    \Phi_{p_1\cdots p_n}
    =
    (-)^{k-1}
    \Phi_{p_1\cdots p_{k-1} p_{k+1}\cdots p_n}
\end{equation}
\begin{equation}
    \hat{a}_{p_k}^\dagger
    \Phi_{p_1\cdots p_{k-1} p_{k+1}\cdots p_n}
    =
    (-)^{k-1}
    \Phi_{p_1\cdots p_n}
\end{equation}
\begin{equation}
    \begin{array}{c}
        \\
        \hat{a}^{q_1q_2\cdots}_{p_ip_j\cdots}
        \Phi_{%
            p_1\cdots
            p_{i-1} p_i p_{i+1}\cdots
            p_{j-1} p_j p_{j+1}\cdots
            p_n
        }
        =
        \Phi_{%
            p_1\cdots
            p_{i-1} q_1 p_{i+1}\cdots
            p_{j-1} q_2 p_{j+1}\cdots
            p_n
        }
        \\[15pt]
        \text{
            \hfill
            (%
                where
                \(
                    \hat{a}^{q_1\cdots q_h}_{p_1\cdots p_h}
                    \equiv
                    \hat{a}_{q_1}^\dagger
                    \cdots
                    \hat{a}_{q_h}^\dagger
                    \hat{a}_{p_h}
                    \cdots
                    \hat{a}_{p_1}
                \)%
            )
        }
        \\[15pt]
    \end{array}
\end{equation}
In words, the annihilation operator deletes a spin-orbital and renormalizes the
state.
Its adjoint constitutes a {\itshape creation operator} which adds a
spin-orbital, and we can string these operators together to form an {\itshape
excitation operator} which substitutes one set of spin-orbitals in the
determinant with another.
In each case, an invalid operation, such as creating an occupied state or
annihilating an unoccupied state, causes the determinant to vanish.
This allows us to expand the wavefunction in terms of single, double, triple,
etc.\ excitations of a {\itshape reference determinant}
\begin{equation}
    \label{eq:intro-ci-expansion}
    \Psi
    =
    \left(
        c_0
        \hat{1}
        +
        c_a^i
        \hat{a}^a_i
        +
        (\tfrac{1}{2})^2
        c_{ab}^{ij}
        \hat{a}^{ab}_{ij}
        +
        (\tfrac{1}{3!})^2
        c_{abc}^{ijk}
        \hat{a}^{abc}_{ijk}
        +
        \cdots
    \right)
    \Phi
\end{equation}
where \(i, j, k\) count over states which are occupied in the determinant, \(a,
b, c\) count over unoccupied states, and we have adopted the Einstein summation
convention for summing over repeated indices.
For weakly correlated states, the reference determinant can be chosen so that
\(c_0 \approx 1\) and the coefficients become negligibly small for higher than
quadruple excitations, allowing us to truncate this expansion to a good
approximation.
Finally, by substituting the following decomposition
\begin{equation}
    \Psi(\mathbf{x}_1, \mathbf{x}_2, \ldots, \mathbf{x}_n)
    =
    \tfrac{1}{\sqrt{n}}
    \sum_p
    \psi_p(\mathbf{x}_1)\,
    (\hat{a}_p\Psi)(\mathbf{x}_2, \ldots, \mathbf{x}_n)
\end{equation}
into a general antisymmetric matrix element of the Hamiltonian, we find that the
restriction of \(\hat{H}\) to antisymmetric states can be expressed in the
following form.
\begin{equation}
    \hat{H}
    =
    h_p^q\,
    \hat{a}^p_q
    +
    \tfrac{1}{4}
    \overline{g}_{pq}^{rs}\,
    \hat{a}^{pq}_{rs}
\end{equation}
This reduces the evaluation of Hamiltonian matrix elements to the algebra of
creation and annihilation operators.
These second quantized operators obey simple anticommutation relationships that
encode the permutational antisymmetry of the electrons, and several mathematical
tools have been developed to facilitate the evaluation of their products and
expectation values.
Most notable is the expansion theorem published by Gian-Carlo Wick in
1950.\cite{Wick:1950p268}


\section{Density Cumulants}

\afterpage{%
    \clearpage
    \centering
    \begin{landscape}
        \vspace*{\fill}
        \captionof{table}{%
            The moment-cumulant relations, where
            \(
                \gamma^{p_1\cdots p_h}_{q_1\cdots q_h}
                \equiv
                \langle\Psi|a^{p_1\cdots p_h}_{q_1\cdots q_h}|\Psi\rangle
            \)
            and
            \(
                \lambda^{p_1\cdots p_h}_{q_1\cdots q_h}
                \equiv
                \langle\Psi|
                a^{p_1\cdots p_h}_{q_1\cdots q_h}
                |\Psi\rangle_\mathrm{c}
            \)
            are the \(h\)-body moments and cumulants of the wavefunction
            density and we show the expansions for \(h=1, 2, 3, 4\).
            Here, \(P_{(R_1/R_2/\cdots/R_m)}\) denotes antisymmetrization over
            riffle shuffle permutations of \(m\) sets of indices.
        }
        \small
        \renewcommand\arraystretch{1.0}
        \vspace{10pt}
        \begin{tabular}{ccc}
            \(
                h
            \)
            &
            moment (\(\gamma\))
            &
            cumulant (\(\lambda\))
            \\\hline
            1
            &
            \(
                \lambda^{p_1}_{q_1}
            \)
            &
            \(
                \gamma^{p_1}_{q_1}
            \)
            \\
            2
            &
            \(
                \lambda^{p_1p_2}_{q_1q_2}
                +
                P_{(q_1/q_2)}
                \lambda^{p_1}_{q_1}
                \lambda^{p_2}_{q_2}
            \)
            &
            \(
                \gamma^{p_1p_2}_{q_1q_2}
                -
                P_{(q_1/q_2)}
                \gamma^{p_1}_{q_1}
                \gamma^{p_2}_{q_2}
            \)
            \\
            3
            &
            \(
                \lambda^{p_1p_2p_3}_{q_1q_2q_3}
                +
                P_{(q_1q_2/q_3)}^{(p_1p_2/p_3)}
                \lambda^{p_1p_2}_{q_1q_2}
                \lambda^{p_3}_{q_3}
                +
                P_{(q_1/q_2/q_3)}
                \lambda^{p_1}_{q_1}
                \lambda^{p_2}_{q_2}
                \lambda^{p_3}_{q_3}
            \)
            &
            \(
                \gamma^{p_1p_2p_3}_{q_1q_2q_3}
                -
                P_{(q_1q_2/q_3)}^{(p_1p_2/p_3)}
                \gamma^{p_1p_2}_{q_1q_2}
                \gamma^{p_3}_{q_3}
                +
                2
                P_{(q_1/q_2/q_3)}
                \gamma^{p_1}_{q_1}
                \gamma^{p_2}_{q_2}
                \gamma^{p_3}_{q_3}
            \)
            \\[10pt]
            4
            &
            \(
                \begin{array}{c}
                    \lambda^{p_1p_2p_3p_4}_{q_1q_2q_3q_4}
                    +
                    P_{(q_1q_2q_3/q_4)}^{(p_1p_2p_3/p_4)}
                    \lambda^{p_1p_2p_3}_{q_1q_2q_3}
                    \lambda^{p_4}_{q_4}
                    +
                    P_{(q_1q_2/q_3q_4)}^{(p_2/p_3p_4)}
                    \lambda^{p_1p_2}_{q_1q_2}
                    \lambda^{p_3p_4}_{q_3q_4}
                    \\
                    +
                    P_{(q_1q_2/q_3/q_4)}^{(p_1p_2/p_3p_4)}
                    \lambda^{p_1p_2}_{q_1q_2}
                    \lambda^{p_3}_{q_3}
                    \lambda^{p_4}_{q_4}
                    +
                    P_{(q_1/q_2/q_3/q_4)}
                    \lambda^{p_1}_{q_1}
                    \lambda^{p_2}_{q_2}
                    \lambda^{p_3}_{q_3}
                    \lambda^{p_4}_{q_4}
                \end{array}
            \)
            &
            \(
                \begin{array}{c}
                    \gamma^{p_1p_2p_3p_4}_{q_1q_2q_3q_4}
                    -
                    P_{(q_1q_2q_3/q_4)}^{(p_1p_2p_3/p_4)}
                    \gamma^{p_1p_2p_3}_{q_1q_2q_3}
                    \gamma^{p_4}_{q_4}
                    +
                    P_{(q_1q_2/q_3q_4)}^{(p_2/p_3p_4)}
                    \gamma^{p_1p_2}_{q_1q_2}
                    \gamma^{p_3p_4}_{q_3q_4}
                    \\
                    -
                    2
                    P_{(q_1q_2/q_3/q_4)}^{(p_1p_2/p_3p_4)}
                    \gamma^{p_1p_2}_{q_1q_2}
                    \gamma^{p_3}_{q_3}
                    \gamma^{p_4}_{q_4}
                    +
                    6
                    P_{(q_1/q_2/q_3/q_4)}
                    \gamma^{p_1}_{q_1}
                    \gamma^{p_2}_{q_2}
                    \gamma^{p_3}_{q_3}
                    \gamma^{p_4}_{q_4}
                \end{array}
            \)
        \end{tabular}
        \vspace*{\fill}
    \end{landscape}
}

It energy expectation value with respect to a generic state is given by
\begin{equation}
    \langle\Psi|
    \hat{H}
    |\Psi\rangle
    =
    h_p^q
    \gamma^p_q
    +
    \tfrac{1}{4}
    \overline{g}_{pq}^{rs}
    \gamma^{pq}_{rs}
\end{equation}
in terms of the one- and two-body reduced density matrices.
\begin{equation}
    \gamma^{p_1\cdots p_h}_{q_1\cdots q_h}
    \equiv
    \langle\Psi|\hat{a}^{p_1\cdots p_h}_{q_1\cdots q_h}|\Psi\rangle
\end{equation}
In principle, all of the information needed to evaluate the energy is contained
in the two-body density matrix, but the structure of this tensor is quite
important.
Truncating the linear expansion of \cref{eq:intro-ci-expansion} at some
excitation level is known to produce artifactual ``unlinked'' contributions to
the density matrix which spoil the qualitative behavior of the energy with
respect to system size and independent subsystems.
The correct structure with respect to connectivity is defined by the cumulant
expansion.\cite{McCullagh:1987}
\begin{equation}
    \langle\Psi|\hat{Q}|\Psi\rangle
    =
    \sum_{k=1}^n
    \sum_{\Pi}^{\mathrm{Part}(\hat{Q}, k)}
    (-)^{\Pi}
    \langle\Psi|\hat{\Pi}_1|\Psi\rangle_\mathrm{c}
    \cdots
    \langle\Psi|\hat{\Pi}_k|\Psi\rangle_\mathrm{c}
\end{equation}
Here, \(\hat{Q}\) is an excitation operator, \(\Pi\) is a partition of the
operator string into \(k\) parts, \((-)^\Pi\) is the signature of the
permutation, and the subscript `\(\mathrm{c}\)' denotes the connected
contributions to the expectation value.
The inverse relationship is given by
\begin{equation}
    \langle\Psi|\hat{Q}|\Psi\rangle_\mathrm{c}
    =
    \sum_{k=1}^n
    (-)^{k+1}
    (k-1)!
    \sum_{\Pi}^{\mathrm{Part}(\hat{Q}, k)}
    (-)^{\Pi}
    \langle\Psi|\hat{\Pi}_1|\Psi\rangle
    \cdots
    \langle\Psi|\hat{\Pi}_k|\Psi\rangle
\end{equation}
which defines the electronic density cumulant.
\begin{equation}
    \lambda^{p_1\cdots p_h}_{q_1\cdots q_h}
    \equiv
    \langle\Psi|\hat{a}^{p_1\cdots p_h}_{q_1\cdots q_h}|\Psi\rangle_\mathrm{c}
\end{equation}



\section{Prospectus}

Density cumulant theory (DCT) expresses the energy as a functional of the
two-body cumulant.
The first variant of the theory was proposed by Kutzelnigg in
2006,\cite{Kutzelnigg:2006p171101} and was first implemented by Simmonett four
years later.\cite{Simmonett:2010p174122}
The theory was further developed by Sokolov, who developed several new
variants.\cite{Sokolov:2013p024107,Sokolov:2013p204110,Sokolov:2014p074111}

\cref{ch:benchmark} presents a benchmark study of the ground-state DCT variants,
demonstrating that the best variant of the theory to date, ODC-12, consistently
outperforms the popular CCSD method for the description of thermochemical and
kinetic processes.
\cref{ch:response} presents the main contribution of this work, which is the
extension of DCT for the description of excited states via linear response
theory.
We present a theoretical framework for LR-DCT, derive the linear response
working equations for the ODC-12 model, as well as the initial implementation
and verification of the theory.
We also present the linearized variant of our model, which is the first
implementation of a linear response theory for the orbital-optimized linearized
coupled-cluster doubles (OLCCD) method.
The benchmarks of the LR-ODC-12 model compared with EOM-CCSD look very
promising.
For well-behaved systems, the mean absolute errors are consistently halved by
the new method.
For the more challenging excited states of ethylene, butadiene, and hexatriene
the differences are more stark.
In the most extreme case, EOM-CCSD misses the \(2{}^1\mathrm{A_g}\) state of
hexatriene by nearly 1~eV, whereas LR-ODC-12 is within 0.15~eV of the reference
value and predicts this states energy gap with the \(1{}^1\mathrm{B_u}\) to
nearly 0.01~eV.
Finally, \cref{ch:davidson} presents the algorithms that were used for our study
of larger systems, particularly hexatriene.
