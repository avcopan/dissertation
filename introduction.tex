\chapter[%
    Introduction and Literature Review
]{%
    Introduction and Literature Review
}

The general strategy of many-body quantum mechanics:
\begin{enumerate}
    \item
        Determine the Hilbert spaces for the individual particles in your
        system, ideally by solving an effective one-body problem for the
        stationary states.
    \item
        The Hilbert space for the many-body system is the tensor-product of
        one-particle Hilbert spaces, i.e.~the many-body states are spanned by
        all possible products of one-particle states.
    \item
        Solve for the many-body wavefunction in the basis one or more of these
        product functions.
        If the particles are approximately decoupled, and the one-body basis is
        carefully chosen, one may be able to describe the wavefunction to a good
        approximation using only one or a few product functions.
    \item
        If one or more particles are highly coupled, one may need to include
        larger expansions, such as all possible combinations of states in a
        given energy range, in order to achieve even qualitative accuracy.
\end{enumerate}

\section{One-Electron States}

Hydrogen or silver or alkali metals
\begin{equation}
    (\hat{h} + \hat{v}_\mathrm{e})
    |\psi\rangle
    =
    \epsilon
    |\psi\rangle
    \qquad
    \hat{h}
    =
    \tfrac{1}{2}
    \hat{\mathbf{p}}^2
    +
    \hat{v}_\mathrm{n}
    \qquad
    \hat{v}_\mathrm{n}
    =
    \sum_a^{\mathrm{nuclei}}
    \frac{q_a}{|\mathbf{r}_a-\hat{\mathbf{r}}|}
\end{equation}
\begin{equation}
    \mathcal{H}
    =
    L^2(\mathbb{R}^3)
\end{equation}
Assume the motion of the nuclei is infinitely slow on the time-scale of
electronic motion (Born-Oppenheimer approximation).\\
Assume the electron only sees an average field describing the shielding due to
the other electrons (mean-field approximation).

\subsection{Symmetry}

Symmetry properties of the potential give us spatial symmetries
\begin{equation}
    \hat{o}
    |\psi\rangle
    =
    o
    |\psi\rangle
\end{equation}
point groups and lattice groups.
For one electron atoms the O(3) symmetries are described by the conservation of
angular momentum, \(\hat{\mathbf{j}}^2\) and \(\hat{j}_z\).

\noindent
Atoms: angular momentum (s, p, d)

\noindent
Free electrons: plane waves

\noindent
Molecules and crystals are described by symmetry adapted combinations of these
functions

\subsection{Naive \(n\)-Electron States and the Mean Field}
The mean field of the electrons could be derived by analyzing the full
Schr\"odinger equation
\begin{equation}
    \hat{H}
    |\Psi\rangle
    =
    E
    |\Psi\rangle
    \qquad
    \hat{H}_1
    =
    \sum_{i=1}^n
    \hat{h}_i
    +
    \sum_{i=1}^n
    \sum_{j=i+1}^n
    \hat{g}_{ij}
    \qquad
    \hat{g}_{ij}
    =
    \frac{1}{|\hat{\mathbf{r}}_i-\hat{\mathbf{r}}_j|}
\end{equation}
\begin{equation}
    \mathcal{H}_n
    =
    \mathcal{H}_1
    \otimes
    \cdots
    \otimes
    \mathcal{H}_1
\end{equation}

Mean field is described by assuming statistically independent electrons.

Discuss how symmetries combine for the \(n\)-electron state.

\section{Spin and Pauli Exclusion}

Goudsmit-Uhlenbeck (Uhlenbeck:1926p264) give us spin, which is a constant intrinsic to the electron
as a particle
\begin{equation}
    \hat{s}^2
    |\psi\rangle
    =
    \tfrac{3}{4}
    |\psi\rangle
\end{equation}

The earlier Stern-Gerlach experiment (Gerlach:1922p349) showed in hindsight that
spin can be oriented up or down along a given axis
\begin{equation}
    \hat{s}_z
    |\psi\rangle
    =
    \pm
    \tfrac{1}{2}
    |\psi\rangle
\end{equation}

\begin{equation}
    \hat{s}_z
    =
    \tfrac{1}{2}
    (
        |\alpha\rangle\langle\alpha|
        -
        |\beta\rangle\langle\beta|
    )
\end{equation}

\begin{equation}
    \mathcal{H}
    =
    L^2(\mathbb{R}^3)
    \otimes
    \mathrm{SU}(2)
\end{equation}


Two problems led to particle stats: the anomalous Zeeman effect, and the absence
of electron {\itshape bunching}.
In an orbital picture, you would expect all of the electrons to fall in the
lowest energy state, such that the ground state is qualitatively described by a
product
\begin{equation}
    \Psi_0(1,\ldots,n)
    \approx
    \psi_0(1)
    \cdots
    \psi_0(n)
\end{equation}
which is indeed what you find for bosons.

Paul describing spin before people knew what it was (Pauli:1925p373):
``The closed energy configurations contribute nothing to the magnetic moment or
angular momentum of the atom.
In particular, in alkali metals the angular momentum values of the atom and its
energy changes in the presence of an external magnetic field are appropriately
interpreted as the sole working of the optically active electron, and the same
situation is thought to be the case in the observations of the anomalous Zeeman
effect.
From this standpoint, the doublet structure of the alkali spectra, as well as
the breakdown of Larmor's theorem, must therefore come from some intrinsic,
classically non-describable type of two-valuedness that is a characteristic of
the optically active electron's state.''
Pauli proposed this without proposing any model for this extra degree of
freedom.
That was the contribution of Uhlenbeck and Goudsmit in 1926.


Pauli exlusion (Pauli:1925p756):
``By considering the case of strong magnetic fields we can reduce [E.~C.]
Stoner's result, that the number of electrons in a completed subgroup [of an
atom] is the same as the number of the corresponding terms of the Zeeman effect
of the alkali spectra, to the following more general rule about the occurrence
of equivalent electrons in an atom:
{\itshape
    There can never be two or more equivalent electrons in an atom for which in
    strong fields the values of all quantum numbers \(n, k_1, k_2, m_1\) (or,
    equivalently \(n, k_1, m_1, m_2\)) are  the same.
    If an electron is present in the atom for which these quantum numbers (in an
    external field) have definite values, this state is ``occupied.''%
}
\dots We cannot give a further justification for this rule, but it seems to be a
very plausible one.
It refers, as mentioned, first of all to the case of strong field.
However, from thermodynamic arguments (the invariance of statistic weights under
adiabatic transformations of the system) it follows that the number of
stationary states of an atom must be the same in strong and weak fields for
given values of the number \(k_1\) and \(k_2\) of the separate electrons and a
value of \(\overline{m_1}=\sum m_1\) for the whole atom.''

This is in contrast to photons in stimulated emission, where they are all forced
to occupy the same state.

\subsection{Antisymmetry}

Heisenberg to Pauli (May 5, 1926) on the connection between Pauli exclusion and
the singlet triplet gap in helium (Mehra:1982):
{\itshape
    I want to write to you that I have found a rather decisive argument that
    your exclusion of equivalent orbitals is connected with the singlet-triplet
    separation [in helium].
}

Heisenberg to Born (May 26, 1926) on the challenge of calculating the energy
states of helium (Mehra:1982):
{\itshape
    The large separation between the singlet and triplet systems could not be
    explained by a magnetic interaction of the spinning electrons; one had
    therefore to assume with Hund a large force of unknown origin between the
    magnets [Hund's rule].
}
\\
He explains everything by qualitative arguments in his paper on ``The Many-Body
Problem and Resonance in Quantum Mechanics'' (Heisenberg:1926p411), which
{\itshape
    aims to lay the groundwork for a quantum-mechanical treatment of the
    many-body problem.
}
\\
He is the first to propose the form of the Slater determinant.
\begin{equation}
    \Phi
    =
    \sum_{\pi\in\mathrm{S}_n}
    (-)^\pi
    \phi_(\pi(1))
    \cdots
    \phi_n(\pi(n))
\end{equation}



\section{Fock Space}

\begin{equation}
    \hat{H}
    =
    \sum_{i=1}^n
    \hat{h}(i)
    +
    \sum_{i=1}^n
    \sum_{j=i+1}^n
    \hat{g}(i,j)
\end{equation}

\begin{equation}
    \hat{g}(i,j)
    =
    \frac{1}{|\hat{\mathbf{r}}_i-\hat{\mathbf{r}}_j|}
\end{equation}

\begin{equation}
    \mathcal{H}^n
    =
    \mathcal{H}(1)\otimes\cdots\otimes\mathcal{H}(n)
\end{equation}

\begin{equation}
    |\Psi\rangle
    =
    \sum_{p_1\cdots p_n}
    c_{p_1\cdots p_n}
    |\psi_{p_1}\rangle
    \otimes
    \cdots
    \otimes
    |\psi_{p_n}\rangle
\end{equation}
