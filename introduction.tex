\chapter[%
    Introduction
]{%
    Introduction\footnote{%
        Adapted from my lecture notes at
        \url{https://github.com/CCQC/chem-8950}.
    }
}

The goal of electronic structure theory is to solve the time-independent
Born-Oppenheimer Schr\"odinger equation with an optimal balance of accuracy and
efficiency for the problem of interest.
\begin{equation}
    \label{eq:electronic-schrodinger-equation}
    \hat{H}\Psi_k
    =
    E_k\Psi_k
\end{equation}
\begin{equation}
    \hat{H}
    =
    V_{\mathrm{nuc}}
    +
    \hat{H}_e
    =
    \sum_{a<b}^{\text{nuc.}}
    \frac{Z_aZ_b}{|\mathbf{R}_a-\mathbf{R}_b|}
    -
    \frac{1}{2}
    \sum_i^{\text{elec.}}
    \nabla_i^2
    -
    \sum_a^{\text{nuc.}}
    \sum_i^{\text{elec.}}
    \frac{Z_a}{|\mathbf{R}_a-\mathbf{r}_i|}
    +
    \sum_{i<j}^{\text{elec.}}
    \frac{1}{|\mathbf{r}_i-\mathbf{r}_j|}
\end{equation}
The most accurate solution possible for a given one-particle basis set of
spin-orbitals
\(
    \{\psi_p\}
\)
results from expanding the wavefunction as a linear combination of all
antisymmetrized products of these one-particle states, which are known as Slater
determinants
\(
    \{\Phi_\mu\}
\).
\begin{equation}
    \label{eq:full-ci-wavefunction-expansion}
    \Psi_k
    =
    \sum_\mu
    \Phi_\mu
    c_{\mu k}
\end{equation}
The expansion coefficients \((\mathbf{c})_k=c_{\mu k}\) are eigenvectors of the
matrix \((\mathbf{H})_{\mu\nu}=\langle\Phi_\mu|\hat{H}|\Phi_\nu\rangle\), which
is the matrix representation of the Hamiltonian in the determinant basis.
This is called the {\itshape full configuration-interaction}\ (FCI) solution.

Any one-electron basis spans the same ``function space'' as the AO basis set
itself, and the full $n$-electron basis $\{\Phi_\mu\}$ spans the same space of
$n$-electron functions regardless of how one forms spin orbitals from the AO
basis set.
As a result, one obtains the same FCI solution for any choice of spin-orbitals.
In general, however, FCI solutions are completely unfeasible for basis sets of
sufficient size to approach the complete basis set limit because the number of
determinants grows exponentially with the dimension of the one-particle basis.
The upshot is that we usually have to omit some Slater determinants in order to get an answer in a reasonable amount of time.
As soon as we truncate our determinant expansion
(\ref{eq:full-ci-wavefunction-expansion}), our choice of spin MOs makes a
significant difference in the quality of our results.
In particular, we need to choose our set of one-electron functions to minimize
the number of Slater determinants it takes to ``get close to'' the exact
wavefunction.
One approach is to determine the best single-determinant approximation to the
wavefunction by optimizing its energy expectation value with respect to
variations in the orbitals
\begin{align}
    \frac{\delta\langle\Phi|\hat{H}_e|\Phi\rangle}{\delta\psi_i}
    \overset{!}{=}
    0
    \qquad
    \Phi(1,\ldots,n)
    =
    \frac{1}{\sqrt{n!}}
    \left|
    \begin{matrix}
      \psi_1(1)&\psi_2(1)&\cdots&\psi_n(1)\\
      \psi_1(2)&\psi_2(2)&\cdots&\psi_n(2)\\
      \vdots    &\vdots    &\ddots&\vdots    \\
      \psi_1(n)&\psi_2(n)&\cdots&\psi_n(n)
    \end{matrix}
    \right|
\end{align}
where the arguments \(1,\ldots,n\) are short hand for the spin,
\(s_1,\ldots,s_n\), and spatial, \(\mathbf{r}_1,\ldots,\mathbf{r}_n\), degrees
of freedom of individual electrons.
This variational condition can be written as an eigenvalue equation of a
one-body ``Fock'' operator which represents an effective Hamiltonian for the
interaction of an electron with the mean field of the other electrons.
\begin{equation}
    \hat{f}
    \psi_i
    =
    \epsilon_i
    \psi_i
\end{equation}
Once we have solved for these variationally optimal ``Hartree-Fock'' orbitals,
the expectation value \(\langle\Phi|\hat{H}_e|\Phi\rangle\) is itself a good
first approximation to the electronic energy.
More importantly, however, when we use this new set of Hartree-Fock spin-orbitals, $\{\psi_p\}$, the FCI expansion tends to converge much more quickly to the true wavefunction.
Specifically, when we rewrite equation~\ref{eq:full-ci-wavefunction-expansion}
in terms of single $\{\Phi_i^a\}$, double $\{\Phi_{ij}^{ab}\}$, triple
$\{\Phi_{ijk}^{abc}\}$, etc.\ replacements of the orbitals in the Hartree-Fock
determinant $\Phi$ with the remaining orbitals in the basis
\begin{align}
  \Psi
=
  \Phi
+
  \sum_{\substack{a\\i}}
  \Phi_i^ac_a^i
+
  \sum_{\substack{a<b\\i<j}}
  \Phi_{ij}^{ab}c_{ab}^{ij}
+
  \sum_{\substack{a<b<c\\i<j<k}}
  \Phi_{ijk}^{abc}c_{abc}^{ijk}
  +\ldots
\end{align}
the coefficients tend to be very small, and are often virtually negligible for higher than quadruple replacements.

Our starting point in density cumulant theory (DCT) is to express the electronic
energy as a trace of the one- and antisymmetrized two-electron integrals (\(
h_p^q \) and \(\overline{g}_{pq}^{rs}\)) with the reduced one- and two-body
density matrices (\(\gamma^p_q\) and \(\gamma^{pq}_{rs}\)):
\begin{equation}
    \label{eq:energy-expression}
    E
    =
    h_p^q
    \gamma^p_q
    +
    \tfrac{1}{4}
    \overline{g}_{pq}^{rs}
    \gamma^{pq}_{rs}
\end{equation}
where summation over the repeated indices is implied.
In DCT, the two-body density matrix \(\gamma^{pq}_{rs}\) is expanded in terms of
its connected part, the two-body density cumulant ($\lambda^{pq}_{rs}$), and its
disconnected part, which is given by an antisymmetrized product of one-body
density matrices:\cite{Kutzelnigg:2006p171101}
\begin{equation}
    \label{eq:two-body-n-rep}
    \gamma^{pq}_{rs}
    =
    \langle\Psi|
    a^{pq}_{rs}
    |\Psi\rangle
    =
    \lambda^{pq}_{rs}
    +
    P_{(r/s)}
    \gamma^p_r
    \gamma^q_s
\end{equation}
where \(P_{(r/s)}v_{rs} = v_{rs} - v_{sr}\) denotes antisymmetrization and
\mbox{$a^{pq}_{rs}=a^{\dag}_{p}a^{\dag}_{q}a^{}_{s}a^{}_{r}$} is the two-body operator in second quantization.
The one-body density matrix \(\gamma^p_q\) is determined from its non-linear
relationship to the cumulant's partial trace:\cite{Sokolov:2013p024107}
\begin{equation}
    \label{eq:one-body-n-rep}
    \gamma^p_q
    =
    \gamma^p_r
    \gamma^r_q
    -
    \lambda^{pr}_{qr}
\end{equation}
This reduces \cref{eq:energy-expression} to a functional of the two-body
cumulant and the basis of spin-orbitals, thereby defining the DCT energy
functional.
The density cumulant is parametrized by choosing a specific Ansatz for the
wavefunction \(|\Psi\rangle\) such that\cite{Sokolov:2014p074111}
\begin{equation}
    \label{eq:cumulant-parametrization}
    \lambda^{pq}_{rs}
    =
    \langle\Psi|
    a^{pq}_{rs}
    |\Psi\rangle_\mathrm{c}
\end{equation}
where $\mathrm{c}$ indicates that only fully connected terms are included in the
parametrization.
\Cref{eq:cumulant-parametrization} can be considered as a set of
\(n\)-representability conditions that ensure that the resulting one- and
two-electron density matrices represent a physical \(n\)-electron wavefunction.
To compute the DCT energy, the functional \eqref{eq:energy-expression} is made
stationary with respect to all of its parameters.
Importantly, due to the connected nature of \cref{eq:cumulant-parametrization},
DCT is both size-consistent and size-extensive for any parametrization of
\(|\Psi\rangle\), and is exact in the limit of a complete
parametrization.\cite{Sokolov:2014p074111}

In this work, we consider the ODC-12
method,\cite{Sokolov:2013p024107,Sokolov:2013p204110} which parametrizes the
cumulant through a unitary treatment of single excitations and a linear
expansion of double excitations.
\begin{equation}
    \label{eq:odc12-wavefunction}
    |\Psi\rangle
    =
    e^{T_1-T_1^\dagger}
    (1 + T_2)
    |\Phi\rangle
\end{equation}
\begin{equation}
    T_1
    =
    \mathbf{t}_1\cdot\mathbf{a}_1
    =
    t_a^i
    a^a_i
\end{equation}
\begin{equation}
    T_2
    =
    \mathbf{t}_2\cdot\mathbf{a}_2
    =
    \tfrac{1}{4}
    t_{ab}^{ij}
    a^{ab}_{ij}
\end{equation}
The exponential singles operator \(e^{T_1-T_1^\dagger}\) has the effect of a
unitary transformation of the spin-orbital basis and is incorportated in our
implementation of the ODC-12 method by optimizing the
orbitals.\cite{Sokolov:2013p204110}
The \(\mathbf{t}_1\) and \(\mathbf{t}_2\) parameters are obtained from the
stationarity conditions
\begin{equation}
    \label{eq:stationarity_conditions}
    \dfrac{\partial E}{\partial \mathbf{t}_1^\dagger}
    \overset{!}{=}
    0 \ ,
    \qquad
    \dfrac{\partial E}{\partial \mathbf{t}_2^\dagger}
    \overset{!}{=}
    0
\end{equation}
and are used to compute the ODC-12 energy.
Explicit equations for the stationarity conditions are given in Refs.\@
\citenum{Sokolov:2013p024107} and \citenum{Sokolov:2013p204110}.
Although in ODC-12 the wavefunction parametrization is linear with respect
to double excitations (\cref{eq:odc12-wavefunction}), the ODC-12 energy
stationarity conditions are non-linear in $\mathbf{t}_2$ due to the non-linear
relationship between the one-particle density matrix and the density cumulant
(\cref{eq:one-body-n-rep}).\cite{Sokolov:2013p024107} Neglecting the non-linear
$\mathbf{t}_2$ terms in \cref{eq:stationarity_conditions} results in the
equations that define the linearized orbital-optimized  coupled cluster doubles
method (OLCCD).
This method is equivalent to the orbital-optimized coupled electron pair
approximation zero (OCEPA$_0$).\cite{Bozkaya:2013p054104}
