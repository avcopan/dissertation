\chapter[%
    Algorithms for Linear-Response Density Cumulant Theory
]{%
    Algorithms for Linear-Response Density Cumulant Theory
}
\label{ch:davidson}

\begin{enumerate}
    \item
        \cref{ch:response} presents a new model for electronic excited states,
        which is based on a linear-response formulation of density cumulant
        theory.
    \item
        In this model, excitation energies and transition properties are
        computed from the eigenpairs of the energy functional's parameter
        Hessian, with respect to a metric that arises from the time-dependence
        of the parameter responses.
    \item
        Since number of parameters in the ODC-12 variant of density cumulant
        theory scales as
        \(
            \mathcal{O}(o^2v^2)
        \)
        with the number of occupied (\(o\)) and virtual (\(v\)) orbitals, the
        memory requirement for the Hessian matrix scales with the fourth power
        of \(o\) and \(v\), and the number of floating point operations needed
        to diagonalize it scales with the sixth power of these dimensions.
    \item
        Such a brute-force approach will rapidly overwhelm available computing
        resources even for relatively small molecules.
    \item
        For the common scenario where we only care about states within a narrow
        energy range, the cost of diagonalization can be drastically reduced
        through the use of ``direct algorithms'' which allow us to compute
        subsets of eigenvectors and eigenvalues without explicitly constructing
        the full matrix of the linear transformation in computer memory.
    \item
        This chapter will explore the use of the Davidson
        algorithm\cite{Liu:1978p49,Davidson:1975p87} to efficiently compute the
        lowest-energy states of the LR-ODC-12 model.
    \item
        After describing the generic Davidson algorithm in
        \cref{sec:davidson:davidson} and analyzing the structure of the
        LR-ODC-12 equations in \cref{sec:davidson:eig}, I will present several
        strategies for solving the LR-ODC-12 model in
        \cref{sec:davidson:strategies},
        along with a benchmark study of their convergence characteristics for
        several small molecules:
        \ce{H2O}, \ce{N2}, \ce{HCN}, \ce{H2CO}, \ce{C2H4}.
    \item
        Finally, in \cref{sec:davidson:disk} I will discuss how to reduce memory
        consumption in the Davidson algorithm by partitioning the expansion
        space into blocks and storing large arrays on disk.
\end{enumerate}

\section{The Davidson Algorithm}
\label{sec:davidson:davidson}

\begin{algorithm}
    \caption{%
        Canonical multiroot Davidson algorithm for a generic eigenvalue problem,
        $\mathbf{L}\mathbf{v}_j=\lambda_j\mathbf{G}\mathbf{v}_j$, with periodic
        subspace collapse.
        Requires linear transformation functions and diagonal approximations
        (indicated by tildes) for \(\mathbf{L}\) and \(\mathbf{G}\)
        and solves for the lowest \(k\) eigenvalues and eigenvectors.
    }
    \label{algo:davidson}
    \begin{algorithmic}[1]
        \Procedure{Davidson}{%
            $
            \mathbf{L}(\cdot),
            \mathbf{G}(\cdot),
            \tilde{\mathbf{L}},
            \tilde{\mathbf{G}},
            \mathbf{U}^{(0)},
            k,
            d_\mathrm{max},
            i_\mathrm{max},
            r_\mathrm{tol}
            $%
        }
        \State
        Initialize the expansion space with a set of guess vectors,
        \(\mathbf{U}\leftarrow\mathbf{U}^{(0)}\).
        \For{$1\leq i\leq i_\mathrm{max}$}{}
            \State
            Construct subspace representation and solve the lowest \(k\)
            eigenvalues.
            \[
                \mathbf{L}^\mathrm{sub}
                =
                \mathbf{U}^\dagger
                \mathbf{L}(\mathbf{U})
            \]
            \[
                \mathbf{G}^\mathrm{sub}
                =
                \mathbf{U}^\dagger
                \mathbf{G}(\mathbf{U})
            \]
            \[
                \mathbf{L}^\mathrm{sub}
                \mathbf{v}_j^\mathrm{sub}
                =
                \lambda_j
                \mathbf{G}^\mathrm{sub}
                \mathbf{v}_j^\mathrm{sub}
            \]
            \State
            Calculate the eigenvector residuals over the full space.
            \[
                \mathbf{r}_j
                =
                (
                    \mathbf{L}(\mathbf{U})
                    -
                    \lambda_j
                    \mathbf{G}(\mathbf{U})
                )
                \mathbf{v}_j^\mathrm{sub}
            \]
            \If{$\max(\mathbf{r}_j) < r_\mathrm{tol}$ for all $j$}
                \State
                Set
                \(\mathbf{v}_j\leftarrow\mathbf{U}\mathbf{v}_j^\mathrm{sub}\)
                and quit the loop.  The eigenvectors are converged.
            \EndIf
            \State
            Determine new direction vectors by preconditioning the residual.
            \[
                \mathbf{d}_j^{(i)}
                =
                -
                (
                    \tilde{\mathbf{L}}
                    -
                    \lambda_j
                    \tilde{\mathbf{G}}
                )^{-1}
                \mathbf{r}_j
            \]
            \State
            Project out the span of \(\mathbf{U}\) and orthogonalize via
            SVD compression.
            \[
                \widehat{\mathbf{U}}^{(i)}
                =
                (\mathbf{1} - \mathbf{U}^\dagger \mathbf{U})
                \mathbf{D}^{(i)}
            \]
            \[
                \widehat{\mathbf{U}}^{(i)}
                \approx
                \mathbf{U}^{(i)}
                \mathbf{\Sigma}^{(i)}
                \mathbf{W}^{(i)\dagger}
            \]
            \If{%
                $
                \mathrm{rank}(\mathbf{U})
                +
                \mathrm{rank}(\mathbf{U}^{(i)})
                <
                d_\mathrm{max}
                $%
            }
                \State
                Extend the expansion space,
                \(
                    \mathbf{U}
                    \leftarrow
                    (\mathbf{U}\ \mathbf{U}^{(i)})
                \)
            \Else
                \State
                Collapse the expansion space,
                \(
                    \mathbf{U}
                    \leftarrow
                    (
                        \mathbf{U}
                        \mathbf{v}_1^\mathrm{sub}\ 
                        \cdots\ 
                        \mathbf{U}
                        \mathbf{v}_k^\mathrm{sub}
                    )
                \).
            \EndIf
        \EndFor
        \State
        {\bfseries return}
        \(
            \lambda_j,
            \mathbf{v}_j
        \)
        \EndProcedure
    \end{algorithmic}
\end{algorithm}

P1
\begin{enumerate}
    \item
        Direct algorithms like the Davidson algorithm represent a linear
        transformation as a function
        \(
            \mathbf{L}(\cdot)
        \)
        mapping vectors in its domain to vectors in its codomain, rather than as
        an array of coefficients
        \(
            [L_{ij}]
            =
            [\mathbf{e}_i \cdot \mathbf{L}(\mathbf{e}_j)]
        \)
        over a complete basis.
    \item
        That is, the result of the transformation is determined {\itshape
        directly}, without explicitly forming its matrix representation in
        computer memory.
    \item
        The Davidson algorithm applies this technique in the context of a matrix
        diagonalization, by progressively growing a basis
        \(
            \{\mathbf{u}_1, \ldots,\mathbf{u}_d\}
        \)
        to span the lowest or highest eigenvectors of a matrix to some threshold
        of accuracy.
    \item
        For a transformation on \(\mathbb{R}^n\), this allows us to reduce our
        computational effort from \(\mathcal{O}(n^3)\) to at most
        \(\mathcal{O}(n^2 d)\), and often much less if \(\mathbf{L}\) is
        constructed from lower-dimensional arrays.
    \item
        Our memory storage requirements are reduced from \(\mathcal{O}(n^2)\) to
        \(\mathcal{O}(nd)\), so as long as the number of roots is much smaller
        than the dimension of the space we can easily satisfy \(d \ll n\) and
        gain considerable savings.
\end{enumerate}

\noindent
P2
\begin{enumerate}
    \item
        The procedure for the Davidson algorithm is presented in
        \cref{algo:davidson}, which solves for the eigenvalues and right
        eigenvectors of a generalized eigenvalue problem, which may or may not
        be symmetric.
    \item
        The general strategy of the algorithm is quite simple.
    \item
        We first expand our transformations in the reduced expansion space as,
        for example,
        \(
            [L_{ij}^\mathrm{sub}]
            =
            [\mathbf{u}_i\cdot \mathbf{L}(\mathbf{u}_j)]
        \).
    \item
        The eigenvalue equation is then solved in this subspace
        \begin{equation}
            \mathbf{L}^\mathrm{sub}
            \mathbf{v}_j^\mathrm{sub}
            =
            \lambda_j^\mathrm{trial}
            \mathbf{G}^\mathrm{sub}
            \mathbf{v}_j^\mathrm{sub}
        \end{equation}
    \item
        In the full space our trial solution is given by
        \(
            \mathbf{v}_j^\mathrm{trial}
            =
            \mathbf{U}\mathbf{v}_j^\mathrm{sub}
        \),
        and the correction vector
        \(
            \mathbf{d}_j
            =
            \mathbf{v}_j^\mathrm{trial}
            -
            \mathbf{v}_j
        \)
        to the exact solution can be approximated as
        \begin{equation}
            \mathbf{d}_j
            \approx
            -
            (
                \tilde{\mathbf{L}}
                -
                \lambda_j^\mathrm{trial}
                \tilde{\mathbf{G}}
            )^{-1}
            \mathbf{r}_j
            \qquad
            \mathbf{r}_j
            \equiv
            (\mathbf{L} - \lambda_j^\mathrm{trial}\mathbf{G})
            \mathbf{v}_j^\mathrm{trial}
        \end{equation}
        where \(\mathbf{r}_j\) is the residual vector and the preconditioner
        \(
            (
                \tilde{\mathbf{L}} - \lambda^\mathrm{trial}\tilde{\mathbf{G}}
            )^{-1}
        \)
        is constructed from diagonal approximations of \(\mathbf{L}\) and
        \(\mathbf{G}\).
    \item
        This can be motivated as an approximate solution to the following
        identity.
        \begin{equation}
            \mathbf{0}
            =
            (\mathbf{L} - \lambda_j\mathbf{G})
            \mathbf{v}_j
            =
            (\mathbf{L} - \lambda_j\mathbf{G})
            (
                \mathbf{v}_j^\mathrm{trial}
                +
                \mathbf{d}_j
            )
        \end{equation}
    \item
        Similar algorithms can be formulated for other large matrix equations,
        such as the linear equation \(\mathbf{L}\mathbf{x}=\mathbf{b}\) which
        leads to a variant of the conjugate gradient method.
\end{enumerate}

\noindent
P3
\begin{enumerate}
    \item
        A key assumption is that the matrices are diagonally dominant.
    \item
        Efficient implementations of the Davidson algorithm will compute images
        \(
            \mathbf{L}(\mathbf{u}_i)
        \)
        only for the new expansion vectors and store these for future
        iterations.
    \item
        When the expansion space becomes large, we can replace the expansion
        vectors with the current trial vectors to keep the memory requirements
        manageable.
    \item
        Frequent collapses every second or third iteration can be used to keep
        I/O requirements to a minimum, at the cost of slower
        convergence.\cite{Leininger:2001p1574}
\end{enumerate}

\section{The LR-ODC-12 Eigenvalue Equation}
\label{sec:davidson:eig}

The LR-ODC-12 eigenvalue equation has a two-by-two block structure which
describes the independent variation of the state parameters and their complex
conjugates.
\begin{equation}
    \label{eq:linear-response-eigenvalue-equation}
    \mathbf{E}(\mathbf{z}_k)
    =
    \omega_k
    \mathbf{M}(\mathbf{z}_k)
    ,
    \quad
    \mathbf{E}
    =
    \begin{pmatrix}
        \mathbf{A} & \mathbf{B} \\
        \mathbf{B}^* & \mathbf{A}^*
    \end{pmatrix}
    ,
    \quad
    \mathbf{M}
    =
    \begin{pmatrix}
        \mathbf{S} & \mathbf{0} \\
        \mathbf{0} & -\mathbf{S}^*
    \end{pmatrix}
    ,
    \quad
    \mathbf{z}_k
    =
    \begin{pmatrix}
        \mathbf{x}_k \\
        \mathbf{y}_k
    \end{pmatrix}
\end{equation}
This block symmetry leads to a paired system of eigenvalues,
\(
    \omega_{\pm k}
    =
    \pm\omega_k
\).
The submatrices in \cref{eq:linear-response-eigenvalue-equation} are further
blocked according to whether they describe variations with respect to one-body
(\(\mathbf{t}_1\)) or two-body (\(\mathbf{t}_2\)) parameters.
\begin{equation}
    \label{eq:conjugate-blocks}
    \mathbf{A}
    =
    \begin{pmatrix}
        \mathbf{A}_{11} & \mathbf{A}_{12} \\
        \mathbf{A}_{21} & \mathbf{A}_{22} \\
    \end{pmatrix}
    \quad
    \mathbf{B}
    =
    \begin{pmatrix}
        \mathbf{B}_{11} & \mathbf{B}_{12} \\
        \mathbf{B}_{21} & \mathbf{B}_{22} \\
    \end{pmatrix}
    \quad
    \mathbf{S}
    =
    \begin{pmatrix}
        \mathbf{S}_{11} & \mathbf{0} \\
        \mathbf{0} & \mathbf{1}_2 \\
    \end{pmatrix}
    \quad
    \mathbf{x}_k
    =
    \begin{pmatrix}
        \mathbf{x}_{k,1} \\
        \mathbf{x}_{k,2}
    \end{pmatrix}
\end{equation}

The bottleneck is in the diagonal two-body Hessian.
The image of an arbitrary two-body vector
\(
    \mathbf{u}_{\mu,2}
    =
    [u_{\mu,ab}^{ij}]
\)
under this transformation is given by
\begin{equation}
    \label{eq:two-body-hessian-function-in-text}
    \begin{array}{r@{\,}l}
        (\mathbf{A}_{22}(\mathbf{u}_{\mu,2}))_{ijab}
        =
        &
        -
        P_{(a/b)}
        \mathcal{F}_a^c
        u_{\mu,cb}^{ij}
        -
        P^{(i/j)}
        \mathcal{F}_k^i
        u_{\mu,ab}^{kj}
        +
        \tfrac{1}{2}
        \overline{g}_{ab}^{cd}
        u_{\mu,cd}^{ij}
        +
        \tfrac{1}{2}
        \overline{g}_{kl}^{ij}
        u_{\mu,ab}^{kl}
        \\[5pt]
        &
        -
        P_{(a/b)}^{(i/j)}
        \overline{g}_{la}^{jc}
        u_{\mu,cb}^{il}
        +
        \tfrac{1}{2}
        P_{(a/b)}
        \mathcal{G}_{af}^{ec}
        t_{eb}^{ij}
        t_{kl}^{fd*}
        u_{\mu,cd}^{kl}
        +
        \tfrac{1}{2}
        P_{(a/b)}
        \mathcal{G}_{ka}^{me}
        t_{eb}^{ij}
        t_{ml}^{cd*}
        u_{\mu,cd}^{kl}
        \\[5pt]
        &
        +
        \tfrac{1}{2}
        P^{(i/j)}
        \mathcal{G}_{me}^{ic}
        t_{ab}^{mj}
        t_{kl}^{ed*}
        u_{\mu,cd}^{kl}
        +
        \tfrac{1}{2}
        P^{(i/j)}
        \mathcal{G}_{mk}^{in}
        t_{ab}^{mj}
        t_{nl}^{cd*}
        u_{\mu,cd}^{kl}
    \end{array}
\end{equation}
where the \(i,j,k,l,\ldots\) run over occupied spin-orbitals, \(a,b,c,d,\ldots\)
run over virtual (un-occupied) spin-orbitals, and summation over repeated
indices is implied.
The bottleneck is the contraction of the \(\mathrm{v}^4\) integrals with the
expansion vector, \(\mathrm{g}_{ab}^{cd}u_{\mu,cd}^{ij}\), which scales as
\(\mathcal{O}(d_\mu\mathrm{o}^2\mathrm{v}^4)\) in the number of floating point
operations.
This term is the rate limiting step in EOM-CCSD as well.
The full set of linear transformation formulas for the LR-ODC-12 Hessian and
metric blocks is given in the appendix
(\cref{sec:linear-transformation-formulas}).


\section{Strategies for Solving the LR-ODC-12 Model}
\label{sec:davidson:strategies}

\section{Blocking and Disk Storage in the Davidson Algorithm}
\label{sec:davidson:disk}


\section{The LR-ODC-12 Eigenvalue Equation}


\section{Algorithms}
\label{sec:algorithms}



\paragraph{Strategy 1}
One approach is to invert the eigenvalue equation, solving for the {\itshape
largest} positive inverse eigenvalues, which is easily done by selecting the
highest rather than the lowest roots at each step in \cref{algo:davidson}.
\begin{equation}
    \mathbf{M}(\mathbf{z}_k)
    =
    \omega_k^{-1}
    \mathbf{E}(\mathbf{z}_k)
\end{equation}
This has the added advantage that the energy Hessian \(\mathbf{E}\) is generally
positive definite, which allows us to treat the subspace diagonalizations as
standard Hermitian eigenvalue equation with real eigenvalues and orthonormal
eigenvectors.
A simple, effective choice of guess vectors in this approach is to determine the
lowest eigenvectors of the following diagonal approximation to
\cref{eq:two-body-hessian-function-in-text}
\begin{equation}
    (\tilde{\mathbf{A}}_{22}(\mathbf{u}_{\mu}))_{ijab}
    \equiv
    -
    (
        \mathcal{F}_i^i
        +
        \mathcal{F}_j^j
        +
        \mathcal{F}_a^a
        +
        \mathcal{F}_b^b
    )
    u_{\mu,ab}^{ij}
\end{equation}
which are simply the standard unit vectors associated with the smallest diagonal
entries.
The virtual coefficients \(\mathcal{F}_a^a,\mathcal{F}_b^b\) in this equation
are generally negative, so this has the form of an orbital energy difference
like we see in Hartree-Fock theory.
For larger systems, we manage the memory requirements of the method by keeping
successive groups of expansion vectors and images,
\(
    \mathbf{U}^{(i)},
    \mathbf{E}(\mathbf{U}^{(i)}),
    \mathbf{M}(\mathbf{U}^{(i)})
\),
on disk and reading them in as needed.
These can be further subdivided into even smaller blocks if needed.


\paragraph{Strategy 2}
Assuming real matrices, we can add and subtract the block rows of
\cref{eq:linear-response-eigenvalue-equation} to arrive at the following pair of
equations.
\begin{equation}
    \label{eq:a-plus-b}
    (\mathbf{A} + \mathbf{B})
    (\mathbf{x}_k + \mathbf{y}_k)
    =
    \omega_k
    \mathbf{S}
    (\mathbf{x}_k - \mathbf{y}_k)
\end{equation}
\begin{equation}
    \label{eq:a-minus-b}
    (\mathbf{A} - \mathbf{B})
    (\mathbf{x}_k - \mathbf{y}_k)
    =
    \omega_k
    \mathbf{S}
    (\mathbf{x}_k + \mathbf{y}_k)
\end{equation}
Multiplying both equations by \(\mathbf{S}^{-1}\), we can write them as follows
\begin{equation}
    \mathbf{H}^+(\mathbf{c}_k^+)
    =
    \omega_k
    \mathbf{c}_k^-
\end{equation}
\begin{equation}
    \mathbf{H}^-(\mathbf{c}_k^-)
    =
    \omega_k
    \mathbf{c}_k^+
\end{equation}
where we have defined the following intermediates
\begin{equation}
    \mathbf{H}^{\pm}
    \equiv
    \mathbf{S}^{-1}
    (\mathbf{A} \pm \mathbf{B})
\end{equation}
\begin{equation}
    \mathbf{c}_k^{\pm}
    \equiv
    \mathbf{x}_k \pm \mathbf{y}_k
\end{equation}
This provides a reduced eigenvalue equation for the squared excitations energies
\begin{equation}
    \overline{\mathbf{H}}(\mathbf{c}_k^+)
    =
    \omega_k^2
    \mathbf{c}_k^+
\end{equation}
\begin{equation}
    \overline{\mathbf{H}}
    =
    \mathbf{H}^-
    \mathbf{H}^+
\end{equation}
The blocks of the generalized eigenvector can then be recovered as follows
\begin{equation}
    \mathbf{x}_k
    =
    \tfrac{1}{2}
    (
        \mathbf{c}_k^+
        +
        \omega_k^{-1}
        \mathbf{H}^+(\mathbf{c}_k^+)
    )
\end{equation}
\begin{equation}
    \mathbf{y}_k
    =
    \tfrac{1}{2}
    (
        \mathbf{c}_k^+
        -
        \omega_k^{-1}
        \mathbf{H}^+(\mathbf{c}_k^+)
    )
\end{equation}

Which also shows that any formula can be reduced to a formula for the reduced
eigenvalues and eigenvectors.
For example, assuming real integral arrays (\(\mathbf{p}^*=\mathbf{p}\))
\begin{equation}
    \mathbf{z}_k^\dagger
    \mathbf{v}'
    =
    \mathbf{c}_k^{+\dagger}
    \mathbf{p}
\end{equation}
and
\begin{equation}
    \mathbf{z}_k^\dagger
    \mathbf{M}
    \mathbf{z}_k
    =
    \mathbf{c}_k^{+\dagger}
    \mathbf{S}
    \mathbf{c}_k^{-}
    =
    \omega_k^{-1}
    \mathbf{c}_k^{+\dagger}
    \mathbf{S}
    \mathbf{H}^+
    \mathbf{c}_k^+
\end{equation}
\begin{equation}
    \lim_{\omega\rightarrow \omega_k}
    (\omega-\omega_k)
    \langle\!\langle \hat{V}; \hat{V} \rangle\!\rangle_\omega
    =
    \frac{%
        \omega_k
        |\mathbf{c}_k^{+\dagger} \mathbf{p}|^2
    }{% 
        \mathbf{c}_k^{+\dagger} 
        (\mathbf{A} + \mathbf{B})
        \mathbf{c}_k^+
    }
\end{equation}


From \cref{eq:conjugate-blocks} we can see that computing the inverse of
\(\mathbf{S}\) requires us to invert the orbital block of the metric.
This matrix is given by
\begin{equation}
    (\mathbf{S}_{11})_{ia,jb}
    =
    \gamma^i_j
    \delta_a^b
    -
    \delta_j^i
    \gamma^b_a
\end{equation}
where
\(
    \gamma_j^i
\)
and
\(
    \gamma_a^b
\)
are occupied and virtual blocks of the one-body density matrix.
By expanding these matrices in the natural spin-orbital basis (indicated by
prime indices) where the one-body density matrix is diagonal
\begin{equation}
    \gamma_j^i
    =
    (\mathbf{Y})_j^{j'}
    (\mathbf{Y}^\dagger)_{i'}^i
    \delta_{j'}^{i'}
    \gamma_{j'}
    \qquad
    \gamma_a^b
    =
    (\mathbf{Y})_a^{a'}
    (\mathbf{Y}^\dagger)_{b'}^b
    \delta_{a'}^{b'}
    \gamma_{a'}
\end{equation}
we can derive the following simple and inexpensive formula for the inverse of
the orbital metric.
\begin{equation}
    (\mathbf{S}_{11}^{-1})_{ia,jb}
    =
    \frac{%
        (\mathbf{Y}^\dagger)_{j'}^i
        (\mathbf{Y})_a^{b'}
    }{%
        \gamma_{j'}-\gamma_{b'}
    }
    (\mathbf{Y}^\dagger)_{b'}^b
    (\mathbf{Y})_j^{j'}
\end{equation}
The appropriate diagonal approximation to \(\mathbf{H}\) can simply be computed
as the square of \(\tilde{\mathbf{A}}\), since in low order we have
\(\mathbf{B}\approx\mathbf{0}\) and \(\mathbf{S}\approx\mathbf{1}\)


\begin{equation}
    \mathbf{H}^\dagger
    (\mathbf{S}(\mathbf{x}_k - \mathbf{y}_k))
    =
    \omega_k^2
    (\mathbf{S}(\mathbf{x}_k - \mathbf{y}_k))
\end{equation}

\afterpage{%
    \clearpage
    \centering
    \begin{landscape}
        \vspace*{\fill}
        \captionof{table}{%
            A comparison of the convergence properties of three different
            solution strategies for LR-ODC-12 for five molecules using the
            def2-SV(P) basis set.
            In each case we converge 10 eigenvectors to \(10^{-5}~\au\) in the
            residual, starting from an initial expansion space of 100 guess
            vectors and collapsing the subspace very 200 vectors.
            The second and third columns show the number of singles and doubles
            parameters for each system, which determine the dimensions of the
            matrix equation, and in each case we give the number of iterations,
            the run-time, and the number of low-lying roots obtained.
            The first row for each molecule shows the results for the FD
            preconditioner and the second row shows the results for the
            PED preconditioner.
            All computations were run on an
            Intel\textsuperscript{\textregistered} Core\texttrademark\ i7-5600U
            processor using four threads.
        }
        \vspace{15pt}
        \begin{tabular}{ccccccccccccc}
            \hline
            \hline
            &
            &
            &
            \multicolumn{3}{c}{strategy 1}
            &
            \multicolumn{3}{c}{strategy 2}
            &
            \multicolumn{3}{c}{strategy 3}
            \\
            &
            \(n_1\)
            &
            \(n_2\)
            &
            iter
            &
            time (s)
            &
            roots
            &
            iter
            &
            time (s)
            &
            roots
            &
            iter
            &
            time (s)
            &
            roots
            \\
            \hline
            \ce{H2O}
            & 260 & 14,625
            & 11 & 23 & 10/10 & 16 & 29 & 10/10 & 16 & 30 & 10/10
            \\
            &&
            & 11 & 23 & 10/10 & 24 & 38 & 10/10 & 24 & 40 & 10/10
            \\
            \ce{N2}
            & 588 & 78,351
            & 14 & 186 & 10/10 & 17 & 213 & 9/10 & 17 & 216 & 9/10
            \\
            &&
            & 10 & 156 & 10/10 & 15 & 179 & 9/10 & 15 & 202 & 9/10
            \\
            \ce{HCN}
            & 644 & 94,185
            & 15 & 251 & 9/10 & 21 & 281 & 9/10 & 21 & 331 & 9/10
            \\
            &&
            & 10 & 205 & 9/10 & 18 & 252 & 9/10 & 19 & 308 & 9/10
            \\
            \ce{H2CO}
            & 768 & 135,360
            & 20 & 478 & 7/10 & 28 & 638 & 6/10 & 27 & 650 & 6/10
            \\
            &&
            & 11 & 362 & 7/10 & 23 & 442 & 7/10 & 25 & 614 & 7/10
            \\
            \ce{C2H4}
            & 896 & 184,800
            & 17 & 640 & 9/10 & 25 & 874 & 9/10 & 25 & 891 & 9/10
            \\
            &&
            & 10 & 500 & 9/10 & 19 & 574 & 9/10 & 20 & 722 & 9/10
            \\
            \hline
            \hline
        \end{tabular}
        \vspace*{\fill}
    \end{landscape}
}


\begin{subappendices}
\section{LR-ODC-12 Linear Transformation Formulas}
\label{sec:linear-transformation-formulas}

\begin{equation}
    \begin{array}{r@{\,}l}
        (\mathbf{A}_{11}(\mathbf{u}_{\mu,1}))_{ia}
        =
        &
        h_j^i
        \gamma_a^b
        u_{\mu,b}^j
        +
        h_a^b
        \gamma_j^i
        u_{\mu,b}^j
        -
        (\bar{\mathbf{F}})_j^i
        u_{\mu,a}^j
        -
        (\bar{\mathbf{F}})_a^b
        u_{\mu,b}^i
        +
        \overline{g}_{nj}^{mi}
        \gamma_{ma}^{nb}
        u_{\mu,b}^j
        \\[5pt]
        &
        +
        \overline{g}_{ma}^{nb}
        \gamma_{nj}^{mi}
        u_{\mu,b}^j
        +
        \overline{g}_{jf}^{ie}
        \gamma_{ae}^{bf}
        u_{\mu,b}^j
        +
        \overline{g}_{ae}^{bf}
        \gamma_{jf}^{ie}
        u_{\mu,b}^j
        +
        \overline{g}_{me}^{ib}
        \gamma_{ja}^{me}
        u_{\mu,b}^j
        \\[5pt]
        &
        +
        \overline{g}_{ja}^{me}
        \gamma_{me}^{ib}
        u_{\mu,b}^j
    \end{array}
\end{equation}

\begin{equation}
    \begin{array}{r@{\,}l}
        (\mathbf{B}_{11}(\mathbf{u}_{\mu,1}))_{ia}
        =
        &
        \overline{g}_{be}^{im}
        \gamma_{ma}^{je}
        u_{\mu,j}^b
        +
        \overline{g}_{ma}^{je}
        \gamma_{be}^{im}
        u_{\mu,j}^b
        +
        \overline{g}_{mb}^{ie}
        \gamma_{ae}^{jm}
        u_{\mu,j}^b
        +
        \overline{g}_{ae}^{jm}
        \gamma_{mb}^{ie}
        u_{\mu,j}^b
        \\[5pt]
        &
        +
        \tfrac{1}{2}
        \overline{g}_{mn}^{ij}
        \gamma_{ab}^{mn}
        u_{\mu,j}^b
        +
        \tfrac{1}{2}
        \overline{g}_{ab}^{mn}
        \gamma_{mn}^{ij}
        u_{\mu,j}^b
        +
        \tfrac{1}{2}
        \overline{g}_{ef}^{ij}
        \gamma_{ab}^{ef}
        u_{\mu,j}^b
        +
        \tfrac{1}{2}
        \overline{g}_{ab}^{ef}
        \gamma_{ef}^{ij}
        u_{\mu,j}^b
    \end{array}
\end{equation}

\begin{equation}
    \begin{array}{r@{\,}l}
        (\mathbf{A}_{22}(\mathbf{u}_{\mu,2}))_{ijab}
        =
        &
        -
        P_{(a/b)}
        \mathcal{F}_a^c
        u_{\mu,cb}^{ij}
        -
        P^{(i/j)}
        \mathcal{F}_k^i
        u_{\mu,ab}^{kj}
        +
        \tfrac{1}{2}
        \overline{g}_{ab}^{cd}
        u_{\mu,cd}^{ij}
        +
        \tfrac{1}{2}
        \overline{g}_{kl}^{ij}
        u_{\mu,ab}^{kl}
        \\[5pt]
        &
        -
        P_{(a/b)}^{(i/j)}
        \overline{g}_{la}^{jc}
        u_{\mu,cb}^{il}
        +
        \tfrac{1}{2}
        P_{(a/b)}
        \mathcal{G}_{af}^{ec}
        t_{eb}^{ij}
        t_{kl}^{fd*}
        u_{\mu,cd}^{kl}
        +
        \tfrac{1}{2}
        P_{(a/b)}
        \mathcal{G}_{ka}^{me}
        t_{eb}^{ij}
        t_{ml}^{cd*}
        u_{\mu,cd}^{kl}
        \\[5pt]
        &
        +
        \tfrac{1}{2}
        P^{(i/j)}
        \mathcal{G}_{me}^{ic}
        t_{ab}^{mj}
        t_{kl}^{ed*}
        u_{\mu,cd}^{kl}
        +
        \tfrac{1}{2}
        P^{(i/j)}
        \mathcal{G}_{mk}^{in}
        t_{ab}^{mj}
        t_{nl}^{cd*}
        u_{\mu,cd}^{kl}
    \end{array}
\end{equation}

\begin{equation}
    \begin{array}{r@{\,}l}
        (\mathbf{B}_{22}(\mathbf{u}_{\mu,2}))_{ijab}
        =
        &
        \tfrac{1}{2}
        P_{(a/b)}
        \mathcal{G}_{ac}^{ef}
        t_{eb}^{ij}
        t_{fd}^{kl}
        u_{\mu,kl}^{cd}
        +
        \tfrac{1}{2}
        P_{(a/b)}
        \mathcal{G}_{na}^{ke}
        t_{eb}^{ij}
        t_{cd}^{nl}
        u_{\mu,kl}^{cd}
        \\[5pt]
        &
        +
        \tfrac{1}{2}
        P^{(i/j)}
        \mathcal{G}_{mc}^{if}
        t_{ab}^{mj}
        t_{fd}^{kl}
        u_{\mu,kl}^{cd}
        +
        \tfrac{1}{2}
        P^{(i/j)}
        \mathcal{G}_{mn}^{ik}
        t_{ab}^{mj}
        t_{cd}^{nl}
        u_{\mu,kl}^{cd}
    \end{array}
\end{equation}

\begin{equation}
    \begin{array}{r@{\,}l}
        (\mathbf{A}_{12}(\mathbf{u}_{\mu,2}))_{ia}
        =
        &
        -
        \tfrac{1}{2}
        \overline{g}_{la}^{cd}
        u_{\mu,cd}^{il}
        -
        \tfrac{1}{2}
        \overline{g}_{kl}^{id}
        u_{\mu,ad}^{kl}
        -
        \tfrac{1}{2}
        (\mathcal{I}_a^i)_k^m
        t_{ml}^{cd*}
        u_{\mu,cd}^{kl}
        -
        \tfrac{1}{2}
        (\mathcal{I}_a^i)_e^c
        t_{kl}^{ed*}
        u_{\mu,cd}^{kl}
        \\[5pt]
        &
        -
        \overline{g}_{ae}^{mc}
        t_{ml}^{ed*}
        u_{\mu,cd}^{il}
        -
        \overline{g}_{ke}^{im}
        t_{ml}^{ed*}
        u_{\mu,ad}^{kl}
        -
        \tfrac{1}{4}
        \overline{g}_{la}^{mn}
        t_{mn}^{cd*}
        u_{\mu,cd}^{il}
        \\[5pt]
        &
        -
        \tfrac{1}{4}
        \overline{g}_{ef}^{id}
        t_{kl}^{ef*}
        u_{\mu,ad}^{kl}
    \end{array}
\end{equation}

\begin{equation}
    \begin{array}{r@{\,}l}
        (\mathbf{B}_{12}(\mathbf{u}_{\mu,2}))_{ia}
        =
        &
        -
        \tfrac{1}{2}
        (\mathcal{I}_a^i)_m^k
        t_{cd}^{ml}
        u_{\mu,kl}^{cd}
        -
        \tfrac{1}{2}
        (\mathcal{I}_a^i)_c^e
        t_{ed}^{kl}
        u_{\mu,kl}^{cd}
        -
        \overline{g}_{ad}^{le}
        t_{ce}^{ki}
        u_{\mu,kl}^{cd}
        -
        \overline{g}_{md}^{il}
        t_{ca}^{km}
        u_{\mu,kl}^{cd}
        \\[5pt]
        &
        -
        \tfrac{1}{4}
        \overline{g}_{ma}^{kl}
        t_{cd}^{im}
        u_{\mu,kl}^{cd}
        -
        \tfrac{1}{4}
        \overline{g}_{cd}^{ie}
        t_{ae}^{kl}
        u_{\mu,kl}^{cd}
    \end{array}
\end{equation}

\begin{equation}
    \begin{array}{r@{\,}l}
        (\mathbf{A}_{21}(\mathbf{u}_{\mu,1}))_{ijab}
        =
        &
        -
        P^{(i/j)}
        \overline{g}_{ab}^{jc}
        u_{\mu,c}^i
        -
        P_{(a/b)}
        \overline{g}_{kb}^{ij}
        u_{\mu,a}^k
        -
        P^{(i/j)}
        (\mathcal{I}_k^c)_m^i
        t_{ab}^{mj}
        u_{\mu,c}^k
        \\[5pt]
        &
        -
        P_{(a/b)}
        (\mathcal{I}_k^c)_a^e
        t_{eb}^{ij}
        u_{\mu,c}^k
        -
        P_{(a/b)}^{(i/j)}
        \overline{g}_{ma}^{ce}
        t_{eb}^{mj}
        u_{\mu,c}^i
        -
        P_{(a/b)}^{(i/j)}
        \overline{g}_{km}^{ie}
        t_{eb}^{mj}
        u_{\mu,a}^k
        \\[5pt]
        &
        -
        \tfrac{1}{2}
        P^{(i/j)}
        \overline{g}_{mn}^{jc}
        t_{ab}^{mn}
        u_{\mu,c}^i
        -
        \tfrac{1}{2}
        P_{(a/b)}
        \overline{g}_{kb}^{ef}
        t_{ef}^{ij}
        u_{\mu,a}^k
    \end{array}
\end{equation}

\begin{equation}
    \begin{array}{r@{\,}l}
        (\mathbf{B}_{21}(\mathbf{u}_{\mu,1}))_{ijab}
        =
        &
        -
        P^{(i/j)}
        (\mathcal{I}_c^k)_m^i
        t_{ab}^{mj}
        u_{\mu,k}^c
        -
        P_{(a/b)}
        (\mathcal{I}_c^k)_a^e
        t_{eb}^{ij}
        u_{\mu,k}^c
        -
        P_{(a/b)}^{(i/j)}
        \overline{g}_{cb}^{je}
        t_{ae}^{ik}
        u_{\mu,k}^c
        \\[5pt]
        &
        -
        P_{(a/b)}^{(i/j)}
        \overline{g}_{kj}^{mb}
        t_{ac}^{im}
        u_{\mu,k}^c
        -
        \overline{g}_{mc}^{ij}
        t_{ab}^{km}
        u_{\mu,k}^c
        -
        \overline{g}_{ab}^{ke}
        t_{ce}^{ij}
        u_{\mu,k}^c
    \end{array}
\end{equation}

\begin{equation}
    (\mathbf{S}_{11}(\mathbf{u}_{\mu,1}))_{ia}
    =
    \gamma^i_j
    u_{\mu,a}^j
    -
    \gamma^b_a
    u_{\mu,b}^i
\end{equation}

\begin{equation}
    (\mathbf{S}_{11}^{-1}(\mathbf{u}_{\mu,1}))_{ia}
    =
    \frac{%
        (\mathbf{Y}^\dagger)_{j'}^i
        (\mathbf{Y})_a^{b'}
    }{%
        \gamma_{j'}-\gamma_{b'}
    }
    (\mathbf{Y}^\dagger)_{b'}^b
    (\mathbf{Y})_j^{j'}
    u_{\mu,b}^j
\end{equation}

\begin{equation}
    (\mathbf{Y}^\dagger)_{q'}^q
    \gamma_q^p
    (\mathbf{Y})_p^{p'}
    =
    \delta_{q'}^{p'}
    \gamma_{q'}
\end{equation}

\begin{equation}
    \tilde{\mathbf{S}}_{11}
    \equiv
    \tilde{\mathbf{S}}_{11}^{-1}
    \equiv
    \mathbf{1}_1
\end{equation}

\begin{equation}
    (\tilde{\mathbf{A}}_{11})_{ia,ia}
    \equiv
    -
    f_i^i
    +
    f_a^a
\end{equation}

\begin{equation}
    (\tilde{\mathbf{A}}_{22})_{ijab,ijab}
    \equiv
    -
    \mathcal{F}_i^i
    -
    \mathcal{F}_j^j
    -
    \mathcal{F}_a^a
    -
    \mathcal{F}_b^b
\end{equation}

\end{subappendices}
