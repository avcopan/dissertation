\chapter[%
    Algorithms for Linear-Response Density Cumulant Theory
]{%
    Algorithms for Linear-Response Density Cumulant Theory
}
\label{ch:davidson}

\begin{enumerate}
    \item
        \cref{ch:response} presented the LR-ODC-12 model for electronic excited
        states, where excitation energies and transition properties are computed
        by diagonalizing the parameter Hessian of the ODC-12 energy functional,
        with respect to a metric that arises from the time-dependence of the
        parameter responses.
    \item
        Since number of parameters in the ODC-12 variant of density cumulant
        theory scales as
        \(
            \mathcal{O}(o^2v^2)
        \)
        with the number of occupied (\(o\)) and virtual (\(v\)) orbitals, the
        memory requirement for the Hessian matrix scales with the fourth power
        of \(o\) and \(v\), and the number of floating point operations needed
        to diagonalize it scales with the sixth power of these dimensions.
    \item
        Such a brute-force approach will rapidly overwhelm available computing
        resources even for relatively small molecules.
    \item
        For the common scenario where we only care about states within a narrow
        energy range, the cost of diagonalization can be drastically reduced
        through the use of so-called {\itshape direct algorithms} which enable
        the determination of subsets of eigenvectors and eigenvalues without
        explicitly constructing the matrix of a linear transformation in
        computer memory.
    \item
        This chapter will explore the use of the Davidson
        algorithm\cite{Liu:1978p49,Davidson:1975p87}
        in solving the LR-ODC-12 model.
    \item
        After describing the algorithm and discussing the structure of the
        LR-ODC-12 eigenvalue equations in
        \cref{sec:davidson:davidson,sec:davidson:eig}, I will present several
        strategies for solving the LR-ODC-12 model along with a benchmark study
        of their convergence characteristics in \cref{sec:davidson:strategies}.
    \item
        Finally, in \cref{sec:davidson:disk} I will discuss how to reduce memory
        consumption in the Davidson algorithm by partitioning the expansion
        space into blocks and storing large arrays on disk.
\end{enumerate}

\section{The Davidson Algorithm}
\label{sec:davidson:davidson}

P1
\begin{enumerate}
    \item
        Direct algorithms represent linear transformations as functions mapping
        vectors
        \(
            \mathbf{v}
        \)
        in their domain to vectors
        \(
            \mathbf{L}(\mathbf{v})
        \)
        in their codomain, rather than as coefficient arrays
        \(
            [L_{ij}]
            =
            [\mathbf{e}_i \cdot \mathbf{L}(\mathbf{e}_j)]
        \)
        over a complete basis.
    \item
        That is, the result of the transformation is determined {\itshape
        directly}, without explicitly forming its matrix representation in
        computer memory.
    \item
        The Davidson algorithm applies this technique in the context of a matrix
        diagonalization, by progressively growing a basis
        \(
            \{\mathbf{u}_1, \ldots,\mathbf{u}_d\}
        \)
        to span the lowest or highest eigenvectors of a matrix to some threshold
        of accuracy.
    \item
        For a transformation on \(\mathbb{R}^n\), this allows us to reduce our
        computational effort from \(\mathcal{O}(n^3)\) to \(\mathcal{O}(n^2 d)\)
        or even less when \(\mathbf{L}\) is constructed from lower-dimensional
        arrays.
    \item
        Memory requirements are reduced from \(\mathcal{O}(n^2)\) to
        \(\mathcal{O}(nd)\) in the Davidson algorithm, so that as long as the
        dimension of the transformation is large relative to the desired number
        of roots we can gain considerable savings.
\end{enumerate}

\begin{algorithm}
    \caption{%
        Canonical multiroot Davidson algorithm for a generic eigenvalue problem,
        $\mathbf{L}\mathbf{v}_j=\lambda_j\mathbf{G}\mathbf{v}_j$, with periodic
        subspace collapse.
        Requires linear transformation functions and diagonal approximations
        (indicated by tildes) for \(\mathbf{L}\) and \(\mathbf{G}\)
        and solves for the lowest \(k\) eigenvalues and eigenvectors.
    }
    \label{algo:davidson}
    \begin{algorithmic}[1]
        \Procedure{Davidson}{%
            $
            \mathbf{L}(\cdot),
            \mathbf{G}(\cdot),
            \tilde{\mathbf{L}},
            \tilde{\mathbf{G}},
            \mathbf{U}^{(0)},
            k,
            d_\mathrm{max},
            i_\mathrm{max},
            r_\mathrm{tol}
            $%
        }
        \State
        Initialize the expansion space with a set of guess vectors,
        \(\mathbf{U}\leftarrow\mathbf{U}^{(0)}\).
        \For{$1\leq i\leq i_\mathrm{max}$}{}
            \State
            Construct subspace representation and solve the lowest \(k\)
            eigenvalues.
            \[
                \mathbf{L}^\mathrm{sub}
                =
                \mathbf{U}^\dagger
                \mathbf{L}(\mathbf{U})
            \]
            \[
                \mathbf{G}^\mathrm{sub}
                =
                \mathbf{U}^\dagger
                \mathbf{G}(\mathbf{U})
            \]
            \[
                \mathbf{L}^\mathrm{sub}
                \mathbf{v}_j^\mathrm{sub}
                =
                \lambda_j
                \mathbf{G}^\mathrm{sub}
                \mathbf{v}_j^\mathrm{sub}
            \]
            \State
            Calculate the eigenvector residuals over the full space.
            \[
                \mathbf{r}_j
                =
                (
                    \mathbf{L}(\mathbf{U})
                    -
                    \lambda_j
                    \mathbf{G}(\mathbf{U})
                )
                \mathbf{v}_j^\mathrm{sub}
            \]
            \If{$\max(\mathbf{r}_j) < r_\mathrm{tol}$ for all $j$}
                \State
                Set
                \(\mathbf{v}_j\leftarrow\mathbf{U}\mathbf{v}_j^\mathrm{sub}\)
                and quit the loop.  The eigenvectors are converged.
            \EndIf
            \State
            Determine new direction vectors by preconditioning the residual.
            \[
                \mathbf{d}_j^{(i)}
                =
                -
                (
                    \tilde{\mathbf{L}}
                    -
                    \lambda_j
                    \tilde{\mathbf{G}}
                )^{-1}
                \mathbf{r}_j
            \]
            \State
            Project out the span of \(\mathbf{U}\) and orthogonalize via
            SVD compression.
            \[
                \widehat{\mathbf{U}}^{(i)}
                =
                (\mathbf{1} - \mathbf{U}^\dagger \mathbf{U})
                \mathbf{D}^{(i)}
            \]
            \[
                \widehat{\mathbf{U}}^{(i)}
                \approx
                \mathbf{U}^{(i)}
                \mathbf{\Sigma}^{(i)}
                \mathbf{W}^{(i)\dagger}
            \]
            \If{%
                $
                \mathrm{rank}(\mathbf{U})
                +
                \mathrm{rank}(\mathbf{U}^{(i)})
                <
                d_\mathrm{max}
                $%
            }
                \State
                Extend the expansion space,
                \(
                    \mathbf{U}
                    \leftarrow
                    (\mathbf{U}\ \mathbf{U}^{(i)})
                \)
            \Else
                \State
                Collapse the expansion space,
                \(
                    \mathbf{U}
                    \leftarrow
                    (
                        \mathbf{U}
                        \mathbf{v}_1^\mathrm{sub}\ 
                        \cdots\ 
                        \mathbf{U}
                        \mathbf{v}_k^\mathrm{sub}
                    )
                \).
            \EndIf
        \EndFor
        \State
        {\bfseries return}
        \(
            \lambda_j,
            \mathbf{v}_j
        \)
        \EndProcedure
    \end{algorithmic}
\end{algorithm}

\noindent
P2
\begin{enumerate}
    \item
        The procedure for the Davidson algorithm is presented in
        \cref{algo:davidson}, which solves for the eigenvalues and right
        eigenvectors of a generalized eigenvalue problem, which may or may not
        be symmetric.
    \item
        The general strategy of the algorithm is as follows.
    \item
        We expand our transformations in the reduced expansion space,
        \(
            [L_{ij}^\mathrm{sub}]
            =
            [\mathbf{u}_i\cdot \mathbf{L}(\mathbf{u}_j)]
        \),
        and solve the eigenvalue equation in this subspace.
        \begin{equation}
            \mathbf{L}^\mathrm{sub}
            \mathbf{v}_j^\mathrm{sub}
            =
            \lambda_j^\mathrm{trial}
            \mathbf{G}^\mathrm{sub}
            \mathbf{v}_j^\mathrm{sub}
        \end{equation}
    \item
        Expanding this trial solution in the full space as
        \(
            \mathbf{v}_j^\mathrm{trial}
            =
            \mathbf{U}\mathbf{v}_j^\mathrm{sub}
        \),
        the correction vector
        \(
            \mathbf{d}_j
            =
            \mathbf{v}_j^\mathrm{trial}
            -
            \mathbf{v}_j
        \)
        taking us to the exact solution can be approximated as
        \begin{equation}
            \label{eq:davidson:davidson-step}
            \mathbf{d}_j
            \approx
            -
            (
                \tilde{\mathbf{L}}
                -
                \lambda_j^\mathrm{trial}
                \tilde{\mathbf{G}}
            )^{-1}
            \mathbf{r}_j
            \qquad
            \mathbf{r}_j
            \equiv
            (\mathbf{L} - \lambda_j^\mathrm{trial}\mathbf{G})
            \mathbf{v}_j^\mathrm{trial}
        \end{equation}
        where \(\mathbf{r}_j\) is termed the residual vector and
        \(
            (
                \tilde{\mathbf{L}} - \lambda^\mathrm{trial}\tilde{\mathbf{G}}
            )^{-1}
        \)
        is called the preconditioner, which is constructed from diagonal
        approximations to \(\mathbf{L}\) and \(\mathbf{G}\).
    \item
        This can be motivated as an approximate solution to the following
        identity.
        \begin{equation}
            \mathbf{0}
            =
            (\mathbf{L} - \lambda_j\mathbf{G})
            \mathbf{v}_j
            =
            (\mathbf{L} - \lambda_j\mathbf{G})
            (
                \mathbf{v}_j^\mathrm{trial}
                +
                \mathbf{d}_j
            )
        \end{equation}
    \item
        These correction vectors are used to iteratively grow the expansion
        space until it spans the desired eigenvector, \(\mathbf{v}_j\).
    \item
        A key assumption of this algorithm is that the matrices are diagonally
        dominant, otherwise the diagonal approximation in
        \cref{eq:davidson:davidson-step} breaks down and the procedure will
        fail to converge on the desired roots.
    \item
        Efficient implementations of the Davidson algorithm will compute images
        \(\mathbf{L}(\mathbf{u}_i)\) only for the new expansion vectors and
        store these for future iterations.
    \item
        When the expansion space becomes large, one can replace the expansion
        vectors with the current set of trial vectors in order to keep the
        memory requirements manageable, which is known as a subspace collapse.
    \item
        For very large matrices, frequent collapses every second or third
        iteration can be used to keep I/O requirements to a minimum, at the cost
        of slower convergence.\cite{Leininger:2001p1574}
    \item
        This approach is quite general and can be adapted to other large matrix
        problems, such as the linear equation
        \(\mathbf{L}\mathbf{x}=\mathbf{b}\), where it yields a variant of the
        conjugate gradient method.
\end{enumerate}


\section{The LR-ODC-12 Eigenvalue Equation}
\label{sec:davidson:eig}

The LR-ODC-12 eigenvalue equation has a two-by-two block structure which
describes the independent variation of the parameters and their complex
conjugates.
\begin{equation}
    \label{eq:linear-response-eigenvalue-equation}
    \mathbf{E}\mathbf{z}_k
    =
    \omega_k
    \mathbf{M}\mathbf{z}_k
    ,
    \quad
    \mathbf{E}
    =
    \begin{pmatrix}
        \mathbf{A} & \mathbf{B} \\
        \mathbf{B}^* & \mathbf{A}^*
    \end{pmatrix}
    ,
    \quad
    \mathbf{M}
    =
    \begin{pmatrix}
        \mathbf{S} & \mathbf{0} \\
        \mathbf{0} & -\mathbf{S}^*
    \end{pmatrix}
    ,
    \quad
    \mathbf{z}_k
    =
    \begin{pmatrix}
        \mathbf{x}_k \\
        \mathbf{y}_k
    \end{pmatrix}
\end{equation}
This block symmetry leads to a paired system of eigenvalues,
\(
    \{\pm\omega_k\}
\).
The submatrices in \cref{eq:linear-response-eigenvalue-equation} are further
blocked according to whether they describe variations of the one-body
(\(\mathbf{t}_1\)) or two-body (\(\mathbf{t}_2\)) parameters.
\begin{equation}
    \label{eq:conjugate-blocks}
    \mathbf{A}
    =
    \begin{pmatrix}
        \mathbf{A}_{11} & \mathbf{A}_{12} \\
        \mathbf{A}_{21} & \mathbf{A}_{22} \\
    \end{pmatrix}
    \quad
    \mathbf{B}
    =
    \begin{pmatrix}
        \mathbf{B}_{11} & \mathbf{B}_{12} \\
        \mathbf{B}_{21} & \mathbf{B}_{22} \\
    \end{pmatrix}
    \quad
    \mathbf{S}
    =
    \begin{pmatrix}
        \mathbf{S}_{11} & \mathbf{0} \\
        \mathbf{0} & \mathbf{1}_2 \\
    \end{pmatrix}
    \quad
    \mathbf{x}_k
    =
    \begin{pmatrix}
        \mathbf{x}_{k,1} \\
        \mathbf{x}_{k,2}
    \end{pmatrix}
\end{equation}
Following Ref.~\citenum{Oddershede:1984p33}, we can can add and subtract the
block rows of \cref{eq:linear-response-eigenvalue-equation} to arrive at the
following pair of equations (assuming real coefficients).
\begin{equation}
    \label{eq:a-plus-b}
    (\mathbf{A} + \mathbf{B})
    (\mathbf{x}_k + \mathbf{y}_k)
    =
    \omega_k
    \mathbf{S}
    (\mathbf{x}_k - \mathbf{y}_k)
\end{equation}
\begin{equation}
    \label{eq:a-minus-b}
    (\mathbf{A} - \mathbf{B})
    (\mathbf{x}_k - \mathbf{y}_k)
    =
    \omega_k
    \mathbf{S}
    (\mathbf{x}_k + \mathbf{y}_k)
\end{equation}
Multiplying both equations by
\(
    \mathbf{S}^{-1}
\)
and substituting one into the other yields the following non-symmetric
eigenvalue equation for the squares of the excitation energies, reducing the
dimension of the transformation by a factor of two.
\begin{equation}
    \label{eq:davidson:reduced-eigenvalue-equation}
    \mathbf{S}^{-1}
    (
        \mathbf{A} - \mathbf{B}
    )
    \mathbf{S}^{-1}
    (
        \mathbf{A} + \mathbf{B}
    )
    (\mathbf{x}_k + \mathbf{y}_k)
    =
    \omega_k^2
    (\mathbf{x}_k + \mathbf{y}_k)
\end{equation}
Solving this equation only gives us the sum \(\mathbf{x}_k + \mathbf{y}_k\), not
the individual blocks, but these can be recovered by using \cref{eq:a-plus-b} to
compute \(\mathbf{x}_k - \mathbf{y}_k\), so we can still calculate transition
from the reduced eigenvalue equation.

The bottleneck in evaluating these transformations is in the diagonal two-body
Hessian, \(\mathbf{A}_{22}\).
The image of an arbitrary two-body vector
\(
    \mathbf{u}_{\mu,2}
    =
    [u_{\mu,ab}^{ij}]
\)
under this transformation is given by
\begin{equation}
    \label{eq:two-body-hessian-function-in-text}
    \begin{array}{r@{\,}l}
        (\mathbf{A}_{22}(\mathbf{u}_{\mu,2}))_{ijab}
        =
        &
        -
        P_{(a/b)}
        \mathcal{F}_a^c
        u_{\mu,cb}^{ij}
        -
        P^{(i/j)}
        \mathcal{F}_k^i
        u_{\mu,ab}^{kj}
        +
        \tfrac{1}{2}
        \overline{g}_{ab}^{cd}
        u_{\mu,cd}^{ij}
        +
        \tfrac{1}{2}
        \overline{g}_{kl}^{ij}
        u_{\mu,ab}^{kl}
        \\[5pt]
        &
        -
        P_{(a/b)}^{(i/j)}
        \overline{g}_{la}^{jc}
        u_{\mu,cb}^{il}
        +
        \tfrac{1}{2}
        P_{(a/b)}
        \mathcal{G}_{af}^{ec}
        t_{eb}^{ij}
        t_{kl}^{fd*}
        u_{\mu,cd}^{kl}
        +
        \tfrac{1}{2}
        P_{(a/b)}
        \mathcal{G}_{ka}^{me}
        t_{eb}^{ij}
        t_{ml}^{cd*}
        u_{\mu,cd}^{kl}
        \\[5pt]
        &
        +
        \tfrac{1}{2}
        P^{(i/j)}
        \mathcal{G}_{me}^{ic}
        t_{ab}^{mj}
        t_{kl}^{ed*}
        u_{\mu,cd}^{kl}
        +
        \tfrac{1}{2}
        P^{(i/j)}
        \mathcal{G}_{mk}^{in}
        t_{ab}^{mj}
        t_{nl}^{cd*}
        u_{\mu,cd}^{kl}
    \end{array}
\end{equation}
where the \(i,j,k,l,m,n\) run over occupied spin-orbitals and \(a,b,c,d,e,f\)
run over virtual (un-occupied) spin-orbitals with implicity summation over
pairs of upper and lower indices.
See \cref{ch:response} for the definitions of these intermediates.
For reasonably sized basis sets, the rate limiting step is the contraction of
the \(\mathrm{v}^4\) integrals with the expansion vector,
\(\mathrm{g}_{ab}^{cd}u_{\mu,cd}^{ij}\), which scales as
\(\mathcal{O}(d_\mu o^2v^4)\) in the number of floating point
operations.
This term is the rate limiting step in EOM-CCSD as well.
The full set of linear transformation formulas for the LR-ODC-12 Hessian and
metric blocks is given in the appendix
(\cref{sec:linear-transformation-formulas}).

The reduced eigenvalue equation, \cref{eq:davidson:reduced-eigenvalue-equation},
requires us to invert the metric, which is an identity matrix but for the
orbital block, \(\mathbf{S}_{11}\).
This matrix is given by
\begin{equation}
    (\mathbf{S}_{11})_{ia,jb}
    =
    \gamma^i_j
    \delta_a^b
    -
    \delta_j^i
    \gamma^b_a
\end{equation}
where
\(
    \gamma_j^i
\)
and
\(
    \gamma_a^b
\)
are occupied and virtual blocks of the one-body density matrix.
For systems of moderate size this metric could be numerically inverted, but we
can derive a simple and inexpensive formula for the inverse by expanding the
density matrices in the natural spin-orbital (NSO) basis where they are
diagonal.
\begin{equation}
    \gamma_j^i
    =
    (\mathbf{Y})_j^{j'}
    (\mathbf{Y}^\dagger)_{i'}^i
    \delta_{j'}^{i'}
    \gamma_{j'}
    \qquad
    \gamma_a^b
    =
    (\mathbf{Y})_a^{a'}
    (\mathbf{Y}^\dagger)_{b'}^b
    \delta_{a'}^{b'}
    \gamma_{a'}
\end{equation}
Inverting in the NSO basis and transforming back to the original basis yields
\begin{equation}
    (\mathbf{S}_{11}^{-1})_{ia,jb}
    =
    \frac{%
        (\mathbf{Y}^\dagger)_{j'}^i
        (\mathbf{Y})_a^{b'}
    }{%
        \gamma_{j'}-\gamma_{b'}
    }
    (\mathbf{Y}^\dagger)_{b'}^b
    (\mathbf{Y})_j^{j'}
\end{equation}
which scales as \(\mathcal{O}(o^2v^3)\) in the number of floating point
operations.
The same strategy can be used to evaluate other analytic functions of the
metric.


\section{Strategies for Solving the LR-ODC-12 Model}
\label{sec:davidson:strategies}

\paragraph{The full inverse (FI) eigenvalue equation.}
\begin{equation}
    \mathbf{M}(\mathbf{z}_k)
    =
    \omega_k^{-1}
    \mathbf{E}(\mathbf{z}_k)
\end{equation}

\paragraph{The canonical reduced (CR) eigenvalue equation.}
\begin{equation}
    \mathbf{H}^-(\mathbf{H}^+(\mathbf{c}_k^+))
    =
    \omega_k^2
    \mathbf{c}_k^+
    \qquad
    \mathbf{H}^{\pm}
    \equiv
    \mathbf{S}^{-1}
    (
        \mathbf{A} \pm \mathbf{B}
    )
    \qquad
    \mathbf{c}_k^{\pm}
    \equiv
    \mathbf{x}_k \pm \mathbf{y}_k
\end{equation}
\begin{equation}
    \mathbf{x}_k
    =
    \tfrac{1}{2}
    (
        \mathbf{c}_k^+
        +
        \omega_k^{-1}
        \mathbf{H}^+(\mathbf{c}_k^+)
    )
    \qquad
    \mathbf{y}_k
    =
    \tfrac{1}{2}
    (
        \mathbf{c}_k^+
        -
        \omega_k^{-1}
        \mathbf{H}^+(\mathbf{c}_k^+)
    )
\end{equation}

\paragraph{The symmetrized reduced (SR) eigenvalue equation.}
\begin{equation}
    \bar{\mathbf{H}}^-(\bar{\mathbf{H}}^+(\bar{\mathbf{c}}_k^+))
    =
    \omega_k^2
    \bar{\mathbf{c}}_k^+
    \qquad
    \bar{\mathbf{H}}^{\pm}
    \equiv
    \mathbf{S}^{-\frac{1}{2}}
    (
        \mathbf{A} \pm \mathbf{B}
    )
    \mathbf{S}^{-\frac{1}{2}}
    \qquad
    \bar{\mathbf{c}}_k^{\pm}
    \equiv
    \mathbf{S}^{\frac{1}{2}}
    (\mathbf{x}_k \pm \mathbf{y}_k)
\end{equation}
\begin{equation}
    \mathbf{x}_k
    =
    \tfrac{1}{2}
    \mathbf{S}^{-\tfrac{1}{2}}
    (
        \mathbf{c}_k^+
        +
        \omega_k^{-1}
        \mathbf{H}^+(\mathbf{c}_k^+)
    )
    \qquad
    \mathbf{y}_k
    =
    \tfrac{1}{2}
    \mathbf{S}^{-\tfrac{1}{2}}
    (
        \mathbf{c}_k^+
        -
        \omega_k^{-1}
        \mathbf{H}^+(\mathbf{c}_k^+)
    )
\end{equation}


\paragraph{The Fock diagonal (FD) preconditioner.}

\begin{equation}
    (\tilde{\mathbf{S}}_{11})_{ia,ia}
    \equiv
    1
    \qquad
    (\tilde{\mathbf{A}}_{11})_{ia,ia}
    \equiv
    -
    f_i^i
    +
    f_a^a
    \qquad
    (\tilde{\mathbf{A}}_{22})_{ijab,ijab}
    \equiv
    -
    \mathcal{F}_i^i
    -
    \mathcal{F}_j^j
    -
    \mathcal{F}_a^a
    -
    \mathcal{F}_b^b
\end{equation}


\paragraph{The product of exact diagonals (PED) preconditioner.}

\begin{equation}
    \begin{array}{r@{\,}l}
        (\tilde{\mathbf{A}}_{22})_{ijab,ijab}
        \equiv
        &
        -
        \mathcal{F}_i^i
        -
        \mathcal{F}_j^j
        -
        \mathcal{F}_a^a
        -
        \mathcal{F}_b^b
        +
        \overline{g}_{ij}^{ij}
        +
        \overline{g}_{ab}^{ab}
        -
        S(i/j|a/b)
        \overline{g}_{ia}^{ia}
        \\[10pt]
        &
        +
        S(a/b)
        \mathcal{G}_{af}^{ea}
        t_{eb}^{ij}
        t_{ij}^{fb}
        -
        S(a/b)
        \mathcal{G}_{af}^{eb}
        t_{eb}^{ij}
        t_{ij}^{fa}
        +
        2
        S(i/j|a/b)
        \mathcal{G}_{ia}^{me}
        t_{eb}^{ij}
        t_{mj}^{ab}
        \\[10pt]
        &
        +
        S(i/j)
        \mathcal{G}_{mi}^{in}
        t_{ab}^{mj}
        t_{nj}^{ab}
        -
        S(i/j)
        \mathcal{G}_{mj}^{jn}
        t_{ab}^{mi}
        t_{nj}^{ab}
    \end{array}
\end{equation}

\begin{equation}
    \begin{array}{r@{\,}l}
        (\tilde{\mathbf{B}}_{22})_{ijab,ijab}
        \equiv
        &
        +
        S(a/b)
        \mathcal{G}_{aa}^{ef}
        t_{eb}^{ij}
        t_{fb}^{ij}
        -
        S(a/b)
        \mathcal{G}_{ba}^{ef}
        t_{eb}^{ij}
        t_{fb}^{ij}
        +
        2
        S(i/j|a/b)
        \mathcal{G}_{ma}^{ia}
        t_{eb}^{ij}
        t_{ab}^{mj}
        \\[10pt]
        &
        +
        S(i/j)
        \mathcal{G}_{mn}^{ii}
        t_{ab}^{mj}
        t_{ab}^{nj}
        -
        S(i/j)
        \mathcal{G}_{mn}^{ij}
        t_{ab}^{mj}
        t_{ab}^{ni}
    \end{array}
\end{equation}


\afterpage{%
    \clearpage
    \centering
    \begin{landscape}
        \vspace*{\fill}
        \captionof{table}{%
            A comparison of three different solution strategies for the
            LR-ODC-12 model for five molecules, using the def2-SV(P) basis set.
            In each case 10 eigenvectors are converged to \(10^{-5}~\au\),
            starting from an initial expansion space of 100 guess vectors and
            collapsing the subspace very 200 vectors.
            The second and third columns show the number of singles and doubles
            parameters for each system, which determine the dimensions of the
            matrix equation, and the remaining columns give the the number of
            iterations, the run-time, and the number of low-lying roots obtained
            for each strategy.
            The first row for each molecule shows the results for the FD
            preconditioner and the second row shows the results for the
            PED preconditioner.
            All computations were run on an
            Intel\textsuperscript{\textregistered} Core\texttrademark\ i7-5600U
            processor using four threads.
        }
        \vspace{15pt}
        \begin{tabular}{ccccccccccccc}
            \hline
            \hline
            &
            &
            &
            \multicolumn{3}{c}{Full Inverse}
            &
            \multicolumn{3}{c}{Canonical Reduced}
            &
            \multicolumn{3}{c}{Symmetrized Reduced}
            \\
            &
            \(n_1\)
            &
            \(n_2\)
            &
            iter
            &
            time (s)
            &
            roots
            &
            iter
            &
            time (s)
            &
            roots
            &
            iter
            &
            time (s)
            &
            roots
            \\
            \hline
            \ce{H2O}
            & 260 & 14,625
            & 11 & 23 & 10/10 & 16 & 29 & 10/10 & 16 & 30 & 10/10
            \\
            &&
            & 11 & 23 & 10/10 & 24 & 38 & 10/10 & 24 & 40 & 10/10
            \\
            \ce{N2}
            & 588 & 78,351
            & 14 & 186 & 10/10 & 17 & 213 & 9/10 & 17 & 216 & 9/10
            \\
            &&
            & 10 & 156 & 10/10 & 15 & 179 & 9/10 & 15 & 202 & 9/10
            \\
            \ce{HCN}
            & 644 & 94,185
            & 15 & 251 & 9/10 & 21 & 281 & 9/10 & 21 & 331 & 9/10
            \\
            &&
            & 10 & 205 & 9/10 & 18 & 252 & 9/10 & 19 & 308 & 9/10
            \\
            \ce{H2CO}
            & 768 & 135,360
            & 20 & 478 & 7/10 & 28 & 638 & 6/10 & 27 & 650 & 6/10
            \\
            &&
            & 11 & 362 & 7/10 & 23 & 442 & 7/10 & 25 & 614 & 7/10
            \\
            \ce{C2H4}
            & 896 & 184,800
            & 17 & 640 & 9/10 & 25 & 874 & 9/10 & 25 & 891 & 9/10
            \\
            &&
            & 10 & 500 & 9/10 & 19 & 574 & 9/10 & 20 & 722 & 9/10
            \\
            \hline
            \hline
        \end{tabular}
        \vspace*{\fill}
    \end{landscape}
}




\section{Blocking and Disk Storage in the Davidson Algorithm}
\label{sec:davidson:disk}



\begin{subappendices}
\section{LR-ODC-12 Linear Transformation Formulas}
\label{sec:linear-transformation-formulas}

\begin{equation}
    \begin{array}{r@{\,}l}
        (\mathbf{A}_{11}(\mathbf{u}_{\mu,1}))_{ia}
        =
        &
        h_j^i
        \gamma_a^b
        u_{\mu,b}^j
        +
        h_a^b
        \gamma_j^i
        u_{\mu,b}^j
        -
        (\bar{\mathbf{F}})_j^i
        u_{\mu,a}^j
        -
        (\bar{\mathbf{F}})_a^b
        u_{\mu,b}^i
        +
        \overline{g}_{nj}^{mi}
        \gamma_{ma}^{nb}
        u_{\mu,b}^j
        \\[5pt]
        &
        +
        \overline{g}_{ma}^{nb}
        \gamma_{nj}^{mi}
        u_{\mu,b}^j
        +
        \overline{g}_{jf}^{ie}
        \gamma_{ae}^{bf}
        u_{\mu,b}^j
        +
        \overline{g}_{ae}^{bf}
        \gamma_{jf}^{ie}
        u_{\mu,b}^j
        +
        \overline{g}_{me}^{ib}
        \gamma_{ja}^{me}
        u_{\mu,b}^j
        \\[5pt]
        &
        +
        \overline{g}_{ja}^{me}
        \gamma_{me}^{ib}
        u_{\mu,b}^j
    \end{array}
\end{equation}

\begin{equation}
    \begin{array}{r@{\,}l}
        (\mathbf{B}_{11}(\mathbf{u}_{\mu,1}))_{ia}
        =
        &
        \overline{g}_{be}^{im}
        \gamma_{ma}^{je}
        u_{\mu,j}^b
        +
        \overline{g}_{ma}^{je}
        \gamma_{be}^{im}
        u_{\mu,j}^b
        +
        \overline{g}_{mb}^{ie}
        \gamma_{ae}^{jm}
        u_{\mu,j}^b
        +
        \overline{g}_{ae}^{jm}
        \gamma_{mb}^{ie}
        u_{\mu,j}^b
        \\[5pt]
        &
        +
        \tfrac{1}{2}
        \overline{g}_{mn}^{ij}
        \gamma_{ab}^{mn}
        u_{\mu,j}^b
        +
        \tfrac{1}{2}
        \overline{g}_{ab}^{mn}
        \gamma_{mn}^{ij}
        u_{\mu,j}^b
        +
        \tfrac{1}{2}
        \overline{g}_{ef}^{ij}
        \gamma_{ab}^{ef}
        u_{\mu,j}^b
        +
        \tfrac{1}{2}
        \overline{g}_{ab}^{ef}
        \gamma_{ef}^{ij}
        u_{\mu,j}^b
    \end{array}
\end{equation}

\begin{equation}
    \begin{array}{r@{\,}l}
        (\mathbf{A}_{12}(\mathbf{u}_{\mu,2}))_{ia}
        =
        &
        -
        \tfrac{1}{2}
        \overline{g}_{la}^{cd}
        u_{\mu,cd}^{il}
        -
        \tfrac{1}{2}
        \overline{g}_{kl}^{id}
        u_{\mu,ad}^{kl}
        -
        \tfrac{1}{2}
        (\mathcal{I}_a^i)_k^m
        t_{ml}^{cd*}
        u_{\mu,cd}^{kl}
        -
        \tfrac{1}{2}
        (\mathcal{I}_a^i)_e^c
        t_{kl}^{ed*}
        u_{\mu,cd}^{kl}
        \\[5pt]
        &
        -
        \overline{g}_{ae}^{mc}
        t_{ml}^{ed*}
        u_{\mu,cd}^{il}
        -
        \overline{g}_{ke}^{im}
        t_{ml}^{ed*}
        u_{\mu,ad}^{kl}
        -
        \tfrac{1}{4}
        \overline{g}_{la}^{mn}
        t_{mn}^{cd*}
        u_{\mu,cd}^{il}
        \\[5pt]
        &
        -
        \tfrac{1}{4}
        \overline{g}_{ef}^{id}
        t_{kl}^{ef*}
        u_{\mu,ad}^{kl}
    \end{array}
\end{equation}

\begin{equation}
    \begin{array}{r@{\,}l}
        (\mathbf{B}_{12}(\mathbf{u}_{\mu,2}))_{ia}
        =
        &
        -
        \tfrac{1}{2}
        (\mathcal{I}_a^i)_m^k
        t_{cd}^{ml}
        u_{\mu,kl}^{cd}
        -
        \tfrac{1}{2}
        (\mathcal{I}_a^i)_c^e
        t_{ed}^{kl}
        u_{\mu,kl}^{cd}
        -
        \overline{g}_{ad}^{le}
        t_{ce}^{ki}
        u_{\mu,kl}^{cd}
        -
        \overline{g}_{md}^{il}
        t_{ca}^{km}
        u_{\mu,kl}^{cd}
        \\[5pt]
        &
        -
        \tfrac{1}{4}
        \overline{g}_{ma}^{kl}
        t_{cd}^{im}
        u_{\mu,kl}^{cd}
        -
        \tfrac{1}{4}
        \overline{g}_{cd}^{ie}
        t_{ae}^{kl}
        u_{\mu,kl}^{cd}
    \end{array}
\end{equation}

\begin{equation}
    \begin{array}{r@{\,}l}
        (\mathbf{A}_{21}(\mathbf{u}_{\mu,1}))_{ijab}
        =
        &
        -
        P^{(i/j)}
        \overline{g}_{ab}^{jc}
        u_{\mu,c}^i
        -
        P_{(a/b)}
        \overline{g}_{kb}^{ij}
        u_{\mu,a}^k
        -
        P^{(i/j)}
        (\mathcal{I}_k^c)_m^i
        t_{ab}^{mj}
        u_{\mu,c}^k
        \\[5pt]
        &
        -
        P_{(a/b)}
        (\mathcal{I}_k^c)_a^e
        t_{eb}^{ij}
        u_{\mu,c}^k
        -
        P_{(a/b)}^{(i/j)}
        \overline{g}_{ma}^{ce}
        t_{eb}^{mj}
        u_{\mu,c}^i
        -
        P_{(a/b)}^{(i/j)}
        \overline{g}_{km}^{ie}
        t_{eb}^{mj}
        u_{\mu,a}^k
        \\[5pt]
        &
        -
        \tfrac{1}{2}
        P^{(i/j)}
        \overline{g}_{mn}^{jc}
        t_{ab}^{mn}
        u_{\mu,c}^i
        -
        \tfrac{1}{2}
        P_{(a/b)}
        \overline{g}_{kb}^{ef}
        t_{ef}^{ij}
        u_{\mu,a}^k
    \end{array}
\end{equation}

\begin{equation}
    \begin{array}{r@{\,}l}
        (\mathbf{B}_{21}(\mathbf{u}_{\mu,1}))_{ijab}
        =
        &
        -
        P^{(i/j)}
        (\mathcal{I}_c^k)_m^i
        t_{ab}^{mj}
        u_{\mu,k}^c
        -
        P_{(a/b)}
        (\mathcal{I}_c^k)_a^e
        t_{eb}^{ij}
        u_{\mu,k}^c
        -
        P_{(a/b)}^{(i/j)}
        \overline{g}_{cb}^{je}
        t_{ae}^{ik}
        u_{\mu,k}^c
        \\[5pt]
        &
        -
        P_{(a/b)}^{(i/j)}
        \overline{g}_{kj}^{mb}
        t_{ac}^{im}
        u_{\mu,k}^c
        -
        \overline{g}_{mc}^{ij}
        t_{ab}^{km}
        u_{\mu,k}^c
        -
        \overline{g}_{ab}^{ke}
        t_{ce}^{ij}
        u_{\mu,k}^c
    \end{array}
\end{equation}

\begin{equation}
    \begin{array}{r@{\,}l}
        (\mathbf{A}_{22}(\mathbf{u}_{\mu,2}))_{ijab}
        =
        &
        -
        P_{(a/b)}
        \mathcal{F}_a^c
        u_{\mu,cb}^{ij}
        -
        P^{(i/j)}
        \mathcal{F}_k^i
        u_{\mu,ab}^{kj}
        +
        \tfrac{1}{2}
        \overline{g}_{ab}^{cd}
        u_{\mu,cd}^{ij}
        +
        \tfrac{1}{2}
        \overline{g}_{kl}^{ij}
        u_{\mu,ab}^{kl}
        \\[5pt]
        &
        -
        P_{(a/b)}^{(i/j)}
        \overline{g}_{la}^{jc}
        u_{\mu,cb}^{il}
        +
        \tfrac{1}{2}
        P_{(a/b)}
        \mathcal{G}_{af}^{ec}
        t_{eb}^{ij}
        t_{kl}^{fd*}
        u_{\mu,cd}^{kl}
        +
        \tfrac{1}{2}
        P_{(a/b)}
        \mathcal{G}_{ka}^{me}
        t_{eb}^{ij}
        t_{ml}^{cd*}
        u_{\mu,cd}^{kl}
        \\[5pt]
        &
        +
        \tfrac{1}{2}
        P^{(i/j)}
        \mathcal{G}_{me}^{ic}
        t_{ab}^{mj}
        t_{kl}^{ed*}
        u_{\mu,cd}^{kl}
        +
        \tfrac{1}{2}
        P^{(i/j)}
        \mathcal{G}_{mk}^{in}
        t_{ab}^{mj}
        t_{nl}^{cd*}
        u_{\mu,cd}^{kl}
    \end{array}
\end{equation}

\begin{equation}
    \begin{array}{r@{\,}l}
        (\mathbf{B}_{22}(\mathbf{u}_{\mu,2}))_{ijab}
        =
        &
        \tfrac{1}{2}
        P_{(a/b)}
        \mathcal{G}_{ac}^{ef}
        t_{eb}^{ij}
        t_{fd}^{kl}
        u_{\mu,kl}^{cd}
        +
        \tfrac{1}{2}
        P_{(a/b)}
        \mathcal{G}_{na}^{ke}
        t_{eb}^{ij}
        t_{cd}^{nl}
        u_{\mu,kl}^{cd}
        \\[5pt]
        &
        +
        \tfrac{1}{2}
        P^{(i/j)}
        \mathcal{G}_{mc}^{if}
        t_{ab}^{mj}
        t_{fd}^{kl}
        u_{\mu,kl}^{cd}
        +
        \tfrac{1}{2}
        P^{(i/j)}
        \mathcal{G}_{mn}^{ik}
        t_{ab}^{mj}
        t_{cd}^{nl}
        u_{\mu,kl}^{cd}
    \end{array}
\end{equation}

\begin{equation}
    (\mathbf{S}_{11}(\mathbf{u}_{\mu,1}))_{ia}
    =
    \gamma^i_j
    u_{\mu,a}^j
    -
    \gamma^b_a
    u_{\mu,b}^i
\end{equation}

\begin{equation}
    (\mathbf{S}_{11}^{-1}(\mathbf{u}_{\mu,1}))_{ia}
    =
    \frac{%
        (\mathbf{Y}^\dagger)_{j'}^i
        (\mathbf{Y})_a^{b'}
    }{%
        \gamma_{j'}-\gamma_{b'}
    }
    (\mathbf{Y}^\dagger)_{b'}^b
    (\mathbf{Y})_j^{j'}
    u_{\mu,b}^j
\end{equation}

\begin{equation}
    (\mathbf{Y}^\dagger)_{q'}^q
    \gamma_q^p
    (\mathbf{Y})_p^{p'}
    =
    \delta_{q'}^{p'}
    \gamma_{q'}
\end{equation}

\begin{equation}
    \begin{array}{r@{\,}l}
        (\tilde{\mathbf{A}}_{22})_{ijab,ijab}
        \equiv
        &
        -
        \mathcal{F}_i^i
        -
        \mathcal{F}_j^j
        -
        \mathcal{F}_a^a
        -
        \mathcal{F}_b^b
        +
        \overline{g}_{ij}^{ij}
        +
        \overline{g}_{ab}^{ab}
        -
        S(i/j|a/b)
        \overline{g}_{ia}^{ia}
        \\[10pt]
        &
        +
        S(a/b)
        \mathcal{G}_{af}^{ea}
        t_{eb}^{ij}
        t_{ij}^{fb}
        -
        S(a/b)
        \mathcal{G}_{af}^{eb}
        t_{eb}^{ij}
        t_{ij}^{fa}
        +
        2
        S(i/j|a/b)
        \mathcal{G}_{ia}^{me}
        t_{eb}^{ij}
        t_{mj}^{ab}
        \\[10pt]
        &
        +
        S(i/j)
        \mathcal{G}_{mi}^{in}
        t_{ab}^{mj}
        t_{nj}^{ab}
        -
        S(i/j)
        \mathcal{G}_{mj}^{jn}
        t_{ab}^{mi}
        t_{nj}^{ab}
    \end{array}
\end{equation}

\begin{equation}
    \begin{array}{r@{\,}l}
        (\tilde{\mathbf{B}}_{22})_{ijab,ijab}
        \equiv
        &
        +
        S(a/b)
        \mathcal{G}_{aa}^{ef}
        t_{eb}^{ij}
        t_{fb}^{ij}
        -
        S(a/b)
        \mathcal{G}_{ba}^{ef}
        t_{eb}^{ij}
        t_{fb}^{ij}
        +
        2
        S(i/j|a/b)
        \mathcal{G}_{ma}^{ia}
        t_{eb}^{ij}
        t_{ab}^{mj}
        \\[10pt]
        &
        +
        S(i/j)
        \mathcal{G}_{mn}^{ii}
        t_{ab}^{mj}
        t_{ab}^{nj}
        -
        S(i/j)
        \mathcal{G}_{mn}^{ij}
        t_{ab}^{mj}
        t_{ab}^{ni}
    \end{array}
\end{equation}

\end{subappendices}
