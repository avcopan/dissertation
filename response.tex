\chapter[%
    Linear-Response Density Cumulant Theory for Excited States:\\
	First Implementation and Benchmark Calculations
]{%
    Linear-Response Density Cumulant Theory for Excited States:\\
	First Implementation and Benchmark Calculations\footnote{%
        A.~V.~Copan and A.~Yu.~Sokolov (to be submitted in
        J.~Chem.~Theory~Comput).
    }
}
\label{ch:response}


\section{Abstract}

We present a linear-response formulation of density cumulant functional theory
(DCT) that provides accurate access to many electronic states.
DCT expresses the electronic energy as a Hermitian, size-extensive, and
stationary functional of the one-particle density matrix and the two-particle
density cumulant.
In the original DCT formulation only the information about a single electronic
state (usually, the ground state) is obtained.
In this research, we combine DCT with linear response theory to obtain
information about many electronic states simultaneously.
We discuss the derivation of linear-response DCT, present its implementation for
the ODC-12 method (LR-ODC-12), and benchmark its performance against highly
accurate equation-of-motion coupled cluster theory with up to full triple
excitations (EOM-CCSDT).
Our results for a set of small molecules demonstrate that LR-ODC-12 vertical
excitation energies are in closer agreement with EOM-CCSDT than those obtained
from equation-of-motion coupled cluster theory with up to double excitations
(EOM-CCSD).
In addition, we report a linear-response formulation of the orbital-optimized
linearized coupled cluster theory with double excitations (LR-OLCCD), which we
obtain by neglecting the non-linear terms in the LR-ODC-12 equations.


\section{Introduction}

Accurate simulation of excited electronic states remains one of the major
challenges in modern electronic structure theory. 
{\it Ab initio}\/ methods for excited states can be divided into
single-reference and multi-reference categories, based on their ability to
treat static electron correlation.
Multi-reference methods can correctly describe static correlation in
near-degenerate valence orbitals and electronic states with multiple-excitation
character, but often lack accurate treatment of important dynamic correlation
effects (e.g., multi-configurational self-consistent field or multi-reference
perturbation theories)
\cite{Knowles:1985p259,Werner:1985p5053,Wolinski:1987p225,Hirao:1992p374,Finley:1998p299,Andersson:1990p5483,Andersson:1992p1218,Werner:1996p645,Angeli:2001p10252,Angeli:2001p297}
or become very costly when the number of near-degenerate orbitals is large
(e.g., multi-reference configuration interaction or coupled cluster theories).
\cite{Mukherjee:1977p955,Lindgren:1978p33,Siegbahn:1980p1647,Jeziorski:1981p1668,Werner:1988p5803,Mahapatra:1998p157,Mahapatra:1999p6171,Pittner:2003p10876,Evangelista:2007p024102,Datta:2011p214116,Evangelista:2011p114102,Kohn:2013p176,Nooijen:2014p081102}
Meanwhile,
single-reference methods
\cite{Foresman:1992p135,Sherrill:1999p143,Geertsen:1989p57,Comeau:1993p414,Stanton:1993p7029,Krylov:2008p433,Crawford:2000p33,Shavitt:2009,Sekino:1984p255,Koch:1990p3345,Koch:1990p3333,Nooijen:1997p6441,Nooijen:1997p6812,Nakatsuji:1978p2053,Nakatsuji:1979p329}
often provide a compromise between the computational
cost and accuracy, and can be used to reliably compute properties of molecules
in low-lying electronic states near the equilibrium geometries. In these
situations, single-reference equation-of-motion coupled cluster theory
(EOM-CC)
\cite{Geertsen:1989p57,Comeau:1993p414,Stanton:1993p7029,Krylov:2008p433,Crawford:2000p33,Shavitt:2009}
is usually the method of choice, especially when high accuracy
is desired. 

The EOM-CC methods yield size-intensive excitation energies
\cite{Koch:1990p3345,Koch:1990p3333}
and can be
systematically improved by increasing the excitation rank of the cluster
operator in the exponential parametrization of the wavefunction. Although EOM-CC
is usually formulated in the context of a similarity-transformed Hamiltonian,
its excitation energies are equivalent to those obtained from linear-response
coupled cluster theory (LR-CC).
\cite{Sekino:1984p255,Koch:1990p3345,Koch:1990p3333}
Both EOM-CC and LR-CC are based on non-Hermitian eigenvalue problems,
complicating the computation of molecular properties (e.g., transition dipoles)
by requiring evaluation of left and right eigenvectors.
\cite{Stanton:1993p8840,Stanton:1994p4695,Stanton:1994p8938,Levchenko:2005p224106}
Several Hermitian alternatives to EOM-CC and LR-CC have been
proposed to avoid these problems, such as  
algebraic diagrammatic construction
\cite{Schirmer:1982p2395,Schirmer:1991p4647,Schirmer:2004p11449,Harbach:2014p064113,Dreuw:2014p82}, 
unitary and variational LR-CC,
\cite{Taube:2006p3393,Kats:2011p062503,Walz:2012p052519}
similarity-constrained CC,
\cite{Kjonstad:2017p4801}
and propagator-based LR-CC.
\cite{Moszynski:2005p1109,Korona:2010p14977}

In this work, we present the development of linear-response density cumulant
functional theory (LR-DCT), a size-intensive approach for excited
electronic states. In density cumulant functional theory (DCT),
\cite{Kutzelnigg:2006p171101,Simmonett:2010p174122,Sokolov:2012p054105,Sokolov:2013p024107,Sokolov:2013p204110,Sokolov:2014p074111,Wang:2016p4833}
the
electronic energy is obtained by optimizing the energy functional directly in
terms of the one-particle reduced density matrix and the two-particle density
cumulant, a fully connected part of the two-particle reduced density matrix
(2-RDM).
\cite{Fulde:1991,Ziesche:1992p597,Kutzelnigg:1997p432,Mazziotti:1998p419,Mazziotti:1998p4219,Kutzelnigg:1999p2800,Ziesche:2000p33,Herbert:2007p261,Kong:2011p214109,Hanauer:2012p50}
In this regard, DCT is related to approaches that are based on
the variational optimization
\cite{Colmenero:1993p979,Nakatsuji:1996p1039,Mazziotti:1998p4219,Mazziotti:2006p143002,Kollmar:2006p084108,DePrince:2007p042501,DePrince:2016p164109}
or parametrization
\cite{Mazziotti:2008p253002,Mazziotti:2010p062515,DePrince:2012p1917}
of 2-RDM\@. On
the other hand, DCT has a close relationship
\cite{Sokolov:2013p024107,Sokolov:2013p204110}
with wavefunction-based electronic
structure theories, such as linearized, unitary, and variational coupled
cluster theory.
\cite{Kutzelnigg:1991p349,Kutzelnigg:1998p65,VanVoorhis:2000p8873,Kutzelnigg:1982p3081,Bartlett:1989p133,Watts:1989p359,Szalay:1995p281,Cooper:2010p234102,Evangelista:2011p224102}
In contrast to variational 2-RDM theory
\cite{Nakata:2009p042109,vanAggelen:2010p114112,Verstichel:2010p114113}
and traditional
coupled cluster methods [e.g., CCSD and CCSD(T)],
\cite{Crawford:2000p33,Shavitt:2009}
DCT naturally combines
size-extensivity and a Hermitian energy functional. In addition, the DCT
electronic energy is fully relaxed with respect to all of its parameters, which
greatly simplifies computation of the first-order molecular properties.
\cite{Scheiner:1987p5361,Salter:1989p1752,Gauss:1991p2623,Gauss:1991p207}
We have successfully applied DCT to a variety of chemical systems with
different electronic structure effects (e.g., open-shell, symmetry-breaking,
and multi-reference).
\cite{Sokolov:2013p204110,Sokolov:2014p074111,Wang:2016p4833,Copan:2014p2389,Mullinax:2015p2487}
One limitation of the original DCT formulation is
that it can only obtain information about the lowest-energy state of a
particular symmetry (usually, the ground state). By combining DCT with linear
response theory, we remove this limitation, providing access to many electronic
states simultaneously.

We begin with a brief overview of DCT (\cref{sec:dct}) and linear response
theory (\cref{sec:lr}). We then discuss the derivation of linear-response
theory for the ODC-12 method (LR-ODC-12, \cref{sec:lr_odc12}).
In section \cref{sec:olccd}, 
we derive equations for the linear-response orbital-optimized linearized
coupled cluster theory with double excitations (LR-OLCCD), which we obtain by
neglecting the non-linear terms in the LR-ODC-12 equations. 
We outline the
computational details in \cref{sec:comp_details}.
In section \cref{sec:results}, we demonstrate that the LR-ODC-12 excitation
energies are size-intensive (\cref{sec:size_intensivity}), test the performance
of LR-ODC-12 for the dissociation of \ce{H2} (\cref{sec:two_electron}), and
benchmark the accuracy of LR-ODC-12 for vertical
excitation energies of small molecules (\cref{sec:vert_excit}).
Finally, we present our conclusions in section \cref{sec:conclusions}.


\section{Theory}

\subsection{Overview of Density Cumulant Theory}
\label{sec:dct}
We begin with a brief overview of DCT for a single electronic state.
Our starting point is to express the electronic energy as a trace of the one-
and antisymmetrized two-electron integrals (\( h_p^q \) and
\(\overline{g}_{pq}^{rs}\)) with the reduced one- and two-body density matrices
(\(\gamma^p_q\) and \(\gamma^{pq}_{rs}\)):
\begin{equation}
    \label{eq:energy-expression}
    E
    =
    h_p^q
    \gamma^p_q
    +
    \tfrac{1}{4}
    \overline{g}_{pq}^{rs}
    \gamma^{pq}_{rs}
\end{equation}
where summation over the repeated indices is implied.
In DCT, the two-body density matrix \(\gamma^{pq}_{rs}\) is expanded in terms of
its connected part, the two-body density cumulant ($\lambda^{pq}_{rs}$), and its
disconnected part, which is given by an antisymmetrized product of one-body
density matrices:\cite{Kutzelnigg:2006p171101}
\begin{equation}
    \label{eq:two-body-n-rep}
    \gamma^{pq}_{rs}
    =
    \langle\Psi|
    a^{pq}_{rs}
    |\Psi\rangle
    =
    \lambda^{pq}_{rs}
    +
    P_{(r/s)}
    \gamma^p_r
    \gamma^q_s
\end{equation}
where \(P_{(r/s)}v_{rs} = v_{rs} - v_{sr}\) denotes antisymmetrization and
\mbox{$a^{pq}_{rs}=a^{\dag}_{p}a^{\dag}_{q}a^{}_{s}a^{}_{r}$} is the two-body operator in second quantization.
The one-body density matrix \(\gamma^p_q\) is determined from its non-linear
relationship to the cumulant's partial trace:\cite{Sokolov:2013p024107}
\begin{equation}
    \label{eq:one-body-n-rep}
    \gamma^p_q
    =
    \gamma^p_r
    \gamma^r_q
    -
    \lambda^{pr}_{qr}
\end{equation}
This reduces \cref{eq:energy-expression} to a functional of the two-body
cumulant and the basis of spin-orbitals, thereby defining the DCT energy
functional.
The density cumulant is parametrized by choosing a specific Ansatz for the
wavefunction \(|\Psi\rangle\) such that\cite{Sokolov:2014p074111}
\begin{equation}
    \label{eq:cumulant-parametrization}
    \lambda^{pq}_{rs}
    =
    \langle\Psi|
    a^{pq}_{rs}
    |\Psi\rangle_\mathrm{c}
\end{equation}
where $\mathrm{c}$ indicates that only fully connected terms are included in the
parametrization.
\cref{eq:cumulant-parametrization} can be considered as a set of
\(n\)-representability conditions that ensure that the resulting one- and
two-electron density matrices represent a physical \(n\)-electron wavefunction.
To compute the DCT energy, the functional \eqref{eq:energy-expression} is made
stationary with respect to all of its parameters.
Importantly, due to the connected nature of \cref{eq:cumulant-parametrization},
DCT is both size-consistent and size-extensive for any parametrization of
\(|\Psi\rangle\), and is exact in the limit of a complete
parametrization.\cite{Sokolov:2014p074111}

In this work, we consider the ODC-12
method,\cite{Sokolov:2013p024107,Sokolov:2013p204110} which parametrizes the
cumulant through a unitary treatment of single excitations and a linear
expansion of double excitations.
\begin{equation}
    \label{eq:odc12-wavefunction}
    |\Psi\rangle
    =
    e^{T_1-T_1^\dagger}
    (1 + \hat{T}_2)
    |\Phi\rangle
\end{equation}
\begin{equation}
    T_1
    =
    \mathbf{t}_1\cdot\mathbf{a}_1
    =
    t_a^i
    a^a_i
\end{equation}
\begin{equation}
    T_2
    =
    \mathbf{t}_2\cdot\mathbf{a}_2
    =
    \tfrac{1}{4}
    t_{ab}^{ij}
    a^{ab}_{ij}
\end{equation}
{\color{red}(missing hats for the operators in many places. I generally prefer to keep hats in the equations)}
The exponential singles operator \(e^{T_1-T_1^\dagger}\) has the effect of a
unitary transformation of the spin-orbital basis and is incorportated in our
implementation of the ODC-12 method by optimizing the
orbitals.\cite{Sokolov:2013p204110}
The \(\mathbf{t}_1\) and \(\mathbf{t}_2\) parameters are obtained from the
stationarity conditions
\begin{equation}
    \label{eq:stationarity_conditions}
    \dfrac{\partial E}{\partial \mathbf{t}_1^\dagger}
    \overset{!}{=}
    0 \ ,
    \qquad
    \dfrac{\partial E}{\partial \mathbf{t}_2^\dagger}
    \overset{!}{=}
    0
\end{equation}
and are used to compute the ODC-12 energy.
Explicit equations for the stationarity conditions are given in Refs.\@
\citenum{Sokolov:2013p024107} and \citenum{Sokolov:2013p204110}.
Although in ODC-12 the wavefunction parametrization is linear with respect
to double excitations (\cref{eq:odc12-wavefunction}), the ODC-12 energy
stationarity conditions are non-linear in $\mathbf{t}_2$ due to the non-linear
relationship between the one-particle density matrix and the density cumulant
(\cref{eq:one-body-n-rep}).\cite{Sokolov:2013p024107} Neglecting the non-linear
$\mathbf{t}_2$ terms in \cref{eq:stationarity_conditions} results in the
equations that define the linearized orbital-optimized  coupled cluster doubles
method (OLCCD).
This method is equivalent to the orbital-optimized coupled electron pair
approximation zero (OCEPA$_0$).\cite{Bozkaya:2013p054104}

\subsection{Linear Response Theory}
\label{sec:lr}
We now briefly review linear response theory in the quasi-energy
formulation.\cite{Norman:2011p20519}
The quasi-energy of a system perturbed by a time-dependent interaction
\(\hat{V}f(t)\) is defined as
\begin{equation}
    Q(t)
    =
    \langle\Psi(t)|
    \hat{H} + \hat{V} f(t) - i\tfrac{\partial}{\partial t}
    |\Psi(t)\rangle
\end{equation}
where \(\Psi(t)\) is the phase-isolated wavefunction, from which the usual
Schr\"odinger wavefunction can be recovered as
\(
    e^{-i\int_0^t dt' Q(t')}
    \Psi(t)
\).
Assuming that the perturbation is periodic
\begin{equation}
    f(t)
    =
    \sum_\omega f(\omega) e^{-i\omega t}
\end{equation}
the time average of the quasi-energy over a period of oscillation, denoted as 
\(
    \{Q(t)\}
\),
is variational with respect to the exact dynamic state.\cite{Helgaker:2012p543}
The independent parameters \(\mathbf{u}(t)\) that define such a state can be
written using a Fourier expansion
\begin{equation}
    \label{eq:parameter-fourier-expansion}
    \mathbf{u}(t)
    =
    \sum_{n=0}^\infty
    \sum_{\omega_1\cdots \omega_n}
    \mathbf{u}(\omega_1,\ldots, \omega_n)
    e^{-i(\omega_1+\cdots+\omega_n)t}
\end{equation}
where the outer sum runs over polynomial orders in \(f(t)\).
The stationarity of the time-averaged quasi-energy then implies the following
relationship\cite{Kristensen:2009p044112}
\begin{equation}
    \label{eq:linear-response-equation}
    \begin{array}{l}
        0
        =
        \left.
            \dfrac{d}{df(\omega)}
            \dfrac{%
                \partial \{Q(t)\}
            }{%
                \partial \mathbf{u}^\dagger(\omega)
            }
        \right|_{f=0}
        =
        \\[20pt]
        \left.
            \dfrac{%
                \partial^2 \{Q(t)\}
            }{%
                \partial \mathbf{u}^\dagger(\omega)
                \partial \mathbf{u}(\omega)
            }
            \dfrac{\partial \mathbf{u}(\omega)}{\partial f(\omega)}
        \right|_{f=0}
            +
        \left.
            \dfrac{%
                \partial^2 \{Q(t)\}
            }{%
                \partial \mathbf{u}^\dagger(\omega)
                \partial f(\omega)
            }
        \right|_{f=0}
    \end{array}
\end{equation}
which constitutes a linear equation for the first-order response of the system
to the perturbation. 
When the frequency $\omega$ is in resonance with an excitation energy of the
system, \cref{eq:linear-response-equation} will result in an infinite
first-order response
\(
    \dfrac{\partial \mathbf{u}(\omega)}{\partial f(\omega)}
\).
From \cref{eq:linear-response-equation}, we find that these poles occur when the
Hessian matrix of the quasi-energy with respect to the wavefunction parameters
\(\mathbf{u}(\omega)\) becomes singular.
We can express this Hessian matrix in the form
\begin{equation}
    \label{eq:quasi-energy-hessian}
    \left.
        \frac{\partial^2\{Q(t)\}}{%
            \partial \mathbf{u}^\dagger(\omega)
            \partial \mathbf{u}(\omega)
        }
    \right|_{f=0}
    \equiv
    \mathbf{E}
    -
    \omega\,
    \mathbf{M}
\end{equation}
where \(\mathbf{E}\) is the Hessian of the time-averaged electronic energy
\(\{\langle\Psi(t)|\hat{H}|\Psi(t)\rangle\}\) and \(\omega\mathbf{M}\) is the
Hessian of the time-derivative overlap
\(\{\langle\Psi(t)|i\dot{\Psi}(t)\rangle\}\).
The excitation energies of the system $\omega_k$ can therefore be determined by
solving the following generalized eigenvalue equation:
\begin{equation}
    \label{eq:linear-response-energy-eigenvalue-equation}
    \mathbf{E}\mathbf{z}_k
    =
    \omega_k
    \mathbf{M}\mathbf{z}_k
\end{equation}
where \(\mathbf{M}\) serves as the metric matrix.
\cref{eq:linear-response-energy-eigenvalue-equation} allows the determination of
excitation energies for an arbitrary parametrization of $|\Psi(t)\rangle$.

The generalized eigenvectors \(\mathbf{z}_k\) can be used to
compute transition properties for excited states.
In particular, in the exact linear response theory\cite{Olsen:1985p3235} the transition strength
of the perturbing interaction,
\(
    |\langle\Psi|\hat{V}|\Psi_k\rangle|^2
\),
is equal to the complex residue of the following quantity at
\(\omega\rightarrow\omega_k\):
\begin{equation}
    \langle\!\langle \hat{V}; \hat{V} \rangle\!\rangle_\omega
    \equiv
    \left.
    \mathbf{v}'^\dagger
    \cdot
    \frac{\partial \mathbf{u}(\omega)}{\partial f(\omega)}
    \right|_{f=0}
\end{equation}
This quantity is known as the linear response function and
\(
    \mathbf{v}'
\)
is termed the property gradient vector,\cite{Sauer:2011} which is defined as
follows:
\begin{equation}
    \label{eq:property-gradient-vector}
    \mathbf{v}'
    \equiv
    \left.
    \frac{%
        \partial^2 \{Q(t)\}
    }{%
        \partial \mathbf{u}^\dagger(\omega)
        \partial f(\omega)
    }
    \right|_{f=0}
\end{equation}
Substituting \cref{eq:property-gradient-vector,eq:quasi-energy-hessian} into
\cref{eq:linear-response-equation} and decomposing the quasi-energy Hessian as
\begin{equation}
    \mathbf{E} - \omega\mathbf{M}
    =
    (\mathbf{Z}^\dagger)^{-1}
    (\mathbf{Z}^\dagger \mathbf{M} \mathbf{Z})
    (\boldsymbol\Omega - \omega\mathbf{1})
    (\mathbf{Z})^{-1}
\end{equation}
where \(\boldsymbol\Omega\) is a diagonal matrix of generalized eigenvalues and
\(\mathbf{Z}\) is a matrix of generalized eigenvectors that simulataneously
diagonalizes \(\mathbf{E}\) and \(\mathbf{M}\), we obtain the
general formula for the transition strengths:
\begin{equation}
    \lim_{\omega\rightarrow \omega_k}
    (\omega-\omega_k)
    \langle\!\langle \hat{V}; \hat{V} \rangle\!\rangle_\omega
    =
    \frac{%
        |\mathbf{z}_k^\dagger \mathbf{v}'|^2
    }{%
        \mathbf{z}_k^\dagger \mathbf{M}\mathbf{z}_k
    }
\end{equation}
In \cref{sec:lr_odc12}, we will use the quasi-energy formalism to derive
equations for the linear-response ODC-12 method (LR-ODC-12).


\subsection{Linear-Response ODC-12}
\label{sec:lr_odc12}
In the ODC-12 method, the electronic energy Hessian can be written in the following
form
\begin{equation}
    \label{eq:lr-odc12-hessian-blocks}
    \mathbf{E}
    =
    \begin{pmatrix}
        \mathbf{A}_{11} & \mathbf{A}_{12} & \mathbf{B}_{11} & \mathbf{B}_{12} \\
        \mathbf{A}_{21} & \mathbf{A}_{22} & \mathbf{B}_{21} & \mathbf{B}_{22} \\
        \mathbf{B}_{11}^* & \mathbf{B}_{12}^* & \mathbf{A}_{11}^* & \mathbf{A}_{12}^* \\
        \mathbf{B}_{21}^* & \mathbf{B}_{22}^* & \mathbf{A}_{21}^* & \mathbf{A}_{22}^* \\
    \end{pmatrix}
\end{equation}
where the submatrices are defined in general as
\begin{equation}
    \label{eq:hessian-blocks}
    \mathbf{A}_{nm}
    =
    \left.
    \frac{\partial^2 E}{%
        \partial \mathbf{t}_n^\dagger
        \partial \mathbf{t}_m
    }
    \right|_{f=0}
    ,\ 
    \mathbf{B}_{nm}
    =
    \left.
    \frac{\partial^2 E}{%
        \partial \mathbf{t}_n^\dagger
        \partial \mathbf{t}_m^*
    }
    \right|_{f=0}.
\end{equation}
These complex derivatives relate to the second derivatives of the
electronic energy with respect to variations of the orbitals ($\mathbf{A}_{11}$, $\mathbf{B}_{11}$) and
cumulant parameters ($\mathbf{A}_{22}$, $\mathbf{B}_{22}$).
Similarly, the mixed second derivatives couple variations in the orbitals
and cumulant parameters ($\mathbf{A}_{12}$, $\mathbf{B}_{12}$). 
The metric matrix \(\mathbf{M}\) has a block-diagonal structure, as a
consequence of the linear parametrization of the wavefunction in
\cref{eq:odc12-wavefunction}:
\begin{equation}
    \label{eq:lr-odc12-metric-blocks}
    \mathbf{M}
    =
    \begin{pmatrix}
        \mathbf{S}_{11} & \mathbf{0} & \mathbf{0} & \mathbf{0} \\
        \mathbf{0} & \mathbf{1}_2 & \mathbf{0} & \mathbf{0} \\
        \mathbf{0} & \mathbf{0} & -\mathbf{S}_{11}^* & \mathbf{0} \\
        \mathbf{0} & \mathbf{0} & \mathbf{0} & -\mathbf{1}_2 \\
    \end{pmatrix}
\end{equation}
where
\(
    \mathbf{1}_2
    =
    \langle\Phi|\mathbf{a}_2^\dagger \mathbf{a}_2|\Phi\rangle
\)
is an identity matrix over the space of unique two-body excitations and the
orbital metric is defined as follows:
\begin{equation}
    \label{eq:metric-blocks}
    \omega\mathbf{S}_{11}
    =
    \left.
        \frac{\partial^2 \{\langle\Psi(t)|i\dot\Psi(t)\rangle\}}{%
            \partial \mathbf{t}_1^\dagger(\omega)
            \partial \mathbf{t}_1(\omega)
        }
    \right|_{f=0}
\end{equation}
Equations for all blocks of $\mathbf{E}$, $\mathbf{M}$, and the property gradient
vector $\mathbf{v}'$ are shown explicitly in the Supporting Information.
We note that, due to the Hermitian nature of the DCT energy functional
\eqref{eq:energy-expression}, the ODC-12 energy Hessian $\mathbf{E}$ is always
symmetric.
As a result, in the absence of instabilities (i.e., as long as the Hessian is
positive semi-definite), the LR-ODC-12 excitation energies are guaranteed to
have real values.

To illustrate the derivation of the LR-ODC-12 energy Hessian, let us consider
the diagonal two-body block of \(\mathbf{E}\).
Expressing the energy \eqref{eq:energy-expression} using the cumulant expansion
\eqref{eq:two-body-n-rep} and differentiating with respect to $\mathbf{t}_2$, we
obtain:
\begin{equation}
    \label{eq:odc12-hessian-initial-form}
    \begin{array}{r@{\,}l}
        \mathbf{A}_{22}
        =
        \dfrac{\partial^2 E}{%
            \partial\mathbf{t}_2^\dagger
            \partial\mathbf{t}_2
        }
        =
        &
        f_p^q
        \dfrac{\partial^2 \gamma^p_q}{%
            \partial\mathbf{t}_2^\dagger
            \partial\mathbf{t}_2
        }
        +
        \overline{g}_{pr}^{qs}
        \dfrac{\partial \gamma^p_q}{\partial\mathbf{t}_2^\dagger}
        \dfrac{\partial \gamma^r_s}{\partial\mathbf{t}_2}
        \\[15pt]
        &
        +
        \tfrac{1}{4}
        \overline{g}_{pq}^{rs}
        \dfrac{\partial^2 \lambda^{pq}_{rs}}{%
            \partial\mathbf{t}_2^\dagger
            \partial\mathbf{t}_2
        }
    \end{array}
\end{equation}
where we have introduced the generalized Fock matrix
\(
    f_p^q
    \equiv
    h_p^q
    +
    \overline{g}_{pr}^{qs}
    \gamma^r_s
\).
The derivatives of the one-body density matrix can be expressed in terms of the derivatives of the density cumulant
\begin{equation}
    \begin{array}{r@{\,}l}
    \label{eq:a22_odc12}
	\mathbf{A}_{22}
        =
        &
        \mathcal{F}_p^q
        \dfrac{\partial^2 \lambda^{pt}_{qt}}{%
            \partial\mathbf{t}_2^\dagger
            \partial\mathbf{t}_2
        }
        +
        \mathcal{G}_{pr}^{qs}
        \dfrac{\partial \lambda^{pt}_{qt}}{\partial\mathbf{t}_2^\dagger}
        \dfrac{\partial \lambda^{ru}_{su}}{\partial\mathbf{t}_2}
        \\[15pt]
        &
        +
        \tfrac{1}{4}
        \overline{g}_{pq}^{rs}
        \dfrac{\partial^2 \lambda^{pq}_{rs}}{%
            \partial\mathbf{t}_2^\dagger
            \partial\mathbf{t}_2
        }
    \end{array}
\end{equation}
where the intermediates $\mathcal{F}_p^q$ and $\mathcal{G}_{pr}^{qs}$ can be computed using a transformation of the one- and two-electron integrals to the natural spin-orbital basis (see \cref{sec:appendix} for details).
These cumulant derivatives are straightforward to evaluate from \cref{eq:odc12-wavefunction,eq:cumulant-parametrization} using either algebraic or diagrammatic techniques.

Next, let us outline the derivation of the one-body metric.
Substituting \cref{eq:odc12-wavefunction} into \cref{eq:metric-blocks} gives
\begin{equation}
    \label{eq:odc12-s11-metric}
    \begin{array}{r@{\,}l}
        \omega
        \mathbf{S}_{11}
        =
        &
        \left.
            \dfrac{1}{2}
            \dfrac{%
                \partial^2
                \{\langle\Psi|
                    [i\dot{T}_1^\dagger(t), T_1(t)]
                |\Psi\rangle\}
            }{%
                \partial \mathbf{t}_1^\dagger(\omega)
                \partial \mathbf{t}_1(\omega)
            }
        \right|_{f=0}
        \\[20pt]
        &
        -
        \left.
            \dfrac{1}{2}
            \dfrac{%
                \partial^2
                \{\langle\Psi|[T_1^\dagger(t), i\dot{T}_1(t)]|\Psi\rangle\}
            }{%
                \partial \mathbf{t}_1^\dagger(\omega)
                \partial \mathbf{t}_1(\omega)
            }
        \right|_{f=0}
    \end{array}
\end{equation}
where we have assumed that we are working in the variational orbital basis so that
\(
    T_1(t)|_{f=0}
    =
    0
\),
and
\(
    \Psi
    =
    \Psi(t)|_{f=0}
\)
denotes the ground state wavefunction.
Using the Fourier expansion of the $\mathbf{t}_1(t)$ parameters
(\cref{eq:parameter-fourier-expansion}), the gradients of the time derivatives
can be evaluated as:
\begin{equation}
    \label{eq:time_deriv_1}
    \left.
    \frac{%
        \partial i \dot{T}_1^\dagger(t)
    }{%
        \partial \mathbf{t}_1^\dagger(\omega)
    }
    \right|_{f=0}
    =
    -
    \omega
    \mathbf{a}_1^\dagger
    e^{+i\omega t}
\end{equation}
\begin{equation}
    \label{eq:time_deriv_2}
    \left.
    \frac{%
        \partial i \dot{T}_1(t)
    }{%
        \partial \mathbf{t}_1(\omega)
    }
    \right|_{f=0}
    =
    +
    \omega
    \mathbf{a}_1
    e^{-i\omega t}
\end{equation}
Substituting \cref{eq:time_deriv_1,eq:time_deriv_2} into \cref{eq:odc12-s11-metric} and evaluating the gradients
of \(T_1\) and \(T_1^\dagger\) similarly gives the final working equation for
the one-body metric:
\begin{equation}
    \begin{array}{r@{\,}l}
        \omega
        (\mathbf{S}_{11})_{ia,jb}
        &=
        \omega
        \langle\Psi|
        [a_a^i, a_j^b]
        |\Psi\rangle
        \\[10pt]
        &=
        \omega
        (
            \delta^b_a
            \gamma^i_j
            -
            \delta^i_j
            \gamma^b_a
        )
    \end{array}
\end{equation}


\subsection{Linear-Response OLCCD}
\label{sec:olccd}
As we discussed in \cref{sec:dct}, the orbital-optimized linearized coupled cluster doubles method (OLCCD) can be considered as an approximation to the ODC-12 method where all of the non-linear $\mathbf{t}_2$ terms are neglected. Similarly, we can formulate the linear-response OLCCD method (LR-OLCCD) by linearizing the LR-ODC-12 equations. This simplifies the expressions for the electronic Hessian blocks that involve the second derivatives with respect to $\mathbf{t}_2$. For example, for the $\mathbf{A}_{22}$ block, we obtain:
\begin{equation}
    \label{eq:a22_olccd}
    \mathbf{A}_{22}
    =
    (f_0)_i^j
    \dfrac{\partial^2 \lambda^{ir}_{jr}}{%
        \partial\mathbf{t}_2^\dagger
        \partial\mathbf{t}_2
    }
    -
    (f_0)_a^b
    \dfrac{\partial^2 \lambda^{ar}_{br}}{%
        \partial\mathbf{t}_2^\dagger
        \partial\mathbf{t}_2
    }
    +
    \tfrac{1}{4}
    \overline{g}_{pq}^{rs}
    \dfrac{\partial^2 \lambda^{pq}_{rs}}{%
        \partial\mathbf{t}_2^\dagger
        \partial\mathbf{t}_2
    }
\end{equation}
where
\(
    (f_0)_p^q
    =
    h_p^q
    +
    \overline{g}_{pi}^{qi}
\)
is the usual (mean-field) Fock operator.
Comparing \cref{eq:a22_olccd} with \cref{eq:a22_odc12} from the LR-ODC-12
method, we observe that the former equation can be obtained from the latter by
replacing the $\mathcal{F}_p^q$ intermediates with the mean-field Fock matrix
elements and ignoring the term that depends on $\mathcal{G}_{pr}^{qs}$.
These simplifications arise from the fact that the $\mathcal{F}_p^q$ and
$\mathcal{G}_{pr}^{qs}$ intermediates contain high-order $\mathbf{t}_2$
contributions that are not included in the linearized LR-OLCCD formulation (see
\cref{sec:appendix} and Ref.\@ \citenum{Sokolov:2013p024107} for details).
For the $\mathbf{B}_{22}$ block, we find that all of the Hessian elements are
zero.
A complete set of working equations for LR-OLCCD is given in the Supporting
Information.


\section{Computational Details}
\label{sec:comp_details}
The LR-ODC-12 and LR-OLCCD methods were implemented as a standalone Python
program, which was interfaced with \textsc{Psi4}\cite{Parrish:2017p3185} and
\textsc{Pyscf}\cite{Sun:2018pe1340} to obtain the one- and two-electron
integrals.
To compute excitation energies, our implementation utilizes the multi-root
Davidson algorithm,\cite{Davidson:1975p87,Liu:1978p49} which solves the
generalized eigenvalue problem
\eqref{eq:linear-response-energy-eigenvalue-equation} by progressively growing
an expansion space for the \(n_\mathrm{root}\) lowest generalized eigenvectors
of the electronic Hessian and the metric matrix.
A key feature of this algorithm is that it avoids storing the Hessian and metric
matrices, significantly reducing the amount of memory required by the
computations.
Our implementation of the energy Hessian was validated by computing the static
response function for a dipole perturbation (i.e., the dipole polarizability):
\begin{equation}
    \langle\!\langle\hat{V}; \hat{V}\rangle\!\rangle_0
    =
    -
    \mathbf{v}'^\dagger
    \mathbf{E}^{-1}
    \mathbf{v}'
\end{equation}
This quantity can also be evaluated numerically as a derivative of the ground state energy
\begin{equation}
    \langle\!\langle\hat{V}; \hat{V}\rangle\!\rangle_0
    =
    \left.
    \frac{%
        d^2 E
    }{%
        df^2
    }
    \right|_{f=0}
\end{equation}
by perturbing the one-electron integrals
\(
    h_p^q
    \leftarrow
    h_p^q
    +
    f\langle \psi_p|\hat{V}|\psi_q\rangle
\)
and solving the ODC-12 equations for different values of \(f\).


We used \textsc{Q-Chem} 4.4\cite{QChem} to obtain results from
equation-of-motion coupled cluster theory with single and double excitations
(EOM-CCSD) and EOM-CCSD with triple excitations in the EOM part [EOM-CC(2,3)].
The \textsc{MRCC} program\cite{MRCC} was used to obtain results for
equation-of-motion coupled cluster theory with up to full triple excitations
(EOM-CCSDT).
All electrons were correlated in all computations.
We used tight convergence parameters in all ground-state ($10^{-8}$ \hartree)
and excited-state computations ($10^{-5}$ \hartree).
In \cref{sec:two_electron,sec:vert_excit}, the augmented aug-cc-pVTZ and
d-aug-cc-pVTZ basis sets of Dunning and co-workers were
employed.\cite{Kendall:1992p6796}
For alkenes (\cref{sec:alkenes}), the ANO-L-pVXZ (X = D, T) basis
sets\cite{Widmark:1990p291} were used as in Ref.~\citenum{Daday:2012p4441}.
To compute vertical excitation energies in \cref{sec:vert_excit}, geometries of
molecules were optimized using ODC-12 (for LR-ODC-12), OLCCD (for LR-OLCCD), or
CCSD [for EOM-CCSD, EOM-CC(2,3), and EOM-CCSDT].
For alkenes in \cref{sec:alkenes}, the frozen-core MP2/cc-pVQZ geometries were
used as in Refs.~\citenum{Daday:2012p4441} and \citenum{Zimmerman:2017p4712}.

