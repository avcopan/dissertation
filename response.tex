\chapter[%
    Linear-Response Density Cumulant Theory for Excited States:\\
	First Implementation and Benchmark Calculations
]{%
    Linear-Response Density Cumulant Theory for Excited States:\\
	First Implementation and Benchmark Calculations\footnote{%
        A.~V.~Copan and A.~Yu.~Sokolov (to be submitted in
        J.~Chem.~Theory~Comput).
    }
}
\label{ch:response}


\section{Abstract}

We present a linear-response formulation of density cumulant functional theory
(DCT) that provides accurate access to many electronic states.
DCT expresses the electronic energy as a Hermitian, size-extensive, and
stationary functional of the one-particle density matrix and the two-particle
density cumulant.
In the original DCT formulation only the information about a single electronic
state (usually, the ground state) is obtained.
In this research, we combine DCT with linear response theory to obtain
information about many electronic states simultaneously.
We discuss the derivation of linear-response DCT, present its implementation for
the ODC-12 method (LR-ODC-12), and benchmark its performance against highly
accurate equation-of-motion coupled cluster theory with up to full triple
excitations (EOM-CCSDT).
Our results for a set of small molecules demonstrate that LR-ODC-12 vertical
excitation energies are in closer agreement with EOM-CCSDT than those obtained
from equation-of-motion coupled cluster theory with up to double excitations
(EOM-CCSD).
In addition, we report a linear-response formulation of the orbital-optimized
linearized coupled cluster theory with double excitations (LR-OLCCD), which we
obtain by neglecting the non-linear terms in the LR-ODC-12 equations.


\section{Introduction}

Accurate simulation of excited electronic states remains one of the major
challenges in modern electronic structure theory. 
{\it Ab initio}\/ methods for excited states can be divided into
single-reference and multi-reference categories, based on their ability to
treat static electron correlation.
Multi-reference methods can correctly describe static correlation in
near-degenerate valence orbitals and electronic states with multiple-excitation
character, but often lack accurate treatment of important dynamic correlation
effects (e.g., multi-configurational self-consistent field or multi-reference
perturbation theories)
\cite{Knowles:1985p259,Werner:1985p5053,Wolinski:1987p225,Hirao:1992p374,Finley:1998p299,Andersson:1990p5483,Andersson:1992p1218,Werner:1996p645,Angeli:2001p10252,Angeli:2001p297}
or become very costly when the number of near-degenerate orbitals is large
(e.g., multi-reference configuration interaction or coupled cluster theories).
\cite{Mukherjee:1977p955,Lindgren:1978p33,Siegbahn:1980p1647,Jeziorski:1981p1668,Werner:1988p5803,Mahapatra:1998p157,Mahapatra:1999p6171,Pittner:2003p10876,Evangelista:2007p024102,Datta:2011p214116,Evangelista:2011p114102,Kohn:2013p176,Nooijen:2014p081102}
Meanwhile,
single-reference methods
\cite{Foresman:1992p135,Sherrill:1999p143,Geertsen:1989p57,Comeau:1993p414,Stanton:1993p7029,Krylov:2008p433,Crawford:2000p33,Shavitt:2009,Sekino:1984p255,Koch:1990p3345,Koch:1990p3333,Nooijen:1997p6441,Nooijen:1997p6812,Nakatsuji:1978p2053,Nakatsuji:1979p329}
often provide a compromise between the computational
cost and accuracy, and can be used to reliably compute properties of molecules
in low-lying electronic states near the equilibrium geometries. In these
situations, single-reference equation-of-motion coupled cluster theory
(EOM-CC)
\cite{Geertsen:1989p57,Comeau:1993p414,Stanton:1993p7029,Krylov:2008p433,Crawford:2000p33,Shavitt:2009}
is usually the method of choice, especially when high accuracy
is desired. 

The EOM-CC methods yield size-intensive excitation energies
\cite{Koch:1990p3345,Koch:1990p3333}
and can be
systematically improved by increasing the excitation rank of the cluster
operator in the exponential parametrization of the wavefunction. Although EOM-CC
is usually formulated in the context of a similarity-transformed Hamiltonian,
its excitation energies are equivalent to those obtained from linear-response
coupled cluster theory (LR-CC).
\cite{Sekino:1984p255,Koch:1990p3345,Koch:1990p3333}
Both EOM-CC and LR-CC are based on non-Hermitian eigenvalue problems,
complicating the computation of molecular properties (e.g., transition dipoles)
by requiring evaluation of left and right eigenvectors.
\cite{Stanton:1993p8840,Stanton:1994p4695,Stanton:1994p8938,Levchenko:2005p224106}
% may result in an incorrect description of potential energy surfaces in the
% vicinity of conical intersections where complex excitation energies may be
% obtained.  \cite{Hattig:2005p37,Kohn:2007p044105,Kjonstad:2017p164105}
Several Hermitian alternatives to EOM-CC and LR-CC have been
proposed to avoid these problems, such as  
algebraic diagrammatic construction
\cite{Schirmer:1982p2395,Schirmer:1991p4647,Schirmer:2004p11449,Harbach:2014p064113,Dreuw:2014p82}, 
unitary and variational LR-CC,
\cite{Taube:2006p3393,Kats:2011p062503,Walz:2012p052519}
similarity-constrained CC,
\cite{Kjonstad:2017p4801}
and propagator-based LR-CC.
\cite{Moszynski:2005p1109,Korona:2010p14977}

In this work, we present the development of linear-response density cumulant
functional theory (LR-DCT), a size-intensive approach for excited
electronic states. In density cumulant functional theory (DCT),
\cite{Kutzelnigg:2006p171101,Simmonett:2010p174122,Sokolov:2012p054105,Sokolov:2013p024107,Sokolov:2013p204110,Sokolov:2014p074111,Wang:2016p4833}
the
electronic energy is obtained by optimizing the energy functional directly in
terms of the one-particle reduced density matrix and the two-particle density
cumulant, a fully connected part of the two-particle reduced density matrix
(2-RDM).
\cite{Fulde:1991,Ziesche:1992p597,Kutzelnigg:1997p432,Mazziotti:1998p419,Mazziotti:1998p4219,Kutzelnigg:1999p2800,Ziesche:2000p33,Herbert:2007p261,Kong:2011p214109,Hanauer:2012p50}
In this regard, DCT is related to approaches that are based on
the variational optimization
\cite{Colmenero:1993p979,Nakatsuji:1996p1039,Mazziotti:1998p4219,Mazziotti:2006p143002,Kollmar:2006p084108,DePrince:2007p042501,DePrince:2016p164109}
or parametrization
\cite{Mazziotti:2008p253002,Mazziotti:2010p062515,DePrince:2012p1917}
of 2-RDM\@. On
the other hand, DCT has a close relationship
\cite{Sokolov:2013p024107,Sokolov:2013p204110}
with wavefunction-based electronic
structure theories, such as linearized, unitary, and variational coupled
cluster theory.
\cite{Kutzelnigg:1991p349,Kutzelnigg:1998p65,VanVoorhis:2000p8873,Kutzelnigg:1982p3081,Bartlett:1989p133,Watts:1989p359,Szalay:1995p281,Cooper:2010p234102,Evangelista:2011p224102}
In contrast to variational 2-RDM theory
\cite{Nakata:2009p042109,vanAggelen:2010p114112,Verstichel:2010p114113}
and traditional
coupled cluster methods [e.g., CCSD and CCSD(T)],
\cite{Crawford:2000p33,Shavitt:2009}
DCT naturally combines
size-extensivity and a Hermitian energy functional. In addition, the DCT
electronic energy is fully relaxed with respect to all of its parameters, which
greatly simplifies computation of the first-order molecular properties.
\cite{Scheiner:1987p5361,Salter:1989p1752,Gauss:1991p2623,Gauss:1991p207}
We have successfully applied DCT to a variety of chemical systems with
different electronic structure effects (e.g., open-shell, symmetry-breaking,
and multi-reference).
\cite{Sokolov:2013p204110,Sokolov:2014p074111,Wang:2016p4833,Copan:2014p2389,Mullinax:2015p2487}
One limitation of the original DCT formulation is
that it can only obtain information about the lowest-energy state of a
particular symmetry (usually, the ground state). By combining DCT with linear
response theory, we remove this limitation, providing access to many electronic
states simultaneously.

We begin with a brief overview of DCT (\cref{sec:dct}) and linear response
theory (\cref{sec:lr}). We then discuss the derivation of linear-response
theory for the ODC-12 method (LR-ODC-12, \cref{sec:lr_odc12}).
% and the details of its implementation (\cref{sec:implementation}). 
In section \cref{sec:olccd}, 
we derive equations for the linear-response orbital-optimized linearized
coupled cluster theory with double excitations (LR-OLCCD), which we obtain by
neglecting the non-linear terms in the LR-ODC-12 equations. 
We outline the
computational details in \cref{sec:comp_details}.
In section \cref{sec:results}, we demonstrate that the LR-ODC-12 excitation
energies are size-intensive (\cref{sec:size_intensivity}), test the performance
of LR-ODC-12 for the dissociation of \ce{H2} (\cref{sec:two_electron}), and
benchmark the accuracy of LR-ODC-12 for vertical
excitation energies of small molecules (\cref{sec:vert_excit}).
Finally, we present our conclusions in section \cref{sec:conclusions}. 
