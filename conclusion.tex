\chapter{Conclusion}
\label{ch:conclusion}

We have presented comprehensive benchmarks to demonstrate that orbital-optimized
density cumulant theory with double excitations (ODC-12) consistently
outperforms coupled-cluster theory with singles and doubles (CCSD) for
the description of noncovalent interactions, hydrogen-transfer barrier
heights, radical stabilization energies, ionization energies, and covalent bond
stretching.
Having established the promising performance of this model for ground state
calculations, we have extended the theory for the calculation of excitation
energies and transition properties through the use of linear response theory.
After numerically demonstrating that our initial working equations and
implementation are correct, we have empirically shown this method to be
size-intensive, i.e.\ displaying the correct qualitative behavior with respect
to excited states of independent systems.
Next, we have shown that this method is more stable with respect to strong
electron correlation than its linearized variant, LR-OLCCD, which often achieves
impressive error cancellation in the absence of strong correlation.
This demonstrates that the infinite order one-particle \(n\)-representability
conditions defining the ODC-12 method contribute to a robust description of the
electron distribution for more challenging states.
Our initial benchmark study of the vertical excitation energies predicted by
this method shows that it reduces the mean absolute error by roughly a factor of
two relative to the popular equation-of-motion coupled-cluster with singles and
doubles (EOM-CCSD) method, similar to our findings for ground states.
For well-behaved systems we find that the linearized model, LR-OLCCD, is an
effective approximation to LR-ODC-12 with a lower cost prefactor.
Finally, we develop some improvements to the algorithms used for solving the
LR-ODC-12 equations using disk-based direct matrix algorithms (variants of the
Davidson algorithm).
These developments allow us to study polyene systems as large as hexatriene with
a natural orbital basis of double-zeta quality.
This calculation involves 44 electrons and 124 spatial orbitals withnearly 20
million unique wavefunction parameters, which would not be feasible without the
new algorithms.
The advantages of LR-ODC-12 over EOM-CCSD and LR-OLCCD for these polyene systems
are even more stark than for our previous benchmarks.
Whereas EOM-CCSD overestimates the energy of the challenging
\termsymbol{2{}^1A_g} state of hexatriene and its gap with the neighboring
\termsymbol{1{}^1B_u} state by close to 1~\eV each, LR-ODC-12 matches its energy
to within 0.15~\eV and matches the energy gap to within 0.01~\eV.
Given the relative sparsity of inexpensive alternatives to EOM-CCSD, we believe
that these results merit further development of algorithms for the LR-ODC-12
method to expand our toolkit for studying excited electronic states.

